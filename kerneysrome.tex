\documentclass[openany,a4paper]{memoir}

\begin{document}

\chapter{
BOOK lY. 
ROME. 

CHAPTER I. 

FROM THE FOUNDATION OF THE CITY TO THE EXPULSION OF TARQUIN, THE LAST KING OF ANCIENT ROME. 
—B. C. 752 TO 509. 

THE early history of this celebrated empire, like that of 
the other nations of antiquity, is greatly involved in 
obscurity. But the history of Rome is properly the his- 
tory of a City, which gradually extended its imperial sway, 
first over all Italy, then over all the countries bordering on 
the Mediterranean Sea. According to the account of the 
poets, ^'?ieas, a Trojan prince, having escaped from the de- 
struction of his native place, after a variety of adventures, 
landed on the shores of Italy, where he was kindly received 
by Lati'nus, king of the Latins, who gave him his daughter 
Lavin'ia in marriage, and made him heir to his throne. The 
succession continued in the family of ^neas for about four 
hundred years, until the reign of Nu'mitor, who was the 
fifteenth king in a direct line from the Trojan hero. 

2. Rhe'a Syl'via, the daughter of Niimitor, was the mother 
of twin brothers, named Roin'uliis and Be'mns. The mother, 
who had been a vestal virgin, was condemned to be buried 
alive, the usual punishment for vestals who had suffered a 
violation of their chastity, and the twins were ordered to 
be thrown into the Tiber. But as the water into which they 
were cast was too shallow to drown them, they were discov- 
ered and rescued from their perilous situation l3y Faus'tulus, 
the king's herdsman, who brought them up as his own chil- 
dren. After a variety of adventures Romulus and Remus, 

Chapter I. — 1. What is said of the early history ? What account do 
the poets give of Jilneas ? — 2. What is said of Khea Sylvia ? To what 
was the mother condemned ? What is said of the twins ? By whom 
were they found ? What is related of Romulus and Remus ? 

82 



ROME. 83 

we are told, were instrumental in restoring Numitor, their 
grandfather, to his throne, from which he had been expelled 
by the usurpation of his brother, Aumulius. 

3. Subsequent to this event the two brothers resolved 
to build a city on the hills where they had passed their 
youth, and formerly tended their flocks ; but a contest arose 
between them relative to the sovereignty, which proved fatal 
to Remus. It is related that he was killed by his brother, 
who struck him dead on the spot, for contemptuously leaping 
over the city wall. 

Romulus being thus left the sole commander, persevered 
in the building of the city, which, from his own name, he 
called Rome. It was founded b. c. 152. Having been 
chosen the first King, Romulus made it the asylum for 
fugitives, and by this means the number of inhabitants rap- 
idly increased. 

4. The newly elected monarch is said to have divided the 
people into three Tribes, each consisting of ten Cu'i'iae ; and 
also into two orders of Patri'cians and Plebe'ians. The Sen- 
ate consisted of one hundred of the principal citizens ; it was 
afterwards increased to two hundred members. Besides a 
guard of three hundred men to attend his person, the king was 
always preceded by twelve Lictors, armed with axes bound up 
in a bundle of rods. The duty of the lictors was to execute the 
laws. These wise regulations contributed daily to increase 
the strength of the new city ; multitudes flocked to it from 
the adjacent towns, and women only were wanted to confirm 
its growing prosperity. Romulus, in order to supply this 
deficiency, invited the Sahi'nes, a neighboring nation, to a 
festival in honor of Neptune ; and while the strangers were 
intent upon the spectacle, a number of the Roman youth 
rushed in among them, and seized the youngest and most 
beautiful of the women, and carried them off by violence. 

5. A sanguinary war followed, which had brought the 
city almost to the brink of ruin, when an accommodation 
was happily effected through the interposition of the Sabine 
women who had been carried off by the Romans. Romulus 
reigned thirty-seven years, and after his death received divine 
honors, under the name of Quiri'nus. 

3. What did they resolve to do? What arose? What is related? 
What did Eomulus now do ? — 4. How did he divide the people ? Of 
what did the senate consist? By what was the king attended? To 
what did these regulations contribute ? What were wanted ? How was 
this deficiency supplied ? — 5. What followed ? How long did Eomulus 
reign ? 



84 HOME. 

6. On the death of Romulus, Nu'ma PompiKius, a native 
of Cures, a Sabine city, was elected the second King of Rome. 
He softened the fierce and warlike disposition of the Romans, 
by cultivating the arts of peace and inculcating obedience 
to the laws and respect for religion. He built the Temple of 
Ja'nus, which was to be open during war, and shut in time 
of peace. He died at the age of eighty, after a reign of forty- 
three years. 

t. TitVlus Hosti'lius was the third King of Rome. His 
reign is memorable for the combat between the Hora'tii and 
Curia'tii, which is said to have taken place during a war 
against the Albans. There were, at the time, in each army, 
three brothers of one birth ; those of the Romans called the 
Horatii, and those of the Albans, the Curiatii, all six remark- 
able for their strength, activity, and courage. To these it 
was resolved to commit the fate of the two parties. Finally, 
the champions met in combat. The contest was for some 
time obstinate and doubtful ; but victory at length declared 
in favor of Rome. The three Curiatii were slain, and only one 
of the Horatii survived. By this victory the Romans became 
masters of Alba. Hostilius died after a reign of thirty-two 
years. 

8. After the death of the late monarch. Aniens Mar'cius, 
the grandson of Niima was elected the fourth King of Rome. 
He conquered the Latins, and suppressed the insurrections 
of the Yien'tes, Fidina'tes, and YoFsci. But his victories 
over his enemies were far less important than his exertions 
in fortifying and embellishing the city ; he erected a prison 
for malefactors, and built the port of Ostia, at the mouth of 
the Timber. Ancus died in the twenty-fourth year of his reign. 

9. Tarqui'nius Pris'cus, or Tar'quin the elder, the son of 
a merchant of Corinth, next succeeded to the throne. His 
reign is chiefly distinguished for his triumph over the Sabines 
and Latins, and by the embellishment of the city with works 
of utility and magnificence. He built the walls of hewn 
stone, erected the circus, founded the capitol, and constructed 
the sewers or aqueducts for the purpose of draining the city 
of its rubbish and superfluous waters. Tarquin was assassi- 

6. Who succeeded? What did he do? What was his age? How 
long did he reign ? — 7. Y/ho was the third king of Eome ? For what 
is his reign memorable ? Relate the circumstances of this combat ? — 
8. Who Avas the fourth king of Rome ? Whom did he conquer? What 
did he erect ? When did he die ?— 9. Wlio next succeeded to the throne ? 
For what was his reign distinguished ? What did he build ? How did 
he die ? 



ROME. 85 

. nated in the fifty-sixth year of his age, and in the thirty-eighth 
of his reign. 

10. Ser'vius TuVlius, who was the son of a female slave, 
and son-in-law of the late monarch, secured his election to 
the throne through the intrigues of Tanaquil, his mother-in- 
law. In order to determine the increase or diminution of 
his subjects, he instituted the census, by which, at the end of 
every fifth year, the number of the citizens, their dwellings, and 
the amount of their property were ascertained. The census 
was closed by an expiatory sacrifice, called a lustrum ; hence 
the period of five years was usually called a lustrum. 

11. Servius, in the early part of his reign, had married his 
two daughters to the two sons of Tarquin, the late King, 
whose names were Tarquin and Aruns. But as their dis- 
positions corresponded with those of his daughters, he took 
care to give Tullia, the younger, who was of a violent dis- 
position, to Aruns, who was mild, and the elder to Tarquin, 
who was haughty and ambitious, hoping thereby that they 
w^ould correct each other's defects. Tarquin and Tullia, 
however, murdered their consorts, and were shortly after- 
w^ards married ; and as one crime often produces another, 
they caused the assassination of Servius, after which Tarquin 
usurped the throne. Tullia, in her eagerness to salute her 
husband as King, is said to have driven her chariot over the 
dead body of her father, which lay exposed in the street that 
led to the senate. Thus died Servius TulHus, after a useful 
and prosperous reign of forty-four years. 

12. Tarquin, surnamed the Proud, having placed himself 
upon the throne, as we have seen, soon disgusted the people 
by his tyranny and cruelty. He refused the late King's 
body a burial, under the pretence of his having been a usurper, 
and, conscious of being hated by all virtuous persons, he 
ordered all those whom he suspected to have been attached 
to Servius to be put to death. 

To divert the attention of the people from his illegal 
method of obtaining the crown, he kept them constantly em- 
ployed either in wars or in erecting public buildings. While 
besieging Ardea, a small town not far from Rome, Sextus, 
his son, left the camp to visit the house of Collati'nus, under 
the mask of friendship. He was kindly received by the vir- 

10. Who succeeded to the throne? What did he institute? — 11. What 
is related of his two daujyhters ? How did Servius die ? Who succeeded 
to the throne? What did Tullia do in her eagerness to salute her hus- 
band as king?— 12. What did Tarquin refuse? What did he order? 
What is related of Sextus ? 



86 ROME. 

tuous Lucre'tia, the wife of Collati'nus, who did not in the 
least suspect his crimioal design. 

13. At midnight, however, the princely ruffian entered her 
chamber with a drawn sword in his hand, and threatened 
her with instant death if she offered to resist. Lucretia, 
though seeing death so near, was yet inexorable, until being 
told if she did not yield he would first kill her, and then lay- 
ing his own slave dead by her side would report that he 
found and killed them both in a criminal act. 

Thus the terror of infamy achieved what death itself could 
not obtain. In the meantime Lucretia, resolving not to 
pardon herself even for the crime of another, sent for her 
husband, CoUatinus, and Spu'rius, her father, who brought 
with them Junius Brutus, the reported idiot, whom they ac- 
cidentally met in the way. They found her overwhelmed 
with grief, and endeavored in vain to console her. "No, 
never," she replied ; ''never shall I find anything in this 
world worth living for, after having lost my honor ;" and 
drawing a poignard from beneath her robe she plunged it 
into her own bosom, and expired without a groan. 

14. The body of Lucretia was brought out and exposed to 
view in the public forum, where Brutus, who had hitherto 
acted as an idiot in order to elude the cruelty of Tarquin, 
inflamed the ardor of the citizens by displaying the horrid 
transaction. He obtained a decree of the senate that Tar- 
quin and his family should be forever banished from Rome ; 
at the same time making it a capital offence for any one to 
plead for his return. That monarch was accordingly ex- 
pelled from his kingdom, in the twenty-fifth year of his reign, 
and the regal government was abolished, after it had lasted 
for two hundred and forty-four years. 

13. At midnight what did he do? What did he threaten? How did 
CoUatinus and Spurius find Lucretia ? What reply did she make them ? 
How did she die?— 14. What did Brutus do? What did he obtain? 
How long had the regal government continued ? 



ROME. 87 



CHAPTER II. 

ROMEAS A REPUBLIC. 

FROM THE ABOLITION OF THE REGAL POWER TO THE 
FIRST PUNIC WAR.—B. C. 509 TO 449. 

THE regal authority having been abolished, a republican 
form of government was established on its ruins. The 
supreme power was still reserved to the Senate and peo- 
ple, but instead of a King, two magistrates, called Consuls, 
were annually chosen, with all authority, privileges, and en- 
signs of royalty. Brutus, the deliverer of his country, and 
Collatimis, the husband of Lucretia, were chosen the first 
Consuls of Kome. 

2. But scarcely had the new republic began to exist when 
a conspiracy was formed for its destruction. Some young 
men of the principal families of the state, who had been edu- 
cated near the King, and had shared in all the luxuries and 
pleasures of the court, formed a party in Rome in favor of 
Tarquin, and undertook to reestablish the monarchy. Their 
design was fortunately discovered before it could be carried 
into execution ; and, surprising as it may appear, the two 
sons of Brutus were found among the number of the conspira- 
tors. Few situations could be more affecting than that of 
Brutus, — a father and a judge. He was impelled by justice 
to condemn ; by nature to spare the children he loved. 

Being brought to trial before him, they were condemned 
to be beheaded in his presence, while the father beheld the 
sad spectacle with unaltered countenance. He ceased to be 
a father, as it has been beautifully observed, that he might 
execute the duties of the consul, and chose to live bereft of 
his children rather than to neglect the public punishment of 
crime. 

3. The insurrection in the city being thus suppressed, Tar- 
quin now resolved to regain his former throne by foreign as- 
sistance, and, having prevailed upon the Vientes to aid him, 
advanced towards Rome at the head of a considerable army ; 
but he was defeated by the Romans, under the command of 

Chapter II. — 1. The regal power being abohshed, what was estab- 
lished ? What two magistrates were chosen ? Who were the first two 
consuls ? — 2. What is said of the republic ? Who were found among the 
conspirators ? What was their fate ? What is said of Brutus ? — 3. What 
did Tarquin now resolve ? By whom was he defeated ? 



88 ROME. 

the two Consuls, Brutus . and Yale'rius. The latter had 
been elected in the place of Collatinus. But while the Ro- 
mans rejoiced in the victory they obtained, they had to la- 
ment the death of Brutus, who fell in the engagement, and 
the Roman matrons honored his memory by wearing mourn- 
ing for a whole year. Valerius returned to the city, and 
was the first Roman who enjoyed the honor of a triumph. 

4. In the meantime, Tarquin, undaunted by his misfor- 
tunes, prevailed upon Porsen'na, one of the kings of Etru'ria, 
to espouse his cause, and in conjunction with him marched 
directly to Rome, and laid siege to the city. This war is 
signalized by the daring intrepidity of Hora^tius Coc'les, 
who alone resisted the whole force of the enemy at the head 
of a bridge which led across the Tiber, and also by that of 
Mutiiis Scaev^olo, who entered the enemy's camp with a de- 
sign to assassinate Porsenna, but, mistaking the monarch, 
killed the secretary, who sat by his side. On Porsenna's 
demanding who he was, Mutius, without reserve, informed 
him of his country and his design, and by way of punish- 
ment of the hand which had missed its aim, he thrust it into 
the fire which was burning upon the altar before him. For- 
senna, admiring this noble intrepidity, offered conditions of 
his peace to the Romans on honorable terms. 

5. Tarquin having induced the Latins to enlist in his 
cause, for a third time approached the city with his army. 
But while a public enemy threatened them from without, 
domestic disorders prevailed within the walls of the city. 
The plebeians, who were poor and oppressed with debt, re- 
fused to aid in repelling the enemy unless their debts were 
remitted on their return, and as the Valerian law gave to 
any condemned citizen the right of appealing to the people, 
the consuls found their authority of no avail. 

6. In this state of things an extraordinary measure was 
necessary. A new magistrate was created, styled Dictator, 
who should continue in office only as long as the danger of the 
state required, and whose power was absolute, not only over 
all ranks of the state, but even over the very laws, with 
which he could dispense in cases of public exigency, without 
consulting the senate or the people. Titus Largius, one of 
the Consuls, being" elevated to the office of Dictator, collected 

What had the Eomans to lament ? — 4. In the meantime what did Tar- 
quin do ? By what is this war signalized ? What is related of Horatius 
and Mutius? — 5, What is said of "Tarquin? What did the plebeians re- 
fuse ?— 6. What new magistrate waa created ? What was his power ? 
Who was the first dictator ? 



ROME. 89 

an army, and having restored tranquillity to the state, re- 
signed the dictatorship before the expiration of six months, 
with the reputation of having exercised it with justice and 
moderation. 

Shortly -after this event, war again was excited by the 
Tarquins ; in this emergency, Posthu'mius was appointed 
dictator; the Romans were completely victorious, and the 
sons of Tarquin were slain. 

T. On the return of peace, Rome was again disturbed by 
domestic dissensions ; the dispute between the creditors and 
debtors was again renewed. The plebeians, despairing of 
being able to affect a redress of their grievances in Rome, 
resolved to move and form a new establishment without its 
limits. Accordingly, under the conduct of a plebeian, called 
Sicin'ius Bellu'tus, they retired to a mountain called Mons 
Sacer, on the banks of the river Anio, about three miles 
from Rome. 

8. At the news of this defection the Senate grew alarmed, 
and immediately deputed ten of the most respectable of their 
body, with authority to grant a redress. Men'enius Agrip'- 
pa, one of the ten commissioners, eminent for his virtue and 
wisdom, is said to have effected a reconciliati-on by relating 
the celebrated fable of the disagreement between the stomach 
and the other members of the human body. The applica- 
tion of the fable was so obvious, that the people unanimously 
cried out that Agrippa should lead them back to Rome. 
Before their departure, however, it was proposed by Lucius 
Junius that for their future security a new order of magis- 
trates should be created, who should have the power of an- 
nulling, by a single vote, any measure which they should 
deem prejudicial to the interests of the people. Those magis- 
trates, called Tribunes, were annually elected ; their number, 
which at first was five, afterwards increased to ten. By this 
measure the aristocracy was restrained and the fury of the 
populace checked. At the same time two magistrates, styled 
JEdiles, were appointed, whose duty it was to assist the tri- 
bunes and take charge of the public buildings. 

9. During the late separation, agriculture having been 
neglected, a famine was the consequence the following season ; 

What was the fate of the sons of Tarquin? — 7. What dispute was 
accain renewed ? What did the plebeians resolve to do ? — 8. At the nev/s 
of this defection what did the senate do ? What is related of Agrippa? 
For their future security what was done ? W^hat were tliese magistrates 
called ? At the same time what other two magistrates were appointed ? 
— 9. During the separation vrhat was neglected? What follovred? 



90 ROME. 

but the timelj arrival of a large quantity of corn from Sicily- 
prevented the evil consequences that were likely to ensue. 
At this time the resentment of the people was strongly ex- 
cited against CorioWnus, who insisted that the corn should 
not be distributed until the grievances of the Senate yvgvq 
removed ; for which proposition he was summoned by the 
Tribunes to a trial before the people, and was condemned to 
perpetual banishment. He retired to the Volsci, and being 
appointed to the command of their army, he invaded the 
Roman territories and carried his devastations to the very 
walls of the city ; but he was at length prevailed upon, by 
the earnest entreaties of his mother and his wife, to with- 
draw his forces. 

10. The proposal of the Agra'rian law, which had for its 
object the division of the land obtained by conquest equally 
among the people, proved a source of discord between the 
plebeians and patricians ; for, w^hile the former repeatedly 
urged the measure, the latter as often strenuously opposed 
the design. The state was, in consequence, thrown into vio- 
lent dissensions. Through the influence of the tribune, 
VoVero, a law was passed that the election of the tribunes 
should be made in the comitia, or public meetings of the 
people. By this law the supreme authority was taken from 
the patricians and placed in the hands of the plebeians, and 
the Roman government became a democracy. 

11. During the dissensions which grew out of the proposi- 
tion for the Agrarian law, Quinc'tius Cincinna'tus, a man 
eminent for his wisdom and virtue, and who had retired from 
pubhc life, was created Dictator ; but scarcely had he restored 
tranquillity to the state and resigned his office, than new 
dangers obliged him a second time to resume it. The ^qui, 
having invaded the territory of the Romans, enclosed the 
army of the Consul Minutius, who had been sent to oppose 
them, in a defile between two mountains, from which there 
was no egress. Cincinna'tus, having raised another army, 
placed himself at its head, and having defeated the ^qui, 
and rescued the army of the Consul from their perilous situa- 
tion, returned in triumph to the city, and, after holding the 
high office of Dictator only for the space of fourteen days, he 
resigned its honors and again retired to labor on his farm. 

Against whom was the resentment of the people excited? To what 
was he sentenced ? Where did he retire ? \Vhat is related of him ? — 
10. What Avas a source of discord between tlie plebeians and patricians? 
What law was passed? What was the nature of this law?— 11. Who 
at this time was created dictator ? What is said of Cinciimatus ? 



ROME. 91 

12. Previous to this period the Romans had not possessed 
any written body of laws. Under the regal government 
the monarch administered justice, and the Consuls who suc- 
ceeded them exercised the same authority. But their arbi- 
trary decisions were frequently the subject of complaint, and 
all ranks of the citizens became desirous of having a fixed 
code of laws for the security of their rights. Three com- 
missioners were accordingly sent to collect, from the most 
civilized states of Greece and Italy, such laws as were deemed 
useful in forming a suitable code. 

13. On the return of the commissioners, ten of the princi- 
pal senators, called Decemvirs, were appointed to digest a 
body of laws, and were invested with absolute power for one 
year. This gave rise to those celebrated statutes, distin- 
guished by the name of the Laias of the Twelve Tables, whioh. 
formed the basis of the Roman jurisprudence, and continued 
to be held in the greatest esteem during the most flourish- 
ing period of the republic. Those laws manifest the stern 
spirit of the people, and were marked by their severity. 
Nine crimes were punishable with death, one of which was 
parricide ; but, to the honor of the Romans, it may be ob- 
served that this crime was unknown among them for more 
than five hundred years after the foundation of the city. 

14. The Decemvirs, during the first year of their power, 
governed with equity and moderation, each in his turn pre- 
sided for a day, and exercised the sovereign authority. At 
the expiration of the term for which they were appointed, 
under a pretence that some laws were yet wanting to com- 
plete the code, they entreated the Senate to allow them 
further time, for, having experienced the charms of power, 
they were unwilling to retire. Soon they threw off the mask 
of moderation, and, regardless of the approbation either of 
the Senate or the people, resolved to continue in decemvirate. 
A conduct so notorious produced general discontent, and 
their flagrant abuse of power brought a speedy termination 
of their office. 

15. While the army was encamped about ten miles from 
Rome, during a war with the Sabines and Volsci, Ap'pius, 
one of the leading members of the decemvirate, who re- 

12. What had the Romans never possessed? For what were the com- 
missioners sent to Greece? — 13. On their return what was done? To 
what did this give rise ? What do these laws manifest ? How many 
crimes were punishable with death ? — 1 4, What is said of the decem- 
virs? What did they entreat? What did they throw off?— 15. What 
is related of Appius ? 



92 ROME. 

mained in the city, appointed Sicin'ius Denta'tus, a Tribune, 
who, on account of his extraordinary valor and exploits, was 
called the Roman Achilles, legate, and put him at the head 
of the supplies which were sent to reinforce the army in the 
field. On his arrival in the camp he was appointed at the 
head of a hundred men, to discover a more commodious place 
for encampment, as he had very candidly assured the com- 
manders that their present situation was badly chosen. The 
soldiers, however, who composed his escort, were assassins, 
and had engaged to murder him. With this view they led 
him into the hollow of a mountain, where they intended to 
put into execution their design. Dentatiis now perceived, 
when too late, the treachery of the Decemvirs, but resolving 
to sell his life as dearly a? possible, he put his back against 
a rock, and defended himself with so much bravery that he 
killed no less than fifteen, and wounded thirty of his assail- 
ants before they were able to accomplish their design. 

16. Another transaction, equally atrocious, inspired the 
citizens with a resolution to break all measures of obedience. 
While Ap'jjius, who remained in the city, was seated on his 
tribunal dispensing justice, he saw a young lady of exquisite 
beauty, named Virginia, passing to one of the public schools, 
attended by her governess. Her charms, heightened by that 
modest glow which innocence and virtue lend to nature, in- 
flamed his wicked heart ; but being himself unable to gratify 
his desires, he employed a profligate dependent to claim her 
as his own, on the pretence of her being the daughter of one 
of his female slaves. The claim being referred to his tribu- 
nal, Appius pronounced an infamous sentence, by which the 
innocent victim was torn from the embraces of her parents 
and placed within the reach of his own power. 

17. In the mean time Virginius, the young lady's father, 
did all that a parent could to save the liberty and honor of 
his daughter, but, finding that all was over, asked permission 
to take his last farewell of one whom he had so long con- 
sidered as his child. With this Appius complied, on condi- 
tion that their endearments should pass in his presence. 
Virginius, with the most poignant anguish, took his almost 
expiring daughter in his arms, for a while supported her 
head upon his breast, and wiped away the tears that rolled 

What were the soldiers who formed the escort of Dentatus ? How 
many did he kill and wound? — 16. What other transaction is related 
of Appius while seated on his tribunal ? What sentence did he pro- 
nounce ? — 17. In the mean time, what did Virginius do ? What did he 
ask? 



ROME. 98 

down her lovely face, then seizing a knife that lay on one of 
the shambles in the forum, he addressed his daughter, say- 
ing : '* My dearest child — this, this only can preserve your 
freedom and your honor " Thus saying, he buried the 
weapon in her breast, then holding it up, reeking from her 
wound, he exclaimed, " By this blood, Appius, I devote thy 
head to the infernal gods." He then ran through the city 
wildly calling on the people to strike for their freedom, and 
thence to the camp to spread the flame of liberty throughout 
the army. Ap'pius and Op'pius died by their own hands in 
prison ; their colleagues were driven into exile, and the De- 
cemvirate was abolished, after it had continued for three 
years. The Consuls were again restored. 

18; Unfortunately for Rome, there always appeared some 
cause left for internal dissensions. By an early law of the 
state, plebeians were prohibited to intermarry with the patri- 
cians, and the office of Consul was limited to the latter. 
After a long contest, the law prohibiting intermarriage was 
repealed. This concession, it was hoped, would satisfy the 
people, but it only stimulated them to urge their claim to be 
admitted to have a share in the consulship ; and on the oc- 
currence of war, they refused to enroll their names unless 
their demand was granted. At length it was agreed on both 
sides that, instead of the Consuls, six military tribunes 
should be chosen, three from the patricians and three from 
the plebeians. But this institution was soon cast aside, and 
the Consuls were again restored. 

19. The Consuls being thus restored, in order to lighten 
the weight of their duties two new magistrates were created, 
styled Censors, to be chosen ever}^ fifth year. Their duty 
was to estimate the number and the estates of the people, to 
distribute them into their proper classes, to inspect the morals 
and manners of their fellow-citizens. The office was one of 
great dignity and importance, and was exercised for nearly 
one hundred years by the patricians, afterwards by men of 
consular dignity, and finally by the Emperors. 

20. The Senate, in order to avoid the evils which fre- 
quently arose from the people's refusing to enlist in the 
army, adopted the wise expedient of giving a regular pay to 

How did he address his daughter? What did he then do? What 
was the fate of Appius and Oppius? 18. By a law of the state, what 
were the plebeians prohibited ? What was done after a long contest ? 
At length, what was agreed on both sides? — 19. What two new magis- 
trates were now created ? What was their duty ? — 20. What wise ex- 
pedient did the Senate adopt ? 



94 ROME. 

the troops. From this period, the Roman system of war 
assumed a new aspect. The Senate had the army under its 
immediate control ; the enterprises of the republic were more 
extensive, and its success more signal and important. As the 
art of war now became a profession, iu stead of an occasional 
employment, it was in consequence greatly improved, and 
from this period the Roman territory began rapidly to extend. 

21. The inhabitants of the city of Veii had repeatedly 
committed depredations on the Roman territories. It was 
at length decreed by the Roman Senate that Yeii should be 
destroyed, whatever it might cost. Accordingly, a siege was 
commenced, which continued with varied success for ten 
years. At length, in order to give greater vigor to the oper- 
ations, Camil'lus was created Dictator, and to him was en- 
trusted the sole management of the long protracted war. He 
caused a passage to be opened under ground, which led into 
the very citadel, and giving his men directions how to enter 
the breach, the city was taken and destroyed. Camillus was 
honored with a splendid triumph, in which his chariot was 
drawn by four white horses ; but being afterwards accused 
of having appropriated a part of the plunder of Yeii to his 
own U3e,"indignant at the ingratitude of his countrymen, he 
went into voluntary banishment. 

22. It was not long before the Romans had cause to re- 
pent of their injustice towards the only man who was able 
to save their country from ruin. 

The Gauls, a barbarous and warlike people, having crossed 
the Alps into the northern part of Italy, under Bren'nus, 
their king, laid siege to Clusium, a city of Etruria. The 
inhabitants of Clusium having applied for assistance to tlie 
Romans, the senate sent three patricians of the Fabian family 
on an embassy to Brennus, to inquire into the cause of 
offence given by the citizens of Clusium. To this he sternly 
replied, that " the right of valiant men lay in their swords ; 
that the Romans themselves had no other right to the cities 
they had conquered." The ambassadors, on entering the 
city, assisted the inhabitants against the assailants. This 
conduct so incensed Brennus that he immediately raised the 
siege of Clusium, and marched directly for Rome, and in a 

What were the consequences of this measure ? — 21. What was at length 
decreed ? Who was created dictator ? What did he cause ? How was he 
honored? Of what was he accused?— 22. What is said of the Gauls? 
What reply did Brennus make to the embassy sent by the Romans ? 
What did one of the ambassadors do ? How did Brennus resent this 
conduct ? 



ROME. 95 

great battle on the banks of the Allia he defeated the Roman 
army with great slaughter. 

23. After this victory the Gauls entered Rome, put to the 
sword all the inhabitants that fell in their way, pillaged the 
city, and then burnt it to ashes. They next laid siege to 
the capitol, which the Romans defended with the utmost 
bravery. At length, having discovered a way which led to 
the top of the Tarpeian Rock, a body of Gauls undertook 
the difficult task of gaining the summit under the cover of 
the night, and even succeeded in accomplishing their design, 
while the Roman sentinel was asleep. At this moment the 
gabbling of some sacred geese in the temple of Juno roused 
the garrison, and, through the exertions of Mariiis Man- 
lius, the Gauls were instantly thrown headlong down the 
precipice. 

24. As the Gauls now gave up all hope of being able to 
reduce the capitol, they agreed to quit the city on condition 
that the Romans would pay them one thousand pounds' 
weight of gold ; but, after the gold was brought forth, the 
Gauls endeavored, by fraudulent weights, to impose upon 
the Romans; and when the latter offered to complain, Bren- 
nus, casting his sword and belt into the scale, replied, that it 
was the only portion of the vanquished to suffer. At this 
moment Camillus, who in the meantime had been restored 
to favor and again appointed Dictator, entered the gates of 
the city at the head of a large army. Having been informed 
of the insolence of the enemy, he ordered the gold to be car- 
ried back to the capitol, saying that it had been the manner 
of the Romans to ransom their country by steel, and not by 
gold. A battle followed, in which the Gauls were entirely 
routed, and the Roman territories delivered from those for- 
midable invaders. 

25. After the defeat of the Gauls, through the exertions 
of Camillus, who was honored as the father of his country 
and the second founder of Rome, the city soon began again 
to rise from its ashes. Shortly after this, Manliiis, whose 
patriotism and valor had shone so conspicuous in defending 
the capitol and saving the last remains of Rome, abandoned 
himself to ambitious views ; and, being accused of aspiring 

23. On entering Eome, what did the Gauls do? Having discovered 
a way to the Tarpeian Rock, what did the Gauls do ? How was the 
garrison roused ? — 24. To what did the Gauls agree ? At this moment 
who appeared at the gates of the city? What did he order? What 
ensued? — 25. After the defeat of the Gauls, what took place? What 
is related of ]\Ianlius ? 



96 ROME. 

to the sovereign power, he was sentenced to he thrown head- 
long from the Tarpeian Rock. Thus the place which had 
been the theatre of his glory became that of his punishment 
and infamy. 

.2(3. The Romans next turned their arms against the Sam- 
nites, who inhabited an extensive tract of country in the 
south of Italy. During this contest, which lasted for about 
fifty years, the Romans were generally successful, with the 
exception of a defeat sustained near Gaudium, when their 
whole army was compelled to pass under the yoke, formed 
by two spears placed upright and a third placed across them. 
But, roused by this defeat rather than discouraged, the Ro- 
mans, the following year having created Papir'ius Cur'sor, 
Dictator, gained a signal victory over the Samnites, and com- 
pelled them in turn to undergo the same disgrace : and, pur- 
suing their good fortune under Fabius Maximus and Decius, 
they finally brought them under subjection. 

27. A war shortly afterwards followed between the Ro- 
mans and Latins ; but as their clothing, arms, and language 
were similar, the most exact discipline was necessary in 
order to prevent confusion in the engagement. Orders were 
therefore issued by Manilas, the Consul, that no soldier 
should leave his ranks under the penalty of death. When 
the armies were drawn out in order of battle. Melius, a 
Latin, challenged to single combat any one of the Roman 
knights. Upon this Titus Manlius, the son of the Consul, 
forgetful of the orders of his father, accepted the challenge, 
and slew his adversary. Then taking the spoils of the 
enemy, he hastened to lay them at the feet of the Consul, 
who, with tears in his eyes, told him that as he had violated 
military discipline, he had reduced him to the deplorable 
extremity of sacrificing his son or his country, but added, 
tha.t a thousand lives would be well lost in such a cause ; 
and accordingly ordered him to be beheaded. In the mean- 
time the battle followed, in which the Latins were van- 
quished and submitted to the Romans. 

28. The Tarentines, who were the allies of the Samnites, 
being unable to defend themselves, applied for aid to Pyr'- 
rhiis, king of Ep'irus, the most celebrated general of his 
age. 

26. Against whom did the Romans next turn their arms ? Where 
did they suffer a defeat ? Who was created dictator?— 27. What war 
next followed? W^hat orders were issued by Manlius? What is re- 
lated of Titus, his son?— 28. To whom did the Tarentines apply for 
aid? 



ROME. 97 

Having accepted the invitation, Pyrrhus immediately- 
sailed for Taren'tum with an army of thirty thousand men 
and twenty elephants. The Consul, Lavinus, hastened to 
oppose him. The Romans, unaccustomed to the mode of 
fighting with elephants, were defeated with the loss of fifteen 
thousand men ; but the loss on the side of the Grecian mon- 
arch was nearly the same, and he was heard to say that 
another such victory would compel him to abandon his en- 
terprise. Struck with admiration at the heroism of the 
enemy, he exclaimed, "Oh, with what ease could I conquer 
the world had I the Romans for soldiers, or had they me for 
their King I " 

29. The conduct of Fabric' his, the Roman general, during 
this war, claims universal admiration. On one occasion, 
having received a letter from the physician of Pyrrhus, 
importing that for a proper reward he would poison the 
king, the noble Roman, indignant at so base a proposal, 
gave immediate information of it to Pyrrhus, who, admiring 
the generosity of his enemy, exclaimed, '' It is easier to turn 
the sun from its course than Fabricius from the path of 
honor." Pyrrhus, after suffering a total defeat near Bene- 
ventum, withdrew to his own dominions, and the Romans, 
shortly after his departure, became masters of all the south- 
ern part of Italy. 



CHAPTER III. 



FROM THE FIRST PU'NIO WAR TO THE CONQUEST OF 
GREECE.— B, C. 264 TO 146. 

AS the history of Rome now becomes connected with 
that of Ca7^thage and Sicily, it may not be improper 
to introduce here a short account of those states. Carthage 
is said to have been founded by Di'do, with a colony of 
Tyrians, about nine hundred years before the Christian era. 
The government was at first monarchical, but afterwards be- 
came republican. It is highly commended by Aristotle as 
one of the most perfect of antiquity, but, according to the 
same author, it had two great defects : the first was investing 



Who was sent to oppose him ? What was the issue of the battle ? 
What did Pyrrhus exclaim ? — 29. What is related of Fabricius ? What 
did Pyrrhus say of him ? 

Chapter III.— 1. What is said of Carthage? Of the government? 
9 G 



98 ROME. 

the same person with different public employments ; and the 
second was that a certain income was required before a man 
could attain to any important office, by which means poverty 
might exclude a person of the most exalted merit from 
holding a civil employment. 

2. The supreme power was placed in the Senate ; there 
were two magistrates annually elected, called Sefifetes, whose 
authority in Carthage answered to that of the consuls at 
Rome. Commerce was the chief occupation of the Car- 
thagin'ians, to which they were indebted for their wealth 
and power. Their religion was a degrading superstition ; 
the cruel practice of offering human victims was exercised 
among them. At the time of the Punic Wars the city of 
Carthage had risen in wealth and commercial importance, 
surpassing any other city in the world. It had under its 
dominion a number of towns in Africa, bordering on the 
Mediterranean, besides a great part of Spain, Sicily, and 
other islands. 

3. From Egypt the Carthaginians brought flax, paper, 
corn, etc. ; from the coast of the Red Sea, spices, perfumes, 
gold, pearls, and precious stones ; from Tyre and Phoenicia, 
purple, scarlet, and the like : in a word, they brought from 
various countries all things that contribute not only to the 
convenience, but even to the luxury and pleasures of life. 
They are represented as being greatly wanting in honor and 
integrity. Cunning, duplicity, and want of faith seem to 
have been a distinguishing feature in their character ; hence 
the phrase — Fu'nica Fi'des — Punic Faith, was used to 
denote treachery. 

4. The Carthaginians seem never to have excelled as a 
literary people ; there were, however, among them several 
distinguished scholars. The great JSan'nihal, who in all 
respects was the ornament of the city, was not unacquainted 
with polite literature. Ma'go, another celebrated general, 
wrote twenty-eight volumes upon husbandry, which were 
afterwards much esteemed by the Romans. There is still 
extant a Greek version of an account written by Eanno, 
relating to a voyage made by him with a considerable fleet 
round Africa for the settling of different colonies. Glito'- 

What were its defects?— 2. In what was the power placed? What 
were the magistrates called ? What is said of religion ? Of Carthage, 
at the time of the Funic wars ?— 3. What did the Carthaginians bring 
from Egypt ? From Tyre ? How are they represented ?— 4. Did they 
ever excel as a literary people ? What is said of Hannibal ? Of Mago ? 
What is still extant ? 



ROME. 99 

machus, called in the Punic tongue As'drubal, was a great 
philosopher. Carthage produced several eminent generals, 
among whom HamiVcar, Asdrubal, and Hannibal were the 
most distinguished. 

5. Sicily is said to have been settled by a colony of ^Phoe- 
nicians, previous to the Trojan war ; but the Greeks at a 
later period made settlements on the island. It contained 
many large and populous cities ; of these Syracuse was the 
most populous and commercial. This city, at an early period, 
was under a democratical form of government, which in the 
course of time was overthrown, and a monarchy established 
in its stead. Gelon, one of its sovereigns, is represented as 
possessed of every virtue ; but the tyranny and cruelty of 
his successors caused a revolution in the state, and the regal 
government was abolished. After a period of sixty years 
it was again restored by Diony' sius, a man of great abili- 
ties ; but his son, Dionysius the younger, a weak and ca- 
pricious tyrant, was dethroned by the aid of Timo'leon, an 
illustrious Corinthian, and banished to Corinth, where he 
ended his life in poverty. 

6. The Romans, being anxious to extend their conquests, 
soon found an opportunity of indulging in their design. 
The Mamertines, a people of Campania, obtained assistance 
of the Romans in a war with Hie'ro, King of Syracuse; 
the Syracusans, in their turn, assisted the Carthaginians ; a 
war was thus brought on between the latter and the Romans, 
called the first Pu'nic War. The first object of both powers 
was to obtain possession of Messa'na, a city which com- 
manded the passage of the straits, but it finally became a 
contest for the dominion of the whole island. 

7. But there seemed an insurmountable obstacle to the 
ambition of Rome. She had no fleet, and Carthage was 
sovereign of the sea. The Romans, however, resolved to 
overcome every obstacle that lay in their way to conquest. 
A Carthaginian vessel, which happened in a storm to be 
driven on the coast, served as a model; and in the short 
space of two months, a fleet consisting of one hundred ves- 
sels was constructed and ready for use. The consul Duillius 
was appointed to the command of the armament, and though 

What did Carthage produce ?— 5. What is said of Sicily ? What did 
it contain ? What is said of Gelon ? What was the fate of Dionysius 
the younger ? — 6. What occasioned the first Punic War ? What was the 
object of both powers ? — 7. What was any obstacle to the ambition of 
Kome? How did the Eomans surmount the difficulty? Who was 
appointed to command the fleet ? 



100 ROME. 

much inferior to the enemy in the management of his fleet, 
yet he gained the first naval victory, defeated the Cartha- 
ginians, and took fifty of their vessels. 

8. At the commencement of the war, the Syracusans, who 
had allied with the Carthaginians, changed their course and 
joined the Romans. The Carthaginians, however, after a 
long siege, took the city of Agrigen'tum. A second naval 
engagement soon afterwards took place, in which the Ro- 
mans were again victorious ; the Carthaginians, under Han- 
no and Hamilcar, lost sixty of their vessels. The consul, 
Beg'ulus, in the meantime, was sent by the senate to carry 
the war into Africa ; and having landed on the coast, defeated 
the Carthaginians, and carried his victorious arms to the 
very walls of their capital. But here his good fortune seemed 
to forsake him ; he was signally defeated by the Carthagin- 
ians under the command of Xanthip'pus, a Spartan general, 
and fell into the hands of the enemy. 

9. The Carthaginians, weary of continuing the war, be- 
came desirous of treating for peace, and with this view they 
sent ambassadors to Rome, and among their number was 
Regulus, who had now been detained four years a prisoner, 
havmg previously exacted a promise, on oath, that he would 
return to Carthage if the negotiation should fail. But Reg- 
ulus, not deeming the terms of peace sufficiently advan- 
tageous to his country, strenuously opposed their being 
accepted, and returned to Carthage, where, after the most 
cruel tortures, he was finally put to death, by being placed in 
a barrel driven full of nails, pointing inwards, and in this 
painful situation he continued until he died. 

10. The war was now renewed on both sides with more 
than former animosity ; but at length the perseverance of the 
Romans was crowned with success. Peace was granted to 
the Carthaginians on the most humiliating conditions. It 
was agreed that they should abandon Sicily, pay the Romans 
thirty-two hundred talents, and release their captives. Thus 
terminated the first Punic War, after it had lasted twenty-four 
years. Sicily was now declared a Roman province, but Syra- 
cuse still maintained its independent government. After this 
war the Romans completed the conquest of Cisalpine Gaul ; 

What was the issue of the engagement? — 8. What is said of the 
Syracusans? What was the result of the second naval engagement? 
What is related of Kegulus? — 9. Whom did the Carthaginians send to 
Eome to negotiate a peace ? What did Eegulus do ? How was he put 
to death? — lO. On what conditions was peace granted to the Carthagin- 
ians ? After the conquest of Cisalpine Gaul what did the Romans do ? 



ROME. 101 

and now being at peace with all mankind, they closed the 
temple of Janus for the first time since the reign of Numa. 

11. The Carthaginians had made peace only because they 
were no longer able to continue the war ; they therefore took 
the earliest opportunity of breaking the treaty. They be- 
sieged Sagun'tum, a city in Spain, then in alliance with 
Rome ; and although requested to desist, they refused to 
comply. This refusal led to the second Punic War. To 
Han'nibal, the son of Hamilcar, the Carthaginians intrusted 
the command of their army. This extraordinary man, whilst 
very young, was brought before the altar and made to take 
an oath that he never would be in friendship with the Ro- 
mans, nor desist from opposing their power until he or they 
should be no more. Being now raised to the chief command 
of the forces of his country, though only in the twenty-sixth 
year of his age, he formed the bold design of carrying the war 
into Italy, as the Romans had before carried it into the do- 
minions of Carthage. 

12. For this purpose, leaving Hanno to guard his conquest 
in Spain, he crossed the Pyrenees, entered Gaul, and with an 
army of fifty thousand foot and nine thousand horse, in a 
short time appeared at the foot of the Alps. It was now in 
the midst of winter. The prodigious height of the mountains, 
their steepness, and summits covered with snow, presented a 
picture that might have discouraged any ordinary individual. 
Bat nothing could subdue the resolution of the Carthaginian 
general. At the end of fifteen days he effected the passage of 
the Alps, and found himself on the plains of Italy — but with 
only a half of his numerous army. 

13. Scarcely had he arrived in Italy when the Romans 
hastened to oppose his progress. But Hannibal gained four 
memorable victories, — the first, over Scvp'io near Ticin'us ; 
the second, over Sempro'nius, the consul, in which twenty- 
six thousand Romans were destroyed ; the third, near lake 
Thrasyme'nus over Flaminius ; and the fourth at Can'nas, 
over jEmil'ius and Varro. The last was the most memor- 
able defeat the Romans ever sustained. More than forty 
thousand of their troops were left dead upon the field, to- 
gether with the consul JEmilius. Among the slain were so 

11. What led to the second Punic War? To whom did the Cartha- 
ginians intrust the command of their array ? What is said of him whilst 
young? What bold design did he form? — 12. Leaving Hanno in Spain 
what did Hannibal do? How many days did he occupy in crossing the 
Alps?— 18. What four memorable victories did he now gain? What 
is said of the last ? 
9* 



102 ROME. 

many Roman knights, that Hannibal is said to have sent to 
Carthage three bushels of gold rings, which they wore on 
their fingers. Hannibal, however, either finding it impracti- 
cable to march directly to Rome, or wishing to give his forces 
rest after so signal a victory, led them to Capua, where he 
resolved to spend the winter. 

14. The chief command of the Roman forces was now 
given to Fabius Maximus, styled the Shield, and to Marcel- 
lus, the Sivord, of Rome. After the battle of Can'nse, the 
good fortune of the Carthaginian general seemed to forsake 
him. At the siege of Nola he was repulsed with considera- 
ble loss by Marcelliis, and his army was harassed and weak- 
ened by Fabius. Marcellus took the city of Syracuse after 
a siege of three years, during which time it was chiefly de- 
fended by the genius of the celebrated Ar chime' des. The 
inhabitants were put to the sword, and among them Ar- 
chimedes himself, who was found by a Roman soldier en- 
gaged in his study. 

15. A large army of Carthaginians, sent from Spain into 
Italy, under the command of Asdrubal, the brother of Han- 
nibal, was defeated, and their General slain by the Romans, 
under the command of the consuls, Livy and Nero. The 
very night on which Hannibal was assured of the arrival of 
his brother, Asdrubal's head was cut off and thrown into his 
camp. Scipio, the younger, surnamed Africanus, after his 
return from the conquest of Spain, was made consul at the 
early age of twenty -nine ; but instead of opposing Hannibal 
in Italy, formed a wiser plan, v/hich was to carry the war 
into Africa. On his arrival at the very walls of their capi- 
tal, the Carthaginians, alarmed for the fate of their empire, 
immediately recalled Hannibal from Italy. When the order 
came, the great commander hastened to return to his native 
country, after having kept possession of the most beautiful 
parts of Italy for about fifteen years. 

16. Having arrived in Africa, he marched to Adrime'tum, 
and finally, upon the plains of Za'ma, he was met by Scipio 
at the head of the Roman army ; and after a fruitless attempt 
to negotiate a peace, a tremendous battle was fought, in which 
the Carthaginians were totally defeated, with the loss of 

How many rings did he send to Carthage ? — 14. To whom was the 
command of the Eoman forces now given ? What were they styled ? 
By whom was the city of Syracuse defended ? \Vhat was his fate ? — 
15. What is said of the Carthaginian army ? What plan did Scipio, 
the younger, form? On his arrival what did the Carthaginians do? 
How long had he remained in Italy ? — 16. Where was he met by Scipio ? 



ROME. 103 

twenty thousand of their troops, which were left dead upon 
the field, and as many more taken prisoners. This victory 
was followed by a peace, on conditions that Carthage should 
abandon Spain, Sicily, and all the islands in the Mediterra- 
nean, surrender all her prisoners, give up her whole fleet, 
except ten galleys, and in future undertake no war without 
the consent of the Romans. To these hard conditions the 
Carthaginians were compelled to subscribe. Thus terminated 
the Second Punic War, after having lasted seventeen years, 
n. Hannibal, after this event, passed the last thirteen 
years of his life in exile from his native country, and finally 
took refuge in the court of Pru'sias, king of Bith'ynia. 
The Romans, who were bent on his destruction, sent ^mil- 
ius, one of their most celebrated generals, to demand him 
from this king, who, fearing the resentment of Rome, deter- 
mined to deliver up his guest. The great but unfortunate 
General, in order to avoid falling into the hands of his ene- 
mies, destroyed himself by poison. 

18. While the Romans were engaged in hostilities with 
the Carthaginians, they also carried on a vigorous war 
against Philip, king of Macedo'nia, which finally terminated 
in favor of Rome. After this the Romans turned their 
arms against Anti'ochus the Great, king of Syria, who was 
defeated by Scipio, surnamed Asiaticus, in the great battle 
of Magne'sia. A second war followed with Macedonia, 
which terminated in the defeat of Perse'us, the last king of 
that country, at the battle of Pydna ; after which Macedonia 
was reduced to a Roman province. 

19. About this time, Masinissa, the Numidian, made in- 
cursions into a territory claimed by the Carthaginians, who 
attempted to repel the invasion. The Romans, pretending 
this as a violation of their treaty, laid hold of it as a pretext 
for commencing the third Punic War, with a determination 
not to desist until the city of Carthage should be destroyed. 
Porcius Cato, one of the most prominent members of the 
senate, strongly insisted on this measure, and usually con- 
cluded his speeches in these words : Delenda est Carthago- — 
Carthage must be destroyed. 

The Carthaginians, conscious of the superior power of the 

What was the issue of the battle? What were the conditions of 
the peace? — 17. Where did Hannibal finally take refuge? How did 
he die? — 18. What other war did the Romans carry on at this time? 
Against whom did they next turn their arms? What happened after 
the battle of Pydna?— 19. What led to the third Punic war? How 
did Cato usually conclude his speeches ? 



104 ROME. 

Komans, endeavored by every species of submission to avert 
the impending' ruin of their country. They yielded to the 
Romans their ships, their arms, and munitions of war ; but 
they were still required to abandon their capital, that it 
might be levelled to the ground. 

20. This demand was received with mingled feelings of 
sorrow and despair ; and, finding no alternative, the wretched 
Carthaginians began to prepare to suffer the utmost extremi- 
ties in order to save the seat of their empire. The vessels 
of gold and silver which adorned their luxurious banquets 
were now converted into arms ; even the women parted with 
their ornaments, and cut off their hair to be made into bow- 
strings. After a desperate resistance for three years, the 
city was taken by Scipio, also called Africanus, and destroyed. 
Thus was Carthage, one of the most renowned cities of an- 
tiquity, with its walls and temples, razed to its foundation. 
Such of the inhabitants as refused to surrender themselves 
prisoners of war either fell by the sword or perished in the 
ruins of their capital. The scenes of horror presented on the 
occasion, it is said, forced tears even from the eyes of the 
Roman general. 

21. The destruction of Carthage was succeeded by the 
conquests of several other states. Corinth was taken and 
destroyed by the Consul Mummius, and Greece reduced to a 
Roman province. Scipio having laid siege to Numantia, a 
city in Spain, the inhabitants, to avoid falling into the hands 
of the enemy, set fire to the town and perished in the flames. 
After this event, Spain fell under the dominion of Rome. 



CHAPTER IV. 



THE SEDITION OF TEE GRAC'CHII; CIVIL WARS; CON- 
SPIRACY OF CATILINE.— B. C. 133 TO 63. 

THE Romans, who had been long distinguished for tem- 
perance and military enterprise, were not as yet a liter- 
ary people. Among them the arts and sciences had been but 

What did the Carthaginians do? — 20. How was this demand re- 
ceived? What did they make of their vessels of gold and silver? 
How long did the siege last ? What is said of the scene ? — 21 . By 
what Avas the reduction of Carthage succeeded ? What is related of the 
inhabitants of Numantia ? 

Chapter I. — 1. What is said of the Romans? 



EOME. 105 

little cultivated. After the conquest of Greece, a favorable 
change took place ; and with the Mxury of that nation was in- 
troduced at Rome a taste for literature. But as they grew 
in power, luxury and a corruption of manners began to pre- 
vail. By the destruction of Carthage Rome was left without 
a rival. Her arms were everywhere successful. 

2. When she had triumphed, however, over all her ene- 
mies abroad, domestic dissensions began to prevail at home. 
Tihe'rius and Ca'ius Grac' chus, men of eloquence and in- 
fluence, distinguished themselves by declaiming against the 
corruptions which began to prevail among the great, and by 
asserting the claims of the people. Tiberius, the elder of the 
two brothers, while tribune, with a view of checking the 
power of the patricians, and abridging their immense es- 
tates, endeavored to revive the Licin'ian law, which or- 
dained that no citizen should possess more than five hun- 
dred acres of public land. In consequence of this proposal 
a tumult followed, in which Tiberius, together with three 
hundred of his friends, was slain in the streets of Rome by 
the partisans of the senate. 

3. When this tragical event took place, Caius Gracchus, 
in the twenty-first year of his age, was yet in retirement, en- 
gaged in the quiet pursuit of study. The fatal example of his 
brother did not deter him from following a similar career. 
Having been elected to the tribuneship, he procured an 
edict granting the freedom of the city to the inhabitants of 
Latium, and afterwards to all the people on that side of the 
Alps ; he also procured that the price of corn should be fixed 
at a moderate rate, and a monthly distribution of it among 
the people. He then proceeded to an investigation of the 
late corruptions of the senate, and that whole body were 
convicted of bribery, extortion, and sale of offices. These 
measures did not fail to enkindle the resentment of the sen- 
ate. Gracchus was marked out for destruction, and he finally 
fell a victim to their vengeance, with three thousand of his 
partisans, who were slaughtered in the streets of Rome by 
the Consul OpinnHus. 

4. Jugur'tha, the grandson of the famous Masinissa, at- 
tempted to usurp the throne of Numid'ia, by destroying his 

After the conquest of Greece, what took place? — 2. What now began 
to prevail ? What did Tiberius Gracchus endeavor to revive ? In con- 
sequence of this, what followed? — 3. Having been elected to the trib- 
uneship, what did Caius Gracchus procure? What did he then pro- 
ceed to do? What was the consequence of these measures? — 4. Wliat is 
said of Jugurtha? 



106 ROME. 

cousins, Hiem'psal and Adher'bal, the sons of the late king 
Micip'sa. The elder fell a' victim to his treachery, but Ad- 
herbal, the youno;er, having escaped, applied for assistance 
to the senate of Rome ; but that body being bribed by Ju- 
gurtha, divided the kingdom between the two. Jugurtha 
having invaded the territories of Adherbal, defeated and slew 
him in battle, then seized upon his whole kingdom ; but by 
this act he drew upon himself the resentment of Rome. War 
having been declared against him, the command of the army 
was at first confided to Metellus, but when on the point of 
gaining a complete triumph over the king of Numidia, he 
was supplanted in the command by the intrigues of Cains 
Ma'rius, who had the honor of terminating the war. Ju- 
gurtha w^as defeated and taken prisoner, and led to Rome in 
chains, and, having adorned the triumph of the conqueror, 
was cast into prison and starved to death. 

5. About this period the Roman republic w^as again con- 
vulsed by domestic dissensions. The Italian states being 
frustrated in their aims of gaining the freedom of Rome, by 
the intrigues of the senate, resolved to gain by force what 
they could not obtain as a favor. This gave rise to the So- 
cial War, which continued to rage for several years, and is 
said to have involved the destruction of three hundred thou- 
sand men. It was finally terminated by granting the rights 
of citizenship to all who should lay down their arms and 
return to their allegiance. 

6. This destructive w^ar being concluded, the Romans next 
turned their arms against Mithrida'tes, king of Pon'tus, the 
most powerful monarch of the East, who caused eighty thou- 
sand Romans, who dwelt in the cities of Asia Minor, to be 
massacred in one day. In this celebrated contest, styled the 
Mithridatic war, the Roman generals, Syl'la, LucnVlus, and 
Pom'pey, successively bore a distinguished part. The chief 
command in the war against Mithridates w^as first given to 
Sylla, a man of great talents and an able general ; but Ma'rius, 
who had been distinguished for his w^arlike genius and ex- 
ploits for nearly half a century, now in the seventieth year 
of his age, had the address to get the command of the army 
transferred from Si/lla to himself. 

7. Sylla, on receiving this intelligence, and finding his 

Who fell a victim to his treachery ? How did he incur the resent- 
ment of Kome? What was his fate?— 5. What is said of the Italian 
states? What did this give rise to? How was it terminated? — 6. 
Against whom did the Eoraans next turn their arms ? What generals 
took part in the Mithridatic war? What is said of Marius?— 7. On re- 
ceiving this intelligence^ what did Sylla do ? 



ROME. 107 

troops devoted to his interest, marched directly to Rome, 
which he entered as a place taken by storm, and, proceeding 
to the senate, compelled that body to issue a decree declaring- 
Marius to be a public enemy. Marius, in the meantime, 
fled to Africa, and Sylla, after some delay, entered upon the 
Mithridatic war. Cin'na, a partisan of Marius, having col- 
lected an army in his favor, recalled the veteran warrior, and 
they soon presented themselves at the gates of Rome. Ma- 
rius refused to enter the city, alleging that having been ban- 
ished by a public decree, it was necessary that another should 
authorize his return. But before the form of annulling the 
sentence of his banishment was concluded, he entered the 
city at the head of his guards, and ordered a general mas- 
sacre of all who had ever been obnoxious to him. Many of 
those who had never offended him were put to death ; and 
at last, even his own officers could not approach him without 
terror. He next proceeded to abrogate all laws made by his 
rival, and associated himself in the consulship with Cinna. 
Thus having gratified his two favorite passions, vengeance 
and ambition, his bloody career was arrested by death, and 
shortly afterwards Cinna was cut off by assassination. 

8. In the meantime, these accounts were brought to Sylla, 
who was pursuing a victorious campaign against Mithridates ; 
but having concluded a peace with that monarch, he hastened 
to Rome to take vengeance on his enemies. Having entered 
the city, he caused a more horrible massacre than that which 
took place under Marius. He ordered eight thousand men, 
who surrendered themselves to him, to be put to death, while 
he, without being the least discomposed, harangued the 
Senate. The day following he proscribed forty Senators 
and sixteen hundred Knights, and after a short interval 
forty Senators more, with a much greater number of the 
most distinguished citizens of Rome. He then caused him- 
self to be proclaimed perpetual Dictator, but after having 
held it for nearly three years, to the astonishment of all man- 
kind he resigned the dictatorship and retired to the country, 
where he passed the remainder of his days in the society of 
licentious persons and the occasional pursuit of literature. 
After his death a magnificent monument was erected to him, 
with the following epitaph, written by himself:, — "I am Sylla, 
the Fortunate, who, in the course of my life, have surpassed 

What did Cinna do in favor of Marius ? What did Marius refuse ? 
Having entered the city, what did he order ? What did he next do ? — 
8. What did Sylla do on entering the city ? What did he cause to be 
proclaimed? What was the epitaph written by himself? 



108 ROME. 

both friends and enemies ; the former in the good, and the 
latter in the evil I have done them." In the civil war be- 
tween Mariiis and Sylla one hundred and fifty thousand Ro- 
man citizens are said to have been sacrificed, including among 
them more than two hundred Senators and persons of dis- 
tinguished rank. 

9. While the commonwealth was yet distracted by the 
old dissensions new calamities w^ere added. Spar'tacus, a 
Thra'cian, who had been kept at Capua as a gladiator, 
placing himself at the head of an army of slaves, laid waste 
the country, but was at length totally defeated by Gras'sus, 
with the loss of forty thousand men. A few years after this 
event a conspiracy which threatened the destruction of Rome 
was headed by Gat'iline, a man of courage and talents, but of 
ruined fortune and of the most profligate character. A plan 
was concerted for a simultaneous insurrection throughout 
Italy ; that Rome should be fired in different places at once, 
and that in the general confusion Catiline, at the head of an 
army, should enter the city and massacre all the Senators. 
The plot was fortunately detected and suppressed by the 
vigilance and energy of Cic'ero, the great Roman orator, 
who was Consul at the time. Catiline, at the head of an 
army of twelve thousand men, was defeated and slain in battle. 



CHAPTER V. 



FROM THE FIRST TRIUM'VTRATE TO THE DISSOLUTION OF 
THE COMMONWEALTH.— B. C. 60 TO 31. 

POM'PEY, who, on account of his military exploits, was 
surnamed the Great, having been appointed to conduct 
the Mithridatic war, brought it to a successful termination. 
He defeated Mithrida'tes and Tigra'nes, king of Arme'nia, 
reduced Syr'ia, together with Jude'a, to a Roman province. 
On his return to Rome he was honored with a splendid tri- 
umph, which continued three days, during which the citizens 
gazed with astonishment on the spoils of eastern grandeur 
which preceded his chariot. 

How many citizens perished in the civil war ? — 9. What is related of 
Spartacus ? What took place after this event? What plan was formed? 
By whom was it detected ? 

Chapter V. — 1. What is said of Pompey? How was he honored 
on his return to Kome ? 



ROME. 109 

2. Pompej, however, found a great rival in Cras'ms, 
who was the richest man in Rome, and courted popularity 
by his extensive patronage and unbounded liberality. As they 
both aspired to the first place in the republic, a mutual jeal- 
ousy existed between them. Such was the state of things 
when Ju'lius Cae'sar, a young man, who had already dis- 
tinguished himself by his military achievements, had the 
address to effect a reconciliation between them and to ingra- 
tiate himself into the favor of both. Pompey, Crassus, and 
Caesar agreed to appropriate to themselves the whole power 
of the state, and entered into that famous league styled the 
First Trmm'virate. 

3. They immediately proceeded to divide the Roman prov- 
inces among themselves. Pompey, who had remained at 
Rome, received Spain and Africa ; Syria fell to the lot of 
Crassus, and Caesar chose Gaul for his portion, and as soon 
as time permitted proceeded to take possession of his prov- 
ince. Crassus, in a war with the Par'thians, was defeated 
and slain, leaving the empire to his two colleagues. The 
brilliant career of victory which attended the arms of Caesar 
in Gaul, his high military reputation and increasing popu- 
larity, did not fail to awaken a spirit of jealousy in the breast 
of Pompey. Caesar, desirous of trying whether his rival 
would promote or oppose his pretensions, applied to the 
Senate for a continuation of his authority, which was about 
to expire. That body, being devoted to the interests of 
Pompey, denied his request and finally ordered him to lay 
down his government and disband his forces within a limited 
time, undeV the penalty of being considered an enemy to the 
commonwealth. 

4. This hasty measure determined the course of Caesar. 
He now resolved to support his claim by force of arms, and, 
finding his troops devoted to his interest, he immediately 
commenced his march towards Italy. Having crossed the 
Alps, he halted at Ravenna and wrote again to the Senate, 
offering to resign all command if Pompey would follow his 
example ; but that body refused to listen to his demand. 
Proceeding on his march, he soon arrived on the banks of 
the Ru'hicon, a small river separating Italy from CisaKpine 
Gaul, and forming the limits of his command. The Romans 

2. In whom did Pompey find a rival ? Who efiected a reconciliation 
jbetween them ? What did they agree to do ? — 3. Where did Pompey 
remain ? What fell to the lot*of Crassus ? What did Csesar choose ? 
What happened to Crassus ? What was the effect of Caesar's career 
pf victory ? What is said of Csesar? — 4. What did he now resolve? 
10 



110 ROME. 

had always been taught to consider this river as the sacred 
boundary of their domestic empire. CsBsar, therefore, when 
arrived at the banks of this famous stream, stopped short, as 
if impressed with the greatness of his enterprise and its 
fearful consequences ; he pondered for some time in fixed 
melancholy, looking upon the river, and then observed to 
PoVlio, one of his generals, " If I pass this river, what mis- 
eries shall I bring upon my country ; and if I now stop 
short, I am undone." Thus saying, he exclaimed, " The die 
is cast !" and putting spurs to his horse, he plunged into the 
stream, followed by his troops. 

5. The news of Csesar^s movement excited the utmost 
consternation at Rome. Pompey, who had boasted that he 
could raise an army by stamping his foot upon the ground, 
finding himself unable to resist Caesar in Rome, where the 
latter had many partisans, led his forces to Capua, where he 
had a few legions ; thence he proceeded to Brundusium, and 
finally passed over to Dyrachium, in Macedonia. In his 
retreat he was followed by the Consuls and the greater part 
of the Senators. Among them were the famous Cato and 
Cicero, the illustrious orator. 

6. Csssar, in the meantime, having made himself master 
of all Italy in the space of sixty days, marched to Rome, en- 
tered the city in triumph, amidst the acclamations of the 
citizens, seized the public treasury, and possessed himself of 
the supreme authority. On every occasion he manifested the 
greatest liberality and clemency ; he said that he had entered 
Italy, not to injure, but to restore the liberties of Rome. 
After a stay of only a few days he proceeded to Spain, 
where he defeated Pompey^s lieutenant, made himself master 
of the whole country, and again returned victorious to 
Rome. The citizens received him with fresh demonstrations 
of joy and created him Consul and Dictator, but the latter 
ofiice he resigned after he had held it eleven days. 

1. While Cassar was thus employed, Pompey was equally 
assiduous in making preparation to oppose him. All the 
monarchs of the east had declared in his favor and sent him 
large supplies ; his army was numerous and his fleet con- 
sisted of five hundred vessels. Caesar, remaining only eleven 

When he arrived on the banks of the Rubicon what is related of 
Caesar? What did he say?— 5. What is said of Pompey? Where did 
he proceed ? By whom was he followed ? — 6. In the meantime, what 
did Csesar do? What did he manifest? Where did he proceed? 
What was he created? — 7. While Caesar was thus employed, what is 
said of Pompey ? 



R () ME. Ill 

days in Rome, led bis forces in pursuit of Pompey. But before 
coming to any general engagement he once more made an 
effort to bring his rival to an accommodation, offering to 
refer all to the Senate and people of Rome. This overture 
was rejected, on the ground that the people of Rome were 
too much in Cassar's interest. 

8. The two armies came in sight of each other near 
Dyra'chium, where an engagement took place which termi- 
nated in favor of Pompey, who afterwards led his forces to 
the plains of Pharsa'lia, where he determined to await the 
arrival of Caesar, and decide the fate of the empire by a sin- 
gle battle. This was what Caesar had long and ardently 
desired ; and now, learning the resolution of Pompey, he 
hastened to meet him. Everything connected with the con- 
test about to follow was intended to excite the deepest in- 
terest. The armies were composed of the bravest soldiers in 
the world, commanded by the two greatest generals of the 
age, and the prize contended for was nothing less than the 
Roman empire. Pompey's army consisted of upwards of 
fifty thousand men, while the forces of Caesar were less than 
half that number, yet under much better discipline. 

9. As the armies approached, the two Generals went from 
rank to rank, encouraging their men, animating their hopes, 
or lessening their apprehensions. Pompey urged the justice 
of his cause, declaring that he was about to battle in the de- 
fence of liberty and his country. Caesar, on the other hand, 
insisted on nothing so strongly to his soldiers as his frequent 
and unsuccessful endeavors for peace ; he spoke of the blood 
he was about to shed with the deepest regret, and only pleaded 
the necessity which urged him to it. There was only so 
much space between the two armies as to give room for 
fighting. The signal for the onset was given. Csesar's 
men rushed to the combat with their usual impetuosity. 
The dreadful conflict raged with unabating fury from early 
in the morning till noon, when the scales of victory turned 
in favor of Caesar, whose loss only amounted to two hun- 
dred men. Fifteen thousand of Pompeifs troops were left 
dead upon the plain, and twenty-four thousand surrendered 
themselves prisoners of war. 

Before coming to any engagement, what did Csesar do ? — 8. Where did 
a slight engagement take place ? Where did Pompey lead his forces ? 
What is said of Caesar? What of the armies? — 9. As the armies 
approached, what was done ? What did Pompey urge ? On what did 
Csesar insist ? What was the issue of the battle ? What was the num- 
ber of the slain ? 



112 ROME. 

10. Caesar, on this occasion, manifested his usual charac- 
teristic disposition of clemency and humanity. He set at 
liberty the Senators and Roman Knights, and incorporated 
with his own army the greater number of the prisoners ; 
and committed to the flames all Pompey's letters without 
reading them. When viewing the field strewed with his 
fallen countrymen, he seemed deeply affected at the melan- 
choly spectacle, and was heard to say: "They would have 
it so." 

11. The situation of Pompey was deplorable in the ex- 
treme. For thirty years he had been accustomed to victory, 
and ruled the councils of the commonwealth ; a single day 
beheld him precipitated from the summit of power, a miser- 
able fugitive. Escaping from the field of battle, and wan- 
dering along the beautiful vale of Tempe, he finally found 
means of sailing to Lesbos, where he met his wife. Gome'- 
lia. Their meeting was deeply affecting. At the news of 
his reverse of fortune she fainted ; but at length recovering, 
she ran through the city to the sea-side. Pompey received 
her without speaking a word, and for some time supported 
her in his arms in silent anguish. The time, however, 
would not permit him long to indulge in grief. Accom- 
panied by Cornelia, he sailed for Egypt with a few friends 
to seek protection of Ptolemy, whose father he had be- 
friended. But as he approached the shore he was basely 
murdered, while yet within sight of his wife, and his body 
thrown upon the sand. His free^man burnt the corpse and 
buried the ashes, over which was placed the following in- 
scription : '' He can scarcely find a tomb, whose merits de- 
serve a temple." 

12. In the meantime Ciesar lost no time in pursuing his 
rival to Egypt, but on his arrival there the first news he re- 
ceived was the account of Pompey's unfortunate end ; and 
shortly afterwards he was presented with the head and ring 
of the fallen General, but turning his face from the sight, he 
gave vent to his feelings in a flood of tears. He soon after- 
wards ordered a splendid monument to bfe erected to Pom- 
pey's memory. The throne of Egypt, at this time, was 
disputed by Ptolemy and his sister, the celebrated Cleopa'tra ; 

10. What is said of Csesar on this occasion? On viewing the field, 
what was he heard to say? — 11. What was the situation of Pompey? 
How did he receive his wife? Where did he sail? What was his 
fate? What inscription was placed on his tomb? — 12. In the mean- 
time what did Csesar do ? What is said of the throne of Egypt at this 
time? 



ROME. 113 

but Caesar, captivated by the charms of the beautiful queen, 
decided the contest in her favor, and at length reduced Egypt 
to the dominion of Rome. Caesar, after this event, aban- 
doned himself to pleasure in the company of Cleopatra, but 
was soon called to suppress the revolt of Phar'naces, the son 
of Mithridates, who had seized upon Colchis and Armenia. 
Caesar defeated him in a battle at Zela with so much ease 
that, in writing to the senate at Rome, he expressed the 
rapidity of his victory and suppression of the revolt in these 
words: Veni, vidi, vici — " I came, I saw, I conquered.''^ 

13. Leaving the scene of conquest in the East, Ccesar has- 
tened to Rome, where his presence was much required by 
reason of the disorders occasioned by the bad administration 
of Antony, who governed the city during his absence ; but 
tranquillity was soon restored. Caesar's stay at Rome was 
short, being called into Africa to oppose an army raised by 
the partisans of Pompey, under the command of Scipio and 
Cato, assisted by Juba, King of Maurita'nia ; he, however, 
defeated their united forces in the battle of Thapsus. Upon 
this, Cato, who was a rigid Stoic and stern republican, fled 
to Utica, where he resolved to resist the power of Caesar, 
but finding that all was lost, determined not to survive the 
liberty of his country, and killed himself in despair. 

14. At the conclusion of the war in Africa, Caesar re- 
turned to Rome, and celebrated a magnificent triumph, which 
lasted four days. The first was for Gaul, the second for 
Egypt, the third for his victories in the East, and the fourth 
for his victory over Jiiba. He distributed liberally rewards 
to his veteran officers and soldiers. The citizens also shared 
his bounty ; and after distributing a certain quantity of corn, 
oil, and money among them, he entertained them at a public 
feast at which twenty thousand tables were set, and treated 
them to a combat of gladiators. The senate and the people, 
intoxicated by the allurements of pleasure, seemed to vie with 
each other in their acts of servility and adulation towards the 
man who had deprived them of their liberty. He was hailed as 
the father of his country, created perpetual dictator, received 
the title of Emperor, and his person was declared sacred. 

15. Having restored order in Rome he again found himself 
obliged to go into Spain, where Labie'nus and the two sons 

What is said of Caesar ? After the battle of Zela how did Caesar ex- 
press the rapidity of his victory? — 13. What was Caesar's next course? 
What called him into Africa ? What is related of Cato ? — 14. At the con- 
clusion of the war what did Caesar do ? How did he entertain the people ? 
How was he hailed ? \& \ c. — 15. Why was he again obliged to go into Spain ? 
10* H 



114 ROME. 

of Pompey had raised an army against him ; but he com- 
pletely defeated them in an obstinate battle, fought on the 
plains of Munda. Gdesar, by this victory, having triumphed 
over all his enemies, devoted the remainder of his life to the 
benefit of the commonwealth. As clemency was his favorite 
virtue, he readily pardoned all who had at any time bore 
arms against him. Without any distinction of party, he 
seemed only to consider the happiness and prosperity of the 
people. He adorned the city with magnificent buildings ; 
rebuilt Carthage and Corinth, sending colonies to both these 
places. He corrected many abuses in the state, reformed the 
calendar, undertook to drain the Pontine marsh, and intended 
to cut through the isthmus of Peloponnesus. 

16. But while he thus meditated projects beyond the limits 
of the longest life, a deep conspiracy was formed against him, 
embracing no less than sixty Senators, among whom were 
Bru'tus and Gas'sius, whose lives had been spared by the 
conqueror after the battle of Pharsalia. It had been ru- 
mored that a crown would be presented to him on the ides 
of March — the 15th of that month. The conspirators there- 
fore fixed upon that day for the execution of their design. 

Accordingly, as soon as Caesar had taken his seat in the 
senate-house, they assembled around him under the pretence 
of soliciting the pardon of a certain individual who had been 
banished by Caesar's order, and assailed him with their dag- 
gers. The illustrious Roman defended himself for some 
time with great vigor, until seeing Brutus, his friend, whom 
he tenderly loved, among the conspirators, he exclaimed, Ut 
tu Brute ! ''And you too, Brutus P^ then resigning himself 
to his fate and covering his face with his robe, he fell, pierced 
with twenty-three wounds, at the base of Pompey's statue. 

Thus perished Julius Caesar, in the fifty-sixth year of his 
age — a man whose ruling passion was ambition, and whose 
redeeming virtue was clemency.* 

17. No sooner was the death of Cassar known than the 
w^hole city was thrown into the utmost consternation. His 
bleeding corpse was exposed in. the forum ; his friend, Mark 
Antony, pronounced over it a funeral oration, and by his elo- 

* See his biography at the close of the volume ; also, Shakspeare's play 
of Julius Ccesar. 

Having triumphed over all his enemies what did he resolve to do? 
Mention some of the acts he now performed. — 16. What was formed 
against him ? What had been rumored ? Wiiat happened as Csesar took 
his seat in the senate-house ? How did he defend himself ? On seeing 
Brutus what did he say ? What was his age ? — 17. What was done by 
Mark Antony? 



ROME, 115 

quent appeals to the sympathy of the people so inflamed 
their resentment against his murderers that they were 
obliged to escape from the city. 

Mark Antony, who was a man of great military talents, 
but of a most profligate character ; Lep'idus, who was pos- 
sessed of immense wealth ; and Octa'vius Cse'sar, afterwards 
surnamed Augustus, who was Caesar's grand-nephew and 
adopted heir, now formed the design of dividing among 
themselves the supreme authority, and thus establishing 
the second Triumvirate, which produced the most dreadful 
calamities in the Roman Republic. 

18. They stipulated that all their enemies should be de- 
stroyed, each sacrificing his nearest friends to the vengeance 
of his colleagues. Thus Antony consigned to death his 
uncle Lucius; Lepidus, his brother Paulus ; and Octavius 
gave up his friend, the celebrated Cicero, to whom he was 
under the most binding obligation, in order to gratify the 
hatred of Antony. The illustrious orator was assassinated 
in the sixty-fourth year of his age, by PopiFlius Lanus, 
whose life he had saved in a capital case. Rome was again 
deluged in the blood of her citizens ; in the horrible proscrip- 
tion that followed three hundred Senators with two thousand 
Knights, besides many other persons of distinguished rank, 
were sacrificed. 

19. In the meantime Brutus and Cassius, having retired 
into Thrace, collected an army of one hundred thousand men, 
and made the last and expiring effort to restore the common- 
wealth. Antony and Octavius marched against them with 
an army superior in number. Again the empire of the world 
depended upon the issue of a single battle. The two armies 
met on the plains of Philip'pi, and after a dreadful conflict, 
which lasted for two days, the death-blow was given to Ro- 
man liberty, by the total defeat of the republican army. 
Brutus and Cassius resolving not to survive the liberties 
of their country, avoided the vengeance of their enemies by 
a voluntary death. 

20. The power of the Trium'viri being thus established 
upon the ruins of the commonwealth, they began to think 
of enjoying the honors to which they had aspired. Lepidus 
was shortly after deposed and banished. Antony went into 

Who composed the second Triumvirate? — 18. What did they stipu- 
late ? What was the fate of the illustrious orator ? What is said of 
Eome? — 19. What was done by Brutus and Cassius? By whom were 
they opposed ? Where did the armies meet ? What was the issue of 
the battle ? What was the fate of Brutus and Cassius ? — 20. What was 
the fate of Lepidus ? Where did Antony go ? 



116 ROME. 

Greece, and having made a brief stay at Athens, he passed 
into Asia. He proceeded from kingdom to kingdom, at- 
tended by a crowd of sovereigns, exacting contributions and 
giving away crowns with capricious insolence. He sum- 
moned Cleopa'tra, Queen of Egypt, to Tarsus, to answer 
to the charge of having aided the conspirators. She accord- 
ingly came, decked in all the emblems of a royal coquette. 
Her galley was covered with gold ; the sails of purple float- 
ing to the wind ; the oars of silver swept to the sound of 
flutes and cymbals. Cleopatra reclined upon a couch span- 
gled with stars of gold, and such ornaments as the poets 
usually ascribe to Yenus. Antony, captivated by her charms, 
forgot to decide upon her cause, and giving up all the pursuits 
of ambition, abandoned himself to pleasure in the company 
of the beautiful Egyptian Queen. He lavished on her the 
provinces of the Roman Empire ; and having on her account 
divorced his wife Octavia, the sister of his colleague, an open 
rupture took place between him and Octamus. 

21. The great battle of Ac'tium decided the contest in 
favor of Octavius, who, by this victory, was left sole master 
of the empire. After this defeat, Antony put an end to 
his life by falling on his sword; and Cleopatra, to avoid 
being led captive to Rome to grace the triumph of Augustus, 
procured her own death by the poison of an asp.* 



CHAPTER VI. 

ROME AS AN EMPIRE. 

THE C^SARS: AUGUSTUS, TIBERIUS, CALIGULA, CLAUDIUS, 
NERO, GALEA, OTHO, VITELLIUS, VESPASIAN, TITUS, AND 
DOMITIAN.—B. C. 31 TO A. B. 96. 

BY the death of Antony, Octavius, now styled Augustus, 
became sole master of the Roman Empire. Having re- 
turned in triumph to Rome, he endeavored, by sumptuous 

* For a fuller account of ancient Roman history, see LiddelFs His- 
tory of Rome. 

What is related of Cleopatra? What did he lavish on her? What 
took place between him and Octavius? — 21. What is said of the battle 
of Actium ? What was the end of Antony and Cleopatra? 

Chapter VI. — 1. Who now became sole master of the empire ? 



ROME. 117 

feasts and magnificent shows, to obliterate the impressions 
of his former cruelty, and resolved to secure, by acts of 
clemency and benevolence, that throne, the foundation of 
which was laid in blood. Having established order in the 
state, Augustus found himself agitated by difierent inclina- 
tions, and considered for some time whether he should retain 
the imperial authority or restore the republic. By Agrip'ioa 
he was advised to pursue the latter course ; but following the 
advice of Maece^nas, he resolved to retain the sovereign au- 
thority. 

2. Augustus, in his administration, affected an appearance 
of great moderation and respect for the public rights, and 
having gained the affections of the people and his soldiers, 
he endeavored by every means to render permanent their 
attachment. As a military commander he was more fortu- 
nate than eminent; though the general character of his reign 
was pacific, still several wars were successfully carried on by 
his lieutenants. He seemed to aim at gaining a character by 
the arts of peace alone. He embellished the city, erected 
public buildings, and pursued the policy of maintaining or« 
der and tranquillity in every portion of his vast dominions. 
During his reign the temple of Janus was closed for the first 
time since the commencement of the second Funic War, and 
the third time from the reign of Numa. 

Augustus having accompanied Tiberius in his march into 
Illyria, was taken dangerously ill, and, on his return, died at 
No'la, near Capua, in the seventy-sixth year of his age, after 
an illustrious reign of forty-four years. 

3. Augustus was possessed of eminent abilities, both as a 
warrior and a statesman ; but the cruelties and treachery ex- 
ercised by him while a member of the triumvirate, have left 
an indelii3le stain upon his character, and render it doubtful 
whether the virtues which he manifested in after-life sprung 
rather from policy than from principle. The Emperor and 
his chief minister, Maecenas, were both eminent patrons of 
learning and the arts ; and the Augustan age of Roman 
literature has been justly admired by all succeeding times. 
Among those who distinguished his reign were the celebrated 
poets Virgil, Horace, and Ovid, with Livy, the historian. 

What did he endeavor to do? By what was he agitated? Whose 
advice did he follow? — 2. What did Augustus effect? What is said 
of him as a general ? During his reign, what was closed ? Where did 
he die? What was his age and length of his reign? — 3. What is 
said of the abilities of Augustus ? Of what was he patron ? Who were 
distinguished in his reign ? 



118 ROME. 

But the most glorious event which took place during the reign 
of Augustus, was the birth of our Lord and Savior Jesus 
Christ, which happened, according to the best authorities, 
in the twenty-sixth year of his reign, and four years before the 
period commonly assigned for the Christian era. 

4. Augustus, previous to his death, had nominated Tibe'- 
rius to succeed him on the throne. The new Emperor, at 
the commencement of his reign, exhibited a show of modera- 
tion and clemency ; but he soon threw off the mask and 
appeared in his natural character, as a cruel and odious tyrant. 
The brilliant success of his nephew German'icus, in Ger- 
many, excited the jealousy of Tiberius, who recalled him to 
Rome, and is supposed to have caused his death by poison. 
Having then taken into his confidence Seja'nus, a Roman 
knight, who became the minister of his cruelty and pleasure, 
he retired to the island of Capreae, and abandoned himself to 
the most infamous debaucheries. Sejanus, now possessed of 
almost unlimited power, committed the most fearful cruelties 
against the citizens of Rome. Nero and Drusus, the sons 
of Germanicus, were starved to death in prison. Sabinus, 
Gallus, and other distinguished persons, were executed upon 
slight pretences. But the career of the brutal Sejanus was 
of short duration. Being accused of treason, he was sud- 
denly precipitated from his elevation and executed by order 
of the Senate ; and his body was afterwards dragged igno- 
miniously through the streets. 

5. This event seemed only to increase the Emperor's rage 
and cruelty. He became weary of particular executions, and 
gave orders that all the accused should be put to death with- 
out further examination. When one Carnu'lius had killed 
himself to avoid the torture, Tiberius exclaimed: "Ah," 
" how has that man been able to escape me I" He died in 
the seventy-eighth year of his age and twenty-second of his 
reign ; his death was hastened either by strangling or poison. 
In the eighteenth year of this Emperor's reign, our Lord Jesws 
ChyHst suffered death upon the cross. 

6. Tiberius adopted for successor Galig'ula, who com- 
menced his reign under the most favorable auspices, and his 

What was the most memorable event that took place during it ? — 4. 
Whom did Augustus nominate? How did he commence his reign? 
What excited his jealousy? Whom did he take into his confidence? 
What is said of Sejanus? What was his fate? — 5. What orders did 
the emperor give now ? What exclamation did he make? Wlien did 
he die? What took place in the eighteenth year of his reign? — 6. 
By whom was he succeeded ? 



ROME. 119 

first acts were even beneficent and patriotic ; but his subse- 
quent conduct was marked by every species of human de- 
pravity. He assumed divine honors, and caused temples to 
be built, and sacrifices to be ofi'ered to himself as a divinity. 
He took such delight in cruelty, that he wished that all the 
Roman people had but one neck, that he might despatch 
them at a single blow. Happily for mankind the reign of this 
monster was of short duration ; he was assassinated in the 
twenty-ninth year of his age and fourth of his reign, a. d. 41. 

t. After the death of Galig'ula, his uncle Clau'dius, the 
grandson of Mark Antony, was raised to the throne. He 
was a man of weak and timid character, and a slave to the 
most degrading vices. The only remarkable enterprise dur- 
ing his reign was his expedition into Britain. Carac'tacus, 
the patriotic king of that island, after a brave resistance, was 
taken prisoner and carried captive to Rome. As he passed 
through the streets and observed the splendor of the city, he 
exclaimed, " How is it possible that men possessed of such 
magnificence at home, should envy Caractacus in an humble 
cottage in Britain ? " 

Claudius was poisoned by his wife Agrippi'na, in the 
fourteenth year of his reign and sixty-fourth of his age, in 
order to make room for Ne'ro, her son by a former husband, 
A. D. 55. 

8. Nero, now in the seventeenth year of his age, began his 
reign with general approbation ; he was even so much inclined 
to clemency and forgiveness, that, when obliged to sign a war- 
rant for the execution of a criminal, he would exclaim, 
"Would to heaven that I had never learned to write." He 
had received an excellent education under the philosopher 
Seneca, and while he followed the counsels of his illustrious 
preceptor, he governed with much applause. But as he ad- 
vanced in age, every trace of virtue vanished with his in- 
creasing years. Abandoning the advice of his virtuous 
counsellors, he soon gave himself up to every species of 
depravity, and rendered his name proverbial in all succeeding 
ages as a detestable tyrant. The first alarming instance of 
his cruelty, was the revolting execution of his own mother 
Agrippina. Among others who fell victims to his brutality, 

What is said of him ? What did he assume ? How did he die ? — 
7. Who was next raised to the throne? What was his character? 
Who was led captive to Rome ? What did he exclaim ? What was the 
end of Claudius? — 8. Who succeeded him? What is said of Nero? 
By whom was he educated? What was the first alarming instance of 
his cruelty ? Who were some of the other victims ? 



120 ROME. 

were Senec'a, the philosopher, Bur'rush, the prefect of the 
pretorian guard, and Lii'can, the poet. 

9- In his wild extravagance he even caused the city of 
Rome to be set on fire, that it might exhibit the representa- 
tion of the burning of Troy, and stood upon a high tbwer 
that he might enjoy the scene. The conflagration continued 
for nine days, and a great part of the splendid city was 
burnt to ashes. But in order to avert from himself the pub- 
lic odium of this action, he openly charged it upon the 
Christians, who had now become numerous at Rome, and 
published against them a violent persecution, during which 
the two illustrious apostles, St, Peter and St. Paul, suffered 
martyrdom. The former was crucified with his head down- 
wards ; the latter being a Roman citizen, had the honor of 
dying by the sword. Nero having rendered himself con- 
temptible by his follies and crimes, was soon destined to 
finish his career by a tragical end. The army in Spain having 
declared against him, raised GaVha to the throne ; and the 
unhappy tyrant, finding himself deserted by all and con- 
demned by the Senate, avoided falling into the hands of his 
enemies by a voluntary death, in the fourteenth year of his 
reign and the thirty-second of his age. 

10. On the death of Nero, Oalha was acknowledged Em- 
peror by the Senate, as he bad been previously declared by 
the legions under his command. He was a man of much 
prudence and virtue, and had acquired a high military repu- 
tation, but he was now in the seventy-second year of his age, 
and soon became unpopular with the army by his severity 
and parsimony. At length, finding himself unable to sus- 
tain the duties of the government alone, he adopted for his 
successor the virtuous Pi'so. This measure, however, gave 
rise to a revolt in the army headed by O'tho, which termi- 
nated in the death both of the Emperor and Pino, after a 
reign of seven months. Tac'itus says of him, that " had he 
never ascended the throne, he would have been deemed by 
all capable of reigning." 

11. Otho was now declared Emperor by the army ; but in 
ViteVlius he found a formidable rival, who now aspired to 



9. What did he cause ? How long did the conflagration last ? How 
did he avert the odium from himself? During the persecution^ who 
Buflered martyrdom ? What did the army in Spain do ? What was the 
end of Nero ? — 10. Who was now acknowledged by the Senate ? What 
is said of Galba ? What did he adopt ? What was his end ? What 
did Tacitus say of him?— U. Who was now declared Emperor? 

r 



ROME, 121 

the imperial throne. Otho, being defeated, slew himself, after 
a reign of ninety-five days. Upon this event, Vitellius was 
proclaimed Emperor, but having rendered himself odious to 
the people by his profligac}^ and tyranny, he was assassinated 
before he had completed the first year of his reign. At the 
same time, Vespa' sian, who was now at the head of the 
army in Egypt, was proclaimed Emperor by his troops. On 
the arrival of the newly elected Emperor at Rome, he was 
received with universal joy. He had risen from an humble 
origin to the highest station in the state ; he was equally 
distinguished for his affability, clemency, and firmness. He 
ornamented the city by erecting various edifices, built the 
amphitheatre or coliseum, cherished the arts, and was a pa- 
tron of learned men, among whom were Jose'phus, the Jew- 
ish historian, QuintiVian, the orator, and Plin'y, the nat- 
uralist. 

12. The most memorable event of the reign of Vespasian 
was the destruction of Jerusalem by his son Ti'tus. After 
a tremendous siege of six months, the city was taken and 
razed to the ground, verifying the predictions of our Divine 
Savior, that** not a stone should remain upon a stone." 
According to Josephus, the number of Jews that perished 
during the siege exceeded one million, and the captives 
amounted to almost one hundred thousand. Vespasian 
having reigned ten years, beloved by his subjects, died at 
Campania, in the seventieth year of his age, a. d. T9. 

13. The late Emperor was succeeded by his son Titus, 
who, on account of his amiable virtues, justice, and human- 
ity, obtained the appellation of the ''Delight of mankind." 
Recollecting one evening that he had done no act of benefi- 
cence during that day, he exclaimed, "My friends, I have 
lost a day." His reign is memorable for the great eruption 
of Mount Vesuvius, which overwhelmed the cities of Her- 
cula'neum and Pompe'ii, and caused the death of Pliny, the 
naturalist, whose curiosity led him too near the scene. Titus 
died in the third year of his reign and in the forty-first of his 
age ; but strong suspicion was entertained that he was poi- 
soned by his brother Domi'tian, who succeeded to the throne 
A. D. 81. 

What was his fate ? Who succeeded ? What was the end of Vitellius ? 
Who was next ? From what had he risen ? Of what was he the patron ? 
— 12. What was the most memorable event of his reign ? What number 
of Jews perished during the siege? When did he die? — 13. By whom 
was he succeeded? What is said of Titus? For what is his reign 
memorable ? When did he die ? 
11 



122 ROME. 

14. Domitian was another Nero in his character. He 
caused himself to be worshipped as a god. Many of the most 
illustrious men of Rome fell victims to his cruelty. He ban- 
ished the philosophers from the city and raised a dreadful 
persecution against the Christians. He frequently shut him- 
self up in his chamber and amused himself by catching flies 
and piercing them with a bodkin, hence his servant, being 
asked if any one was with the Emperor, replied, " No, not 
even a fly." His reign was signalized by the success of the 
Roman arms in Britain, under the command of Agric'ola, a 
distinguished general who had been sent to the country hj 
Yespasian, and conquered all the southern portion of the 
island. Domitian was assassinated at the instigation of his 
wife, in the fifteenth year of his reign, a. d. 96. He was the 
last of those Emperors called the Twelve Gsesars, Julius 
Csesar, the Dictator, being considered the first — although 
Augustus was the first who was generally styled Emperor. 



CHAPTER yil. 

FROM NERVA TO CONSTANTINE THE GREAT.— A. D. 96 TO 305. 

AFTER the death of Domitian, Ner'va was elected to the 
throne. He was a man distinguished for virtue and 
clemency, but did not possess sufficient energy to suppress 
the disorders of the empire ; and having adopted Tra'jan 
for his successor, he died after a short reign of sixteen months. 
2. Trajan, a native of Seville, in Spain, is esteemed one 
of the greatest and most powerful of the Roman Emperors. 
He was equally distinguished for aff'ability, clemency, and 
munificence. On presenting the sword to the Prefect of the 
pretorian guard, he made use of these remarkable words: 
'' Make use of it for me, if I do my duty ; if not, use it 
against me." The Senate conferred on him the title of Op- 
timiis, the Best, and that body was long accustomed to salute 
every newly elected Emperor with this expression : " Reign 
fortunately as Augustus, and virtuously as Trajan.''^ 

14. What is said of Domitian, his successor ? What instance is given 
of his cruelty ? By what was his reign signalized ? How did he die ? 
Of whom was he tlie last ? 

Chapter VII. — 1. Who was now elected to the throne? What is 
said of Nerva? — 2. What is said of Trajan ? What words did he make 
use of on presenting the sword to the Prefect of the guard ? 



ROME. 123 

3. Trajan was one of the greatest generals of his age. He 
enlarged the boundaries of the empire, subdued the Par- 
thians, brought under subjection Assyria, Arabia Felix, and 
Mesopotamia; and in commemoration of his victory over 
the Dacians he erected a pillar at Rome, which bears his 
name, and which still remains as one of the most remarkable 
monuments of that city. 

He w^as a munificent patron of literature, and in his 
reign Pliny, the younger, Ju' venal, and Plu'tarch flourished. 
Although this prince was much celebrated for his virtues, 
still his character has been tarnished by a want of equity 
with regard to the Christians, who were persecuted during 
his reign. He died of apoplexy, in the sixty-third year of 
his age and the twentieth of his reign, A. D. lit. 

4. Trajan was succeeded by A'drian, his nephew, who, in 
some respects, was the most remarkable of the Roman Em- 
perors. His administration was generally just and benefi- 
cent. He was highly skilled in all the accomplishments of 
the age ; he composed with great beauty, both in prose and 
verse ; he pleaded at the bar, and was one of the best orators 
of his time. Deeming the limits of the empire too extensive, 
he abandoned the career of conquest and devoted himself to 
the arts of peace. He spent thirteen years in visiting the 
provinces of the empire, and during his progress he reformed 
abuses, relieved his subjects from many burdens, and rebuilt 
various cities. While in Britain, he caused a turf wall to be 
erected across the island from Carlisle to Newcastle, in order 
to prevent the incursions of the Picts. 

5. He rebuilt the city of Jerusalem, and changed its name 
to ^lia Capitolina. In consequence of an insurrection of 
the Jews, he sent against them a powerful army, which de- 
stroyed about one thousand of their towns and nearly six 
hundred thousand of these unfortunate people ; he then ban- 
ished all those who remained, and by a public decree forbade 
them to return within view of their native soil. He passed 
several wise regulations, among which was a law prohibit- 
ing masters to kill their slaves, as had been before allowed, 
but ordained that they should be tried by the laws enacted 
against capital ofi*ences. Adrian, having adopted for his suc- 

3. What was Trajan ? What did he erect ? Of what was he the patron ? 
What has tarnished his character ? When did he die ? — 4. By whom 
was he succeeded ? In what was he skilful ? What did he abandon ? 
In what did he spend thirteen years of his reign ? What did he do in 
Britain ? — 5. What city did he rebuild ? What severity did he exer- 
cise against the Jews ? Whom did he adopt for his successor ? 



124 ROME. 

cessor Titus Antoni'nus, died after a prosperous reign of 
twenty -two years, and in the sixty-third year of his age, a.d. 138. 

6. Antoninus, surnamed the Pious, was eminently distin- 
guished for his public and private virtues, although his reign 
was marked by few striking events. He showed himself one 
of the most excellent princes for justice, clemency, and mod- 
eration. During his reign, St. Justin, the Martyr, wrote his 
'' Apology for the Christians," and directed it to the Empe- 
ror, the Senate, and the people of Rome. Still many Chris- 
tians continued to suffer for their faith. Having adopted 
Mar'cus Au'relius Antoninus for his successor, he expired 
at Lorium, near Rome, in the twenty-third year of his reign 
and in the seventieth of his age, a. d. 161. 

7. Marcus Aurelius was esteemed as a model of pagan 
virtue, and was greatly attached, both by nature and educa- 
tion, to the Stoic philosophy, which he exemplified in his life, 
as well as illustrated in his book, entitled "Meditations." 
While engaged in a war with the Germans, his army expe- 
rienced a remarkable deliverance through the prayers of a 
Christian legion then serving under his command. The 
Emperor, in a letter to the Senate, after stating the dis- 
tressed situation of his army, says: "I put up my fervent 
prayers to the gods for our relief; but the gods were deaf. 
I knew there were many Christians in the army. I called 
them around me and commanded them to address their God 
in our behalf. No sooner had they fallen upon their knees 
to pray, than a copious and refreshing rain fell from the 
heavens. But -while the rain was refreshing to us, it drove 
furiously against our enemies, like a tempest of hail, at- 
tended with vivid flashes of lightning and dreadful claps of 
thunder. Wherefore, since the prayers of these people are 
so powerful with their God, let us grant to the Christians 
full liberty of professing themselves such, lest they employ 
their prayers against us. My will is that their religion be 
no longer considered a crime in them." 

8. The Christian soldiers who had saved the Roman army 
by their prayers were afterwards distinguished by the name 
of the Thundering Legion. But notwithstanding the hu- 
mane disposition of Aurelius, many Christians suffered dur- 

When did he die ? — 6. What did Antoninus show himself? Who wrote 
an apoloary for the Christians ? When and where did he die ? — 7. What 
is said of Marcus Aurelius ? In a war with the Germans, what did he 
experience ? Can you relate, in substance, his letter to the Senate ? — 
8. What is said of the Christian soldiers ? Of the Christians during 
his reign ? 



ROME. 125 

ing his reign, owing chiefly to the violence of Yerus, his 
colleague in the empire. Among the most illustrious who 
received the crown of martyrdom were St. Jus'tin, and St. 
PoVycarp, the illustrious Bishop of Smyrna. Aurelius died 
in the nineteenth year of his reign and the fifty-ninth of his 
age, A. D. 180. He was the last of those styled the five good 
Emperors. 

9. Aurelius was succeeded by his degenerate son. Com- 
mo'dus, whose whole reign was a tissue of folly, cruelty, 
and injustice ; but his crimes finally brought him to a tragi- 
cal end. He was assassinated in the thirteenth year of his 
reign and thirty-second of his age. Per'tinax, a man of 
humble birth, who had risen by his merit, and was styled 
the ''tennis-ball of fortune," on account of the various con- 
ditions through which he had passed, was proclaimed Empe- 
ror by the pretorian guards. But having given ofTence by 
his severity, in correcting abuses, he was put to death by the 
hands of the very soldiers who had raised him to the throne 
only three months before. 

10. The Empire was now put up for sale by the soldiers, 
and purchased by Did'ius Julian'us, for the sum of $9,000,- 
000. But the new Emperor only enjoyed the honors of 
royalty for the space of five months, being assassinated 
by the order of Sep'timns Seve'rus, who was proclaimed 
Emperor in his stead. Severus having triumphed over his 
two competitors, Ni'ger and Alhi'nus, governed with great 
ability. He made an expedition into Britain, and built a 
stone wall extending from Solway Frith to the German 
Ocean, and nearly parallel with that of Adrain. He died at 
York, in the eighteenth year of his reign, and in the sixty- 
sixth of his age, a. d. 211. 

11. Severus left the Empire to his two sons, Carac'alla and 
Geta, but Caracalla resolving to govern alone, murdered his 
brother in the arms of his mother. His tyranny and cruelty 
at length excited against him the resentment of Maori' nus, the 
commander of his forces, who caused him to be assassinated, 
in the sixth year of his reign. Macri'nus was immediately 
declared Emperor in his place, but, after a reign of fourteen 

Who were the most illustrious of the sufferers? At Avhat age, and 
when did he die ?— 9. What is said of Commodus ? What was his end ? 
By whom was he succeeded ? What was the fate of Pertinax ? — lO.What 
was now done with the empire? By whom was it purchased? What 
was his end ? Who succeeded ? When and where did Severus die ? — 
11. To whom did Severus leave the empire ? What is related of Carac- 
aUa ? What was his fate ? Who was declared emperor ? 
11* 



126 ROME. 

months, was in his turn supplanted by Heliogab'alus, by 
whose command he was put to death. Heliogabalus was 
only in the fourteenth year of his age when he succeeded to 
the throne, yet he showed himself to be a monster of vice, 
cruelty, and extravagance. He was murdered by the soldiers, 
and his body thrown into the Tiber, after a brief reign of 
four years, having in that short period married and divorced 
six wives. 

12. Alexan'der Sev'erus, his cousin, who was chosen to 
succeed him, was a mild and amiable prince, whose excel- 
lent character shines with redoubled lustre when contrasted 
with those who preceded and followed him. His acquire- 
ments were equal to his virtues. He excelled in music, paint- 
ing, sculpture, and poetry. During an expedition against 
the Germans, who had made an irruption into the Empire, 
he was murdered by a mutiny of his soldiers, in the four- 
teenth year of his reign, and twenty-ninth of his age, a. d. 
235. 

13. On the death of Alexander, Max'imin, who had headed 
the mutiny against him, was elevated to the throne. Maxi- 
min was the son of a herdsman of Thrace, and was no less 
remarkable for the symmetry of his person and extraordi- 
nary strength than for his gigantic stature, being eight and 
a half feet in height. He was also distinguished for his 
military talents. Previous to his elevation he was remark- 
able for his simplicity, discipline, and virtue ; but after his 
accession to the throne he became a monster of cruelty, 
and seemed to sport with the terrors of mankind. He was 
finally assassinated by his soldiers, in the third year of his 
reign. 

14. The interval from the reign of Max'imin, and that of 
Diocle'tian, was filled by sixteen reigns, which furnish little 
that is pleasing, interesting, or instructive. Of all the Empe- 
rors who successively occupied the throne during that period 
of forty-six years, Claudius and Tacitus alone died a natural 
death. The Emperor Yale'rian, in a war with Sa^por, King 
of Persia, was defeated and taken prisoner. The Persian 
monarch treated his captive with the greatest indignity and 

By whose command was he put to death, and who succeeded ? What 
is said of Heliogabalus? What was his end? — 12. Y/ho was chosen to 
succeed him ? What is said of Alexander Severus ? How and when 
did he die? — 13. Who succeeded to the throne? For what was Maxi- 
min remarkable? How did he die? — 14. How many reigns between 
that of Maximin and Diocletian? What is related of the emperor 
Valerian ? 



ROME. 127 

cruelty. He used him as a footstool for mounting his horse, 
and finally ordered him to be put to death, then caused him 
to be flayed, and his skin to be painted red, and suspended 
in one of the Persian temples, as a monument of disgrace 
to the Romans. 

15. The reign of Aure'lian was distinguished for brilliant 
military achievements. He defeated the Goths and repelled 
the incursions of the Germans ; but his most renowned 
victory was that over Zeno'hia, the famous queen of Pal- 
my'ra, who fell into his hands ; her secretary, Longi'kus, 
the celebrated critic, was put to death by the order of the 
conqueror. On his return to Rome, Aurelian was honored 
with a most splendid triumph. Zenobia was reserved to 
grace the scene, bound in chains of gold, and decked with a 
profusion of pearls and diamonds. 

16. Diocletian, who was a son of a Dalma'tian slave, rose 
by his merit from the rank of a common soldier to that of 
an eminent commander, and was finally elevated to the 
throne, on the death of Numerian, a. d. 284. 

Two years after his accession he associated with himself, 
in government, his friend Maximin ; and in the year 292 
they took two other colleagues, Gale'rius and Constan'tius, 
each bearing the title of Caesar. The Empire was now di- 
vided into four parts, under the government of two Emperors 
and two Caesars, each nominally supreme, but in reality con- 
trolled by the superior talents of Diocletian. 

1*1. At this time happened the tenth and last persecution 
of the Christians, which continued for several years with so 
much violence that the brutal tyrants boasted that they had 
extinguished the Christian name. 

Diocletian and Maximin, in the midst of their triumphs, 
surprised the world by resigning their dignities on the same 
day and both retiring into private station, a. d. 304. It is 
generally believed that they were compelled to take this step 
by Galerius, who, together with Constan'tius, was imme- 
diately afterwards acknowledged Emperor. Diocletian seems 
to have been contented with his lot. He retired to Salona, 
in his native country, Dalmatia, where he lived eight years, 
and amused himself in cultivating a small garden. Maximiij 

15. For what was the reign of Aurelian distinguished? What was 
his most renowned victory ? What is said of Zenobia ? — 1 6. What is 
said of Diocletian ? Whom did he associate with himself in the gov- 
ernment? How was the empire now divided? — 17. What happened 
at this time ? How did Diocletian and Maximin surprise the world ? 
Where did Diocletian retire ? What is said of Maximin ? 



128 ROME. 

attempted several times, but in vain, to resume the sovereign 
power which he had abdicated, and even to murder his son- 
in-law, Con'stantine ; but being detected, he slew himself in 
despair. 



CHAPTER VIII. 



FROM THE ACCESSION OF CON'STANTINE TO THE FALL OF 
THE WESTERN EMPIRE.— A. D. 306 TO 476. 

CONSTAN'TIUS died at York, in Britain, having pre- 
viously appointed his son Con'stantine, surnamed the 
Great, his successor. Con'stantine had several competitors 
for the crown. Of these, Maxentius was the most formida- 
ble, who had made himself master of Italy and Rome. As 
the Emperor was on his march, at the head of his army, 
against his rival, he saw in the heaven, after mid-day, a lu- 
minous cross, bearing this inscription in Greek : ^ By this 
conquer. ''^'^ This circumstance is related by several historians 
of that period, particularly by Eusebius, in his life of Con- 
stantine. In consequence of this vision the Emperor avowed 
himself the friend and supporter of Christianity, and caused 
a splendid banner, called the Laba'rum, to be carried before 
his army, bearing a representation of the cross he had seen 
in the heavens. He now prosecuted the war against Max- 
en'tius with redoubled energy. A final battle was fought on 
the banks of the Tiber, in which Constantine was victorious. 
Maxentius himself perished in the river, a. d. 312. 

2. On the following day Constantine made a solemn entry 
into Rome, where he was received with universal joy, and 
hailed as the deliverer of the Empire. As a lasting monu- 
ment of this event, a magnificent triumphal arch was built 
at the foot of Mount Palatine, which remains at the present 
time. He published an edict in favor of Christianity, which 
he now openly embraced, and claims the honor of being en- 
rolled as the first Christian Emperor. He put an end to the 
persecution of the Christians, also to the combats of gladi- 



Chapter VIIT.— 1. What had Constantine? What is related of him 
as he marched at the head of his army ? In consequence of this vision, 
what did he do ? Where was a final "battle fought ? — 2. On the follow- 
ing day what did Constantine do ? What did he publish ? 



ROME. 129 

ators, and other barbarous exhibitions. His reign forms an 
important era in ecclesiastical history, as the Roman govern- 
ment now lent all its influence to support that religion which 
it had vainly but repeatedly attempted to destroy. The 
reign of Constantine is memorable for the celebrated Council 
of Nice, convened for the purpose of condemning the heresy 
of A7^ius, who denied the divinity of Christ. 

3. But the most important event of his reign was the re- 
moval of the capital of the Empire from Rome to Byzan'- 
tium, in Thrace, where he built a magnificent city, called 
from his own name, Constantino'ple. . As the Empire had 
long been verging to ruin, this measure is thought by many 
to have hastened its downfall. Constantine died at Nico- 
med'ia, after an illustrious reign of thirty-one years, and the 
sixty-third of his age, a. d. 337. The character of Constan- 
tine has been variously represented by difTerent writers. His 
greatest fault was his severity towards his son Crispus, a 
young prince of the most amiable character ; the Emperor 
being so far deceived by artful calumny as to believe him 
guilty of the most atrocious design, and in the first moment 
of indignation caused him to be pat to death. He has also 
been charged with a want of political sagacity in removing 
the seat of government. Still, whatever may have been his 
faults, we must admire and esteem his eminent qualities. 
The splendor of his military, political, and religious achieve- 
ments has deservedly gained for him the surname of Great, 
which posterity has conferred upon him 

4. Constantine left the Empire divided among his three 
sons, Constantine II., Constans, and Constantius. In the 
space of a few years the two former were slain, leaving Con- 
stantius, the youngest, sole master of the Empire. But his 
character was marked by weakness, jealousy, and cruelty. 
During his reign the Empire was harassed and weakened 
by the inroads of the barbarians from the north and the 
incursions of the Persians on the eastern provinces. Con- 
stantius died, after an unfortunate reign of twenty-four years, 

A. D. 361. 

5. Constantius was succeeded by his cousin Ju'lian, sur- 
named the Apostate, on account of his having renounced 
Christianity, in which he had been educated, and relapsing 

What does his reign form? For what is it memorable ?— 3. What 
was the most important event of his reign? Where and when did he 
die ? What is said of his character ? — 4. How did Constantine leave 
the empire? What is said of Constantius? Where did he die ? — 5. By 
whom was Constantius succeeded ? 

I 



130 ROME. 

into Paganism. He was a man possessed of considerable 
learning, of great military talents, but the slave of malice 
and the most bigoted superstition. He restored the pagan 
worship, and impiously attempted to suppress the Christian 
religion. With a design of falsifying the prediction of our 
Blessed Savior, he even undertook to reassemble the Jews 
and to rebuild their Temple ; but his design is stated, by a 
number of ancient writers, to have been miraculously de- 
feated by the eruption of fire-balls from the ground, which 
dislodged the stones, melted the iron instruments, and dis- 
persed the workmen. This royal apostate was slain in a Avar 
with the Persians, in the second year of his reign, and the 
thirty-second of his age, a. d. 363. 

6. Julian was succeeded by Jovian, who restored the 
Christian religion and recalled St. Athana'sius, who had been 
banished by the order of Julian ; but he died suddenly, after 
a mild and equitable reign of seven months. 

Valenti'nian, who was chosen to succeed him, associated 
with himself in the Empire his brother Yalens, who governed 
the eastern provinces ; and from this period the division of 
the Empire into Eastern and Western became fixed and per- 
manent. The barbarians continued to make inroads into 
different parts of the Empire, and the Goths finally obtained 
a settlement in Thrace. The domestic administration of 
Yalentinian was equitable and wise. His temper, however, 
was violent. On a certain occasion, when transported with 
rage, he ruptured a blood-vessel, and expired in a few hours, 
in the fifty-fourth year of his age, and in the twelfth of his 
reign. 

1. In the east, Val'ens held the sceptre with a weak and 
inefficient hand. Favoring the Arian heresy, he threw the 
provinces into confusion and contention, and at the same 
time exposed his dominions to the inroads of the barbarians. 
He was defeated and slain in an expedition against the Goths, 
in the fifteenth year of his reign. 

Gra'tian, the son and successor of Yalentinian, associated 
with himself Theodo'sius, afterwards surnamed the Great. 
The reign of this illustrious monarch was signalized by the 
complete triumph of Christianity and the downfall of pagan- 
ism throughout the Roman dominions. By his great mili- 

What is said of Julian ? What did he restore ? What did he under- 
take ? What is said of his design ? How did he die ? — 6. What is said 
of Jovian ? Who was chosen to succeed hirn ? What is said of the 
barbarians? How did Yalentinian die? — 7. What is said of Yalens? 
How did he die? Whom did Gratian associate with himself? 



ROME. ' 131 

tary abilities he successfully repelled the encroachments of 
the barbarians, and by his wise administration he strength- 
ened in some measure the Empire, which had been already 
hastening to its ruin. After a splendid reign of eighteen 
years, Theodosius left his dominions to his two sons, Eono'- 
rius in the West, and Area' dins in the East, a. d. 395. 

8. Theodosius was the last monarch who presided over 
both divisions of the Empire. By all the authors of that 
period, with the exception of Zos'imus, a Pagan writer, he 
is represented as a model of every public and private virtue, 
and worthy of the imitation of all Christian princes. His 
inclinations were naturally violent ; but if he committed any 
fault contrary to his usual clemency and meekness, he soon 
repaired it in a manner worthy of his character. On one 
occasion the populace of Thessalo'nica, in a tumultuous in- 
surrection, stoned their governor to death. Theodosius, on 
receiving intelligence of this outrage, in a moment of irrita- 
tion, gave orders for the soldiery to be let loose on the in- 
habitants of the city for three hours ; the commission was 
executed with so much fury, that seven thousand persons 
were put to the sword. But no sooner was the great St. 
Am'hrose, Archbishop of Milan, informed of this awful deed, 
than he declared to the Emperor that he could not admit him 
into the church until he had atoned, by a public penance, for 
the enormity of the massacre he had occasioned. Theodo- 
sius humbly submitted to the decision of the prelate, and re- 
mained excluded from the church for eight months. 

9. During the weak reign of Honorius and Arcadius, the 
barbarians made a successful irruption into the Empire, and 
possessed themselves of several of the most fertile provinces. 
The Goths, under the famous AVaric, spread their devas- 
tations to the very walls of Constantinople, and filled all 
Greece with the terror of their arms. Alaric then pene- 
trated into Italy at the head of a powerful army, but was 
defeated, with great loss, by the Eomans under the command 
of Sti'lieo. After the death of this General, Alaric invaded 
Italy a second time, and having taken and pillaged several 
cities, he at length pitched his camp before the walls of 
Rome. This famous capital, which had for ages been the 



To whom did Theodosius leave his dominions ? — 8. What is said of 
Theodosius ? Of his inclinations ? On one occasion what is related of 
him? What did St. Ambrose do? — 9. What happened during the 
reign of Honorius and Arcadius ? What is said of the Goths ? What 
did Alaric do a second time ? 



132 ROME. 

mistress of the world, and had enriched lierself by the spoils 
of vanquished nations, was now reduced to the greatest ex- 
tremities by famine and pestilence. 

10. Rome was finally taken by Alaric, who gave up the 
city to be plundered by his soldiers, with a charge to spill 
the blood of none but those whom they found in arms, and to 
spare all those who took refuge in the churches. The fearful 
devastation continued for six days, during which the fierce 
barbarians, notwithstanding the injunctions of the chieftain, 
indulged their cruelty and ferocity without pity or restraint. 
Alaric now prepared to invade Sicily and Africa, but death 
suddenly put an end to all his ambitious projects. The Goths, 
however, having elected Ataid^phus for their leader, took pos- 
session of the southern part of Gaul, and laid the foundation 
of their kingdom in Spain. 

11. A few years after the sacking of Home by Alaric, the 
country was again devastated by the Huns, a barbarous people 
of Scythian origin, under the command of At'tila, their king, 
styled the scourge of God. Having overrun the Eastern 
Empire, he invaded Gaul with an army of five hundred 
thousand men ; but he was defeated on the plains of Cha- 
lons, by the combined forces of the Romans, under ^'tius, 
and the Goths, under Theod'oric, with a loss of one hundred 
and sixty thousand men. The warlike spirit of Attila v/as 
checked by this defeat, but not subdued ; placing himself 
again at the^head of his army, he shortly afterwards invaded 
Italy, and having extended his ravages to the gates of Rome, 
compelled Valentin'ian III. to purchase a peace by the pay- 
ment of immense sums of money, with his sister Honoria 
in marriage. Attila died shortly after this event ; and his 
body is said to have been buried, enclosed in three coffins, 
the first of gold, the second of silver, and the third of iron. 
The men who dug the grave were put to death, lest they 
should reveal the place of his burial. 

12. Every circumstance now seemed to hasten the down- 
fall of the empire, which had been long on the verge of ruin. 
^'tius, the only man capable of defending it against its 
numerous enemies, was slain by the hand of Yalentinian him- 
self, on a pretended charge of conspiracy. 

10. What is said of Rome ? How long did the devastation continue ? 
What is said of Alaric?— ll. By whom was the country next devas- 
tated ? With how large an army did he invade Gaul ? Where and 
by whom was he defeated ? How did Valentinian purchase a peace ? 
What is said of the body of Attila ?— 12. What is said of iEtius ? How 
did Valentinian die? 



ROME. 133 

Shortly after this event, Valentinian was assassinated in 
his turn, at the instigation of Petrohiius Max'imus, who 
was proclaimed Emperor in his stead, and the empress 
Eudox'ia invited Gen^seric, king of the YanMals, to avenge 
the murder of her husband. He eagerly embraced the op- 
portunity, landed in Italy with a numerous army of Moors 
and Vandals, took the city of Rome, which he gave up to his 
soldiers to be pillaged for eleven days ; and after having de- 
stroyed many of the monuments of art and literature which 
Alaric had spared, and enriched himself with the spoils of 
the city, he returned to Carthage. 

13. From the reign of Valentinian III. the Western em- 
pire dragged out a precarious existence under nine successive 
Emperors, for the space of twenty-one years, until its final ter- 
mination, in 4Y6, by the resignation of Romulus Augustus, the 
last of its Emperors, to Odoa'cer, the chief of the Heru'li, 
who assumed the title of king of Italy. Thus terminated 
the Roman Empire in the West, twelve hundred and twenty- 
nine years after the building of Rome, and five hundred and 
seven years after the battle of Actium. Such, observes 
Goldsmith, was the end of this mighty Empire, which had 
conquered mankind by its arms, and instructed the world 
by its wisdom ; which had risen by temperance and fell by 
luxury; which had been established by a spirit of patriotism, 
and sunk to ruin when the Empire had become so extensive 
that a Roman citizen was but an empty name.* 



CHAPTER IX. 

ROMAN ANTIQUITIES. 



THE political state or government among the Romans 
varied very much during the successive periods of its 
existence. It was at first a monarchy. It afterwards be- 

■^For a fuller account of the Roman Empire, see Fredet's Modern 
History. 

What was done by Eudoxia? What is said of Genseric? — 13. From 
the reign of Valentinian, what is said of the Western empire? When 
did the empire terminate? Who was the last of the emperors? How 
long had the Roman empire lasted ? What does Goldsmith observe ? 

Chapter IX. — 1. What is said of the political state? What was it 
at first? 

12 



134 ROME. 

came a republic, with a preponderance of aristocratic power, 
which gradually gave way to the influence of the people. 
The republican form of government was overthrown by Ju- 
lius Caesar, and finally destroyed by Augustus, when it be- 
came a desp()tic monarchy. 

2. The Kings of Rome were not absolute or hereditary, 
but limited and elective. They could neither enact laws nor 
make war or peace, without the consent of the Senate and 
people. They wore a white robe, adorned with stripes of 
purple or fringed with the same color ; their crown was gold, 
and their sceptre was made of ivory. They sat in the curule 
chair, which was a chair of state made of ivory, and were 
attended by twelve lictors, carrying fasces, which were a 
bundle of rods, with an axe bound in the middle of them. 

3. The Senate at first consisted of one hundred members, 
but was afterwards increased to two hundred by Tarquin 
the elder, and towards the latter days of the republic it con- 
sisted of one thousand. The Senators were at first nomi- 
nated by the King, but afterv&gt;^ards chosen by the consuls, 
and finally by the censors. They usually assembled three 
times a month, but oftener if special business required it. A 
decree passed by a majority of the Senate, and approved 
of by the Tribunes of the people, was termed a senatus con- 
sultum. The Senators were styled patres, or fathers, on 
account of their age, gravity, and paternal care of the state, 
and from them the patricians derived their designation. The 
magistrates of the Roman republic were elective, and pre- 
vious to their election they were called candidati, or candi- 
dates, from the white robe which they wore while soliciting 
the votes of the people. 

4. The Consuls had the same badges as the Kings, with 
the exception of the crown ; and their authority was nearly 
equal, except that it was limited to one year. In dangerous 
conjunctures, they were clothed with absolute power by the 
solemn decree, " that the consuls take care that the common- 
wealth sustain no harm." In order to be a candidate for the 
consulship, the person was required to be forty-three years 
of age. The Praetors were next in dignity to the Consuls, 

What did it become? By whom was the repubhcan government 
overthrown ?— 2. What is said of the kings? What did they wear? 
By whom were they attended?— 3. Of what did the Senate consist? 
How often did they assemble? What was a decree termed? What 
were senators called ? — 4. What is said of the consuls ? In dangerous 
conjunctures, with what were they clothed ? What age was required ? 
Who were next in dig-nitv ? 



ROME. 135 

and in their absence supplied their place ; it was their duty 
to preside at the assemblies of the people, and to convene 
the senate upon any emerg-ency. 

5. The office of Censor was esteemed more honorable than 
that of consul, though attended with less power. There 
were two Censors, chosen every five years, and their most 
important duty was to take, every fifth year, the census of 
the people, after which they made a solemn lustration, or 
expiratory sacrifice in the Campus Martins. The Tribunes 
were officers, created to protect the plebeians against the pa- 
tricians. The Ediles were officers whose duty it was to take 
care of the public edifices, baths, aqueducts, roads, markets, 
etc. The Questors were elected by the people to take care 
of the public revenue. These were of two orders ; the mili- 
tary Questors, who accompanied the army, and took care of 
the payment of soldiers, and the provincial Questors, who 
attended the Consuls into the provinces and received the 
taxes and tribute. 

6. The assemblies of the people, in order to elect their 
magistrates, or to decide concerning war or peace, and the 
like, were called a comitia ; of which there were three kinds, 
the curiata, centuriala, and the tributa. The comitia curi- 
ata consisted of an assembly of the resident Roman citizens, 
who were divided into thirty curias. The comitia centuri- 
ata were the principal assembly of the people, in which they 
gave their votes according to the census. They elected, 
during these comitia, the consuls, praetors, and censors ; im- 
portant laws were enacted, and cases of high treason were 
tried ; and they were held in the Campus Martins. The 
comitia tributa were an assembly in which the people voted 
in tribes, according to their regions and wards ; and thej^ 
were held to create inferior magistrates, to elect certain 
priests, etc. The comitia continued to be assembled for 
upwards of seven hundred years, until the time of Julius 
Caesar, who abridged that liberty, and shared with the people 
the right of creating the magistrates. Augustus infringed 
still further on this right, and Tiberius finally deprived the 
people altogether of the privilege of election. 

t. The Priests among the pagan Romans did not form a 

5. What is said of the office of censor? Who were the tribunes? 
Who were the ediles? The questors? Of how many orders were 
they ? — 6. What was the assemblies of the people called ? Of Avhat did 
the comitia curiata consist? What was done at the comitia centuriata? 
What was the comitia tribnta f How long did they continue to assem- 
ble ? — 7. What is said of the priests ? 



136 ROME. 

distinct order of the citizens, but were chosen from the most 
honorable men of the state. The Pontifices, fifteen in num- 
ber, were priests who judged all causes relating to religion, 
regulated the feasts, sacrifices, and all other sacred institu- 
tions. The Pontifex Maxinius, or High-Priest, was a person 
of great dignity and authority ; he held his office for life, and 
all other priests were subject to him. The Augurs w^ere fif- 
teen in number, whose duty it was to foretell future events, 
to interpret dreams, oracles, prodigies, etc. The Haruspices 
were priests, whose business it was to examine the beasts 
offered in sacrifice, and from them to divine the success of 
any enterprise, and to obtain omens of futurity. The Quincle- 
cemviri were fifteen priests who had the charge of the Sib'yl- 
line books, which were three prophetic volumes, said to con- 
tain the fate of the Roman empire ; they were procured from 
a woman of extraordinary appearance during the reign of 
Tarquin the Proud. The Vestal Yirgins were females, con- 
secrated to the worship of Yesta. 

8. The Gladiators were persons who fought with weapons 
in the public circus or amphitheatre for the amusement of 
the people. These combats were introduced about four hun- 
dred years after the foundation of the city, and became the 
most favorite entertainment. The combatants were at first 
composed of captive slaves and condemned malefactors ; but 
in the more degenerate period of the empire free-born citizens, 
and even Senators, engaged in this inhuman and disgraceful 
amusement, in which numbers were destroyed. After the 
triumph of Trajan over the Dacians, spectacles were ex- 
hibited for one hundred and twenty-three days, in which 
eleven thousand animals of different kinds were killed, and 
ten thousand gladiators fought. 

9. The toga and the tunica were the most distinguished 
part of the Roman dress. The toga, or gown, worn by the 
Roman citizen only, was loose and flowing, and covered the 
whole body ; it had no sleeves, and was disposed in graceful 
folds, which gave the wearer a majestic appearance. The toga 
virilis was assumed by young men at the age of seventeen 
years. The tunica, or tunic, was a white woollen vest, 
which came down below the knees and was fastened about 
the waist by a girdle. The dress of the women was similar 

Of the Pontifices? Who was the Pontifex Maximusf The Augurs? 
The Haruspices ? The Quindecemviri ? The Vestal Virgins ?— 8. Who 
were the Gladiators? When were these combats introduced? What 
is said of them after the triumph of Trajan?— 9. What was the toc/af 
The toga virilis f The tunica ? What is said of the dress of women ? 



ROME. 137 

to that of the men ; their tunic, however, was longer and 
furnished with sleeves ; they wore jewels, bracelets, rings, 
and various other ornaments in great profusion. Hats and 
caps were worn by the Romans only on journeys, or at the 
public games ; in the city they usually went without any 
covering on the head. 

10. The principal meal among the Romans was their sup- 
per, which they took about four o'clock in the afternoon. The 
breakfast was not a regular meal : it was taken by each one 
separately and without order, and their dinner was only a 
slight repast. In the early ages the diet of the Romans con- 
sisted chiefly of milk and vegetables, and they sat upright at 
the table on benches ; but in the latter days of the republic, 
when riches were introduced by their conquests, luxury was 
carried to excess, and they then reclined at their meals on 
sumptuous couches. These couches were similar to the 
modern sofa, and generally intended to hold three persons. 
People so reclined upon them that the head of the one was 
opposite the breast of the other, and in serving themselves 
the}^ used only one hand. 

11. Fathers at Rome were generally invested with the 
power of life and death over their children. The exposure 
of infants was at first somewhat frequent, but at length en- 
tirely ceased. Slaves constituted a large portion of the pop- 
ulation of Rome. Their lives were at the disposal of their 
masters. They were not only employed in domestic service, 
but also in various trades and manufactures. They were 
considered as mere property, and were publicly sold in the 
market-place ; and if capitally convicted, their punishment 
was crucifixion. At the feasts of Saturn and at the Ides of 
August the slaves were allowed great privileges, and masters 
at those periods waited on th^m at table. 

12. The system of education among the Romans, which 
was in its highest state of improvement during the reign of 
Augustus, was much admired. The utmost attention was 
bestowed on the early formation of the mind and character. 
The Roman matrons themselves nursed their own children, 
and next to the care bestowed on their morals, a remarkable 
degree of attention seems to have been given to their lan- 
guage. From the earliest dawn of reason a regular course 

Of hats, etc.? — 10. What was the principal meal? What is said of 
breakfast ? In the early ages, what was the diet of the Komans ? How 
did they sit at table? What is said of these couches? — 11. What is 
said of fathers ? Of infants ? What is said of slaves? — 12. What is said 
of education ? Of the Roman matrons ? 
12* 



138 MYTHOLOGY OF ANCIENT NATIONS. 

of discipline was pursued by some matrou of the family, and 
as the children grew towards manhood they were habituated 
to all the athletic exercises that could impart agility or grace, 
and fit them for the profession of arms. Eloquence and the 
military art were the surest road to preferment. Oratory, 
which led to the highest honors in the state, was the favorite 
study at Rome, and was taught as a science in the public 
schools. In this art the name of Cicero stands pre-eminent. 
But Roman prose-writing reaches its highest perfection in 
the historical works of Lwy, Caesar, and Tacitus. Poetry 
among the Romans, as with most of other nations, appears 
to have been the earliest intellectual eifort. The names that 
adorn the Roman drama are those of Liv'ius Andron'ictis, 
En'nius, Plau'tus, and CseciVius. In epic poetry, Vir'gil 
has excelled all other poets of ancient times, with the excep- 
tion of Homer. Philosophy was first taught at Rome 
about the end of the third Panic war, and was introduced 
from Greece. The system of the Stoics was at first most 
generally received ; the philosophy of Aristot'le was after- 
wards greatly cultivated ; but with the introduction of lux- 
ury the philosophy of Epicu'rus became fashionable. 



CHAPTER X. 

' MYTHOLOGY OF ANCIENT NATIONS. 

ALL the nations of antiquity, except the Jews, were 
heathens and idolaters. Their system of religion was 
called Polytheism, as it acknowledged a plurality of gods, 
and they worshipped their divinities by various images called 
idols. The first objects of adoration among the pagan na- 
tions, after they had lost the correct knowledge of the true 
God, were the heavenly bodies. Hence we find that the 
names of the principal gods correspond with the names of 
the chief planets, such as Sat'urn, Ju'piter, Ve'nus, etc. 
Osi'ris and I' sis, the principal deities among the Egyptians, 
are supposed to have been the sun and moon. In the pro- 

What was pursued ? What is said of eloquence, etc. ? Of orator}' ? 
Of poetry? 

Chapter X. — 1. What were all the nations of antiquity? What 
was their system called? What were the first objects of adoration? 
Vv^hat do we find ? In the process of time, what did they do ? 



MYTHOLOGY OF ANCIENT NATIONS. 139 

cess of time, they built temples to the heavenly bodies, as 
being subordinate agents of the divine power, and by wor- 
shipping them, they supposed they would obtain the favor of 
the Deity. From this they descended to the worship of ob- 
jects on the earth, as they were thought to represent the 
stars or the Deity. Thus idolatry arose shortly after the 
Deluge. 

2. In the course of time, adoration was bestowed on those 
objects which were thought to confer peculiar benefits on man. 
Thus the Egyptians regarded the Nile as sacred, because 
by its inundations it fertilized the earth. Again, great heroes 
and persons, who, during their lives, had been benefactors 
to the human race, were deified after their death. From 
these, the ancient pagans descended to the worship of the 
most degrading objects, and paid divine honors to beasts, 
birds, insects, and even to vegetables, such as leeks and 
onions ; moreover, temples were dedicated to evil demons and 
the most debasing passions. 

3. The Babylonians adored the heavenly bodies, and 
among them Jupiter was worshipped, under the name of 
Begins, to whom magnificent temples were erected at Baby- 
lon. The Ca'naanites and Syrians worshipped Ba'al, Tarn'- 
muz, Ma'gog, and As'tarte. Mo'loch was the Saturn of the 
Phoenicians and Carthaginians, to whom human victims, 
particularly children, were immolated. Baal-peor was the 
idol of the Mo'abites ; his rites were degrading and cruel. 
Da'gon was the chief god of the Philistines ; his figure was a 
compound of a man and a fish. Among the Celts, the sacred 
rites were performed in groves dedicated to their gods, to 
whom human victims were frequently offered ; colossal 
images of wicker-work were filled with human criminals and 
consumed by fire. 

4. According to the pagan theology, there were twelve 
chief deities engaged in the creation and government of the 
universe. Agreeably to this theory, Jupiter, Nep'tune, and 
VuVcan fabricated the world : Ce'res, Ju'no, and Dia'na 
animated it ; Mer'cury, Ve'nus, and ApoVlo harmonized it ; 
and lastly, Ves'ta, Miner' va, and Mars presided over it with 
a guardian power, and these twelve were called the celestial 
deities. 

2. What did the Egyptians regard ? Why ? To what was divine 
honors paid ? — 3. What is said of the Babylonians ? What was Mo- 
loch ? Baal-peor? Dagon? Among the Celts, where were the sacred 
rites performed? — 4. Agreeable to this theory what is said of the world? 
What were these twelve called ? 



140 MYTHOLOGY OF ANCIENT NATIONS. 

Jupiter, who was represented as supreme, and styled the 
father of the gods and men, was the son of Saturn and 
Cyh'ele, and was born on Mount Ida, in Crete. He deposed 
his father, and divided the world between himself and his 
two brothers, Neptune and Pluto. Neptune had the juris- 
diction over the sea, and Pluto that of the infernal regions ; 
but the sovereignty of heaven and earth he reserved to him- 
self. One of his chief exploits was the conquest of the 
Ti'tans or giants, who are said to have placed several moun- 
tains on each other, in order to scale the heavens. He is 
generally represented as a majestic personage, seated upon 
a throne, with a sceptre in one hand and thunderbolts in the 
other. The heavens trembled at his nod, and he governed all 
things except the Fates. 

5. ApoFlo was the son of Jupiter^ and Lato'na, and was 
born on the island of Delos. He presided over music, medi- 
cine, poetry, the fine arts, and archery. For his offence in 
killing the Cy'clops, he was banished from heaven, and 
obliged to hire himself as a shepherd to Adme'tus, King of 
Thessaly, in which employment he remained for nine years. 
His exploits are represented as extraordinary ; among others 
he caused Mi'das to receive a pair of asses' ears, for pre- 
ferring Pan''s music to his; he turned into a violet the 
beautiful boy Hyacinth, whom he accidentally killed ; and 
changed Daph'ne into a laurel. 

6. Mars was the son of Jupiter and Juno. He was the god 
of war, and the patron of all that is cruel and furious ; the 
horse, the wolf, the magpie, and vulture were offered to him. 
During the Trojan war. Mars was wounded by Diome'des^ 
and retreating to heaven, he complained to Jupiter that Mi- 
nerva had directed the weapon of his antagonist. He is 
represented as an old man, armed and standing in a chariot 
drawn by two horses, called Fright and Terror. His sister 
Bello'na was his charioteer. Discord went before him in a 
tattered garment with a torch, Anger and Clamor followed. 

T. Mer'cury, the son of Jupiter and Mai' a, was the mes- 
senger of the gods, and the patron of travellers, shepherds, 
orators, merchants, thieves, and dishonest persons. He was 
doubtless some enlightened person, in a remote age, who, on 

What is Jupiter styled ? What did he do ? What is one of his chief 
exploits ? How is he represented ? — 5. Who was Apollo ? Over what 
did he preside ? From where was he banished ? What were his ex- 
ploits ? — 6. Who was Mars ? Of what was he the god ? During the 
Trojan war what is said of him ? How is he represented ? — 7. What 
was Mercury? 



MYTHOLOGY OF ANCIENT NATIONS. 141 

account of his actions and services, was worshipped after his 
death. He seems to have been the first who taught the arts 
of civilization. 

VuKcan, the son of Jupiter and Juno, was the g-od of fire, 
and the patron of those who wrought in the metallic arts. 
He was kicked out of heaven by Jupiter, for attempting to 
deliver his mother from a chain by which she was suspended. 
He continued to descend for nine successive da3^s and nights, 
and at length fell upon the isle of Lemnos, but was crippled 
by the fall. He was the artificer of heaven, and forged the 
thunderbolts of Jupiter, also the arms of the gods. 

8- Juno, styled the queen of heaven, was both the sister 
and wife of Jupiter. In her character she was haughty, 
jealous, and inexorable. In her figure she was lofty, grace- 
ful, and majestic. Iris, displaying the rich colors of the 
rainbow, was her usual attendant. 

Minerva, the goddess of wisdom, was the most accom- 
plished of all the goddesses, and the only divinity that seems 
equal to Jupiter. She is said to have instructed man in the 
arts of shipbuilding, navigation, spinning, and weaving. Her 
worship was universally established, but at Athens it claimed 
particular attention. The owl was sacred to her. 

Venus, the goddess of love and beauty, was the daughter 
of Jupiter and Dio'ne, or, as some say, she sprurfg from the 
foam of the sea. Her worship was licentious in a high de- 
gree, and attended with the most disgraceful ceremonies. 

Dia'na was the queen of the woods and the goddess of 
hunting. She devoted herself to perpetual celibacy, and 
was attended by eighty nymphs. The poppy was sacred 
to her. 

9. Ce'res, the daughter of Saturn and Cyhele, was the 
goddess of corn and harvest, and the first who taught the 
cultivation of the earth. The Eleusin'ian Mysteries were 
celebrated in her honor. 

Yesta was the goddess of fire and the guardian of houses. 
She was represented in a long flowing robe, a veil on her 
head, a lamp in one hand, and a javelin in the other. 

10. Neptune, the brother of Jupiter, was the second in 

What did he teach ? Who was Vulcan ? What is said of him ? 
What did he forge ? — 8. What was Juno ? What was she in her char- 
acter? In her figure? What was Minerva? What is said of her? 
What was sacred to her? Who was Venus ? What is said of her wor- 
ship? Who was Diana? — 9. Who was Ceres? What were celebrated 
in her honor ? What was Vesta ? How was she represented ? — 10. What 
was Neptune ? 



142 MYTHOLOGY OF ANCIENT NATIONS. 

rank among the gods, and reigned over the sea. He is rep- 
resented seated on a chariot drawn by dolphins and sea- 
horses ; in his hand he holds a trident or sceptre, with three 
prongs. Oce'anus, a sea god, was called the father of rivers. 
Tri'ton, also a marine deity, was the son of Neptune and 
Amphritite ; he was his father's companion and trumpeter. 
Ne'reus, a sea god, the son of Oceanus, was the father of 
fifty daughters, who were called Nereides. Pro'tevs, the 
son of Oceanus, could foretell future events, and change 
himself into any shape. 

11. The infernal deities were Pluto and his consort Pros' - 
erpine, Plii'tus, Cha'ron, the Furies, Fates, and the three 
judges, Mi'nos, uE'acus, and Rhadaman'thus. Pluto, who 
exercised dominion over the infernal regions, was the brother 
of Jupiter. The goddesses all refusing to marry him, on 
account of his deformity and gloomy disposition, he seized 
upon Proserpine, the daughter of Ceres, in Sicily, opened a 
passage through the earth, and carried her to his residence ; 
and having married her made her queen of hell. There were 
no temples raised to his honor. Plutus, an infernal deity, 
was the god of riches ; he was lame, blind, injudicious, and 
timorous. 

12. Cha'ron was the ferryman who conducted the ghosts 
across the river Le'the, on their way to Pluto's regions. He 
is represented as an old man, with white hair, a long beard, 
and garments deformed with filth, and remarkable for the 
harshness of his speech and ill temper. None could enter 
Charon's boat if they had not received a regular burial; 
without this, they were supposed to wander a hundred years 
amidst the mud and slime of the shore. Each ghost paid a 
small brass coin for his fare. 

13. The Furies were three in number, namely : Alec' to, 
Tisiph'one, and Megse'ra. They had the faces of women, 
but their looks were full of terror — they held lighted torches 
in their hands, and snakes lashed their necks and shoulders. 
Their office was to punish the crimes of wicked men, and to 
torment the consciences of secret ofi'enders. 

The Fates were three daughters of Jupiter and Themis. 
Their names were Clo'tho, Lach'esis, and At'ropos. They 

How is she represented ? Who was Oceanns ? Triton ? Nereus ? Pro- 
teus?— 11. Name the infernal deities. What is said of Pluto? Who 
was Plutus ? What was he ?— 1 2. Who was Charon ? How is he rep- 
resented ? What is said of those who did not receive a regular burial ? 
—13. Name the Furies. What had they? What was their office? 
Name the Fates. 



MYTHOLOGY OF ANCIENT NATIONS. 143 

decided on the fortunes of mankind. Clotho drew the thread 
of life. Lachesis turned the wheel ; and Atropos cut it 
with her scissors. The duty of the three judges was to 
assign the various punishments of the wicked, adapted to 
their crimes, and to place the good in the delightful realms 
of Elys'ium. 

14. There were many other divinities of various charac- 
ters, such as Bac'chus, Cupid, the Muses, the Graces, etc. 
Bacchus, the son of Jupiter and Semele, was the god of 
wine. His festivals were celebrated by persons of both 
sexes, who dressed themselves in skins, and ran shouting 
through the hills and country places ; these solemnities were 
attended with the most disgusting scenes of intoxication and 
debauchery. The fir, the fig-tree, ivy, and vine were sacred 
to him. 

Cupid, representing the passion of love, was a beautiful 
winged boy ; often with a bandage over his eyes, also with 
a bow and arrow in his hand, with which to wound the 
hearts of mortals. 

The Muses were nine in number, namely : Calli'ope, who 
presided over eloquence and heroic and epic poetry ; Clio 
presided over history ; Er'ato was the Muse of elegiac and 
lyric poetry ; Euter'pe presided over music ; Melpom'ene 
was the inventress and muse of tragedy ; Polyhym'nia was 
the muse of singing and rhetoric; Terpsich'ore presided 
over dancing ; Thali'a, the muse of pastoral and comic 
poetry ; and Ura'nia, who presided over hymns and sacred 
subjects ; and also the muse of astronomy. 

15. The Graces were the three daughters of Bacchus and 
Venus. They were supposed to give to beauty all its charms 
of attraction. Besides these, there were several rural deities, 
such as Pan, the god of shepherds and hunters ; Sylva'nus, 
who presided over the woods ; -Fria'pus, the god of the 
gardens ; Ter'minus, who was considered as watching over 
the boundaries of land, and others. 

The Si'rens were three fabulous persons, who are said to 
have had the faces of women, and the lower parts of their 
bodies like a fish. They had such melodious voices that 
mariners were often allured and destroyed by them. The 

What did each one do ? What was the duty of the judo^es ?— 14. Name 
some of the other divinities. What was Bacchus ? What is said of 
his festivals ? What is said of Cupid ? What was the number of the 
Muses, and over what did they preside?— 15. Who were the Graces? 
What was Pan? Sylvanus? Priapus? Terminus? What were the 
Sirens ? 



144 MYTHOLOGY OF ANCIENT NATIONS. 

Gor'gons were three sisters, who are said to have had the 
power of transforming those into stones who looked upon 
them. The Har'pies were winged monsters, which had the 
face of a woman, the body and wings of a vulture, claws on 
the hands and feet, and the ears of a bear. 

16. The objects of worship among the ancient nations, 
particularly among the Greeks and Romans, are said to have 
amounted to thirty thousand. To these temples were erected, 
festivals instituted, games celebrated, and sacrifices offered, 
with a greater or less degree of pomp, according to the de- 
gree of estimation in which the deity was held. The most 
celebrated temples of antiquity were those of Dia'na at 
Eph'esus, of Apollo, in the city of Mile'tus, of Ceres and 
Proserpine, at JEleusis, and that of Jupiter Olympus, and 
the Parthenon of Minerva, at Athens. The famous temple 
of Diana, at Ephesus, one of the seven wonders of the world, 
was completed two hundred and twenty years after its foun- 
dation. It was four hundred and twenty-five feet in length, 
and two hundred in breadth : the roof was supported by one 
hundred and twenty-seven columns, sixty feet high, placed 
there by so many kings. This temple was burnt on the 
night that Alexander the Great was born, by ErosHratus, 
who alleged that he perpetrated the deed merely for the pur- 
pose of immortalizing his name in destroying so magnificent 
a building. 

n. Oracles were consulted, particularly by the Greeks 
and Romans, on all important occasions, and their determi- 
nations were held sacred and inviolable. The most cele- 
brated oracles were those of Apollo, at Delphi and Belos ; 
the oracles of Jupiter, at Dodo'na; and that of Tropho'- 
nius, where future events were made known to those who 
sought to know the will of the gods. The responses were 
generally delivered by a priestess, who was supposed to be 
divinely inspired ; but usually in verse, and couched in very 
am-biguous language, so that one answer would agree with 
various and sometimes opposite events. It must, however, 
be confessed that sometimes the answers of the oracles were 
substantially correct, a fact which is proved by many pas- 
sages in ancient history ; but it is a question among the 
learned, whether the answers of the oracles should be ascribed 

The Gorgons? The Harpies?— 16. What was the number of objects 
of worship ? Name the most celebrated temples. What is said of the 
temple of Diana at Ephesus? By whom was it burnt? — 17. What is 
said of Oracles? Which were the most celebrated? How were the 
responses given? What is a question among the learned? 



MYTHOLOGY OF ANCIENT NATIONS. 145 

to the operations of demons, or only to the imposture of men. 
The best established opinion is, that demons were the real 
agents in the oracles, although we find many instances in 
Grecian history where the Delphic priestess suffered herself 
to be corrupted by presents, and gave an answer to suit the 
will or to gratify the passions or inclinations of those who 
came to consult her. 

18. There is one fact, however, deserving of notice, namely, 
that the responses of the oracles ceased when the Christian 
religion began to be preached — not on a sudden, but in pro- 
portion as its salutary doctrines became known to mankind. 
Tertul'lian, in one of his apologies, challenges the pagans 
to make the experiment, and consents that a Christian should 
be put to death if he did not oblige the oracles to confess 
themselves 'devils. Lactan'tius informs us that every Chris- 
tian could silence the oracles only by making the Sign of the 
Cross. When Julian, the Apostate, went to Daphne, near 
Antioch, to consult Apollo, the god, notwithstanding all the 
sacrifices offered to him, continued mute, and only recovered 
his speech to answer those who inquired the cause of his 
silence, and ascribed it to the interment of certain Christian 
bodies in the neighborhood. 

19. The ancients generally inculcated the belief in a future 
state of existence, believing that the virtuous would be happy 
in Elysium, or Paradise, and that the wicked would be 
miserable in Tar'tariis, or Hell. Of hell, they drew the 
most gloomy and horrible picture. It was a place where 
men, who had been remarkable for their crimes while on 
earth, were punished with a variety of tortures. On the 
other hand, the prospect of Elysium was described as beau- 
tiful and inviting in the highest degree. In that delightful 
region there was no inclement weather, but mild winds con- 
stantly blew from the ocean to refresh the inhabitants, who 
lived without care or anxiety; the sky was perpetually serene, 
and the fertile earth produced, twice a year, delicious fruit 
in abundance. 

• 
What is the best established opinion? — 18. What fact deserves no- 
tice? Of what does Lactantins inform us? What is related of Julian? 
— 19. Of what did the ancients inculcate the belief? Of hell, what pic- 
ture did they draw ? How was Elysium described ? 
13 K 



