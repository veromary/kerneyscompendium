\documentclass[openany,a4paper]{memoir}
%\usepackage{mathspec}
\usepackage{libertine}
%\usepackage{fontspec}

\title{Book III: Greece}

\author{M. J. Kerney}

\begin{document}

\maketitle

\tableofcontents

\chapter{Glances at Early Greek History.} 

AMONG the various nations of antiquity, Greece deservedly holds the most distinguished rank, both for 
the patriotism, genius, and learning of its inhabitants, as 
well as the high state of perfection to which they carried the 
arts and sciences. 

It formerly comprised various small independent states, 
differing from each other in forms of government and in 
the character of the people, but still united in a confederacy 
for mutual defence, by the counsel of \emph{Amphic$\,'$tyons}, and by 
their common language, religion, and public games. 

2. The name \emph{Greece} was never used by the ancient inhabitants of that country. They called their land \emph{Hel$\,'$las}, and 
themselves \emph{Hel$\,'$lenes}. It is from the Romans that we have 
derived the word Greece; but why they gave it a different 
appellation from that used by the natives cannot be determined. The original inhabitants, who were generally considered as the descendants of \emph{Ja$\,'$van}, the son of Japhet, 
lived in the lowest condition of barbarism, dwelling in huts, 
feeding on acorns and berries, and clothing themselves in 
the skins of wild beasts, when \emph{Ce$\,'$crops} with a colony from 
Egypt, and \emph{Cadmus} with a body of \emph{Ph\oe ni$\,'$cians}, landed in 
Greece, and planted on its shores the first rudiments of 
civilization. 

The early form of government in Greece was a limited 
monarchy, which was finally abolished, and a republican form 
generally prevailed. 

3. The history of this famous land may be divided into 
two parts : 1st, the period of uncertain history, which extends from the earliest accounts of the country to the first 
Persian war, in the year 490 b. c.; 2d, the period of authentic 
history, extending from the Persian invasion to the final 
subjugation of Greece by the Romans, BC 146. The first 
period is generally reckoned from the foundation of Sic$'$yon, 
the most ancient kingdom of Greece, and comprises a space 
of about sixteen hundred years. This long succession of ages, 
though greatly involved in obscurity and fable, is still interspersed with several interesting particulars. It contains no 
records that properly deserve the name of history. 

4. Grecian history, however, derives some authenticity at 
this period from the \emph{Chronicle of Paros}, preserved among 
the \emph{Arundelian} marbles at Oxford. The authority of this 
chronicle has, indeed, been much questioned; but still, by 
many, it is thought to be worthy of considerable credit. It 
fixes the dates of the most important events in the history 
of Greece, from the time of \emph{Cecrops} down to the age of 
\emph{Alexander the Great}. 

5. \emph{Sic$\,'$yon}, the capital of the ancient kingdom of that 
name, was founded by \emph{\AE gi$\,'$alus}; \emph{Argos} by \emph{In$\,'$achus}, the 
last of the \emph{Ti$\,'$tans}; Athens, which afterwards bore such a 
distinguished part in the history of Greece, was founded by 
\emph{Cecrops}, with a colony from Egypt. He was an eminent 
legislator, and instituted the court of \emph{Areop$\,'$agus}. \emph{Thebes} 
was founded by \emph{Cadmus}, who is said to have introduced 
letters into Greece from Phoenicia; the alphabet, however, 
only consisted of sixteen letters, and the mode of writing 
was alternately from right to left, and from left to right. 

6. In the time of \emph{Cranaus}, who succeeded \emph{Cecrops}, happened the deluge of \emph{Deuca$\,'$lion}, which, though much magnified by the poets, was probably only a partial inundation. 

The other memorable institutions that distinguish this 
period were the \emph{Eleusin$\,'$ian} mysteries, the \emph{Olymp$\,'$ic} and 
other games,---of which we shall speak hereafter,---and the 
marvellous exploits of \emph{Her$\,'$cules} and \emph{The$\,'$seus}. 


\section{Review Questions}

1. What is said of Greece? What did it formerly comprise? How were they united?

2. What was its ancient name? From 
whom were the inhabitants descended? What was their condition when 
Cecrops landed in Greece? 

3. How is the history of Greece divided? How do these periods extend? What is said of the first period? 

4. From what does the Grecian 
history derive authenticity? Of what does this chronicle fix the date?  

5. By whom was Sicyon founded? Argos? Athens? Thebes? What is 
said of Cadmus?  

6. In the time of Cranaus, what happened? What 
institutions distinguished this period? 





\chapter{The Fabulous and Heroic Ages.} 

THE fabulous age comprises the period from the foundation of the principal cities to the commencement of civilization, and the introduction of letters and arts into Greece. 
The first great enterprise undertaken by the Greeks was the 
Argonaut$'$ic expedition, which appears in its details to partake more of fable than of history. It was commanded by 
Ja$'$son, the son of the king of Iolchos, who was accompanied by many of the most illustrious men of Greece, 
among whom were Her$'$cules, The$'$seus, Cas$'$tor and Pollux, Or$'$pheus, AEsculap$'$ius the physician, and Chi$'$ron the 
astronomer. 

2. They sailed from lolchos, in Thes$'$saly, to Col$'$chis, on 
the eastern coast of the Eux$'$ine Sea; they received the 
name Argonauts from the ship Ar$'$go in which they sailed, 
said to have been the first sea vessel ever built. This famous 
voyage, which was probably a military and mercantile adventure, is commonly represented to have been undertaken 
for the purpose of recovering the golden fleece of a ram, 
which originally belonged to their country. The fleece is 
pretended to have been guarded by bulls that breathed fire, 
and by a dragon that never slept. 

3. The Heroic Age was particularly distinguished by the 
Tro$'$jan war, the history of which rests on the authority of 
Homer, and forms the subject of his Il$'$iad,\footnote{From Ilium, or Troy.} the noblest poem 
of antiquity. According to the poet, HeVlen, the daughter 
of Tyn$'$darus, king of Sparta, was reputed the most beautiful woman of her age, and her hand was solicited by the 
most illustrious princes of Greece. Her father bound all her 
suitors by a solemn oath, that they would abide by the choice 
that Hellen should make of one among them; and that, 
should she be taken from the arms of her husband, they would 
assist, to the utmost of their power, to recover her. 


4. Hellen gave her hand to Menelaus, and after her nuptials, Tyndarus, her father, resigned the crown to his son-in-law. Paris, the son of Pri$'$am, king of Troy, a powerful city 
founded by Dar$'$danus, having adjudged the prize of superior beauty to Venus, in preference to Juno and Minerva, was 
promised by her the most beautiful woman of the age for 
his wife. Shortly after this event, Paris visited Sparta, 
where he was kindly received by Menelaus; but in return 
for the kind hospitality tendered to him, he persuaded Hellen 
to elope with him to Troy, and carried off with her a considerable amount of treasure. 

5. This act of treachery and ingratitude produced the Trojan war. A confederacy was immediately formed by the 
princes of Greece, in accordance with their engagement, to 
avenge the outrage. An army of one hundred thousand men 
was conveyed in a fleet of twelve hundred vessels to the 
Trojan coast. Agamem$'$non, king of Argos, brother of Menelaus, was selected as commander-in-chief. Some of the other 
princes most distinguished in this war, were Achilles, the 
bravest of the Greeks; also Ajax, Menelaus, Ulys$'$ses, Nes$'$tor, and Diome$'$des. 

6. The Trojans were commanded by Hec$'$tor, the son of 
Priam, assisted by Paris, Deiph$'$ohus, AE$'$neas, and Sarpe$'$don. After a siege of ten years, the city was taken by 
stratagem, plundered of its wealth, and burnt to the ground.\footnote{Much light has been thrown on the ruins of Troy by the recent researches of Dr. Schliemann.} 
The venerable Priam, king of Troy, was slain, and all his 
family carried into captivity. About eighty years after the 
destruction of Troy, the civil war of the Heracli$'$dae began; 
it is usually called the return of the Heraclidae into Peloponne$'$sus. Hercules, king of Myce$'$nae, a city of Peloponnesus, 
was banished from his country, with all his family, while 
the crown was seized by Atre$'$us, the son of Pe$'$lops. After 
the lapse of about a century, the descendants of Hercules returned to Peloponnesus, and, having expelled the inhabitants, 
again took possession of the country. 



\section{Review Questions}



1. What do the fabulous ages comprise? What was the 
first great enterprise? Who commanded it? and who accompanied 
him? 

2. From where did they sail? For what was this famous voyage 
undertaken?

3. For what is the Heroic Age distinguished? What is 
said of Hellen? How did her father bind all her suitors? 

4. To whom did Hellen give her hand? What is said of Paris? 

5. What did this treachery produce? Who was commander-in-chief of 
the Grecian forces? Mention the other princes.

6. By whom were 
the Trojans commanded? What is said of the city? Of Priam? What 
happened about eighty years after this? What is said of Hercules? 



\chapter{The Republic of Sparta.} 

SPAR$'$TA, or Lacedae$'$mon, was the capital of Laco$'$nia, in the 
southern part of Peloponne$'$sus. After the return of the 
Heraclidae, the government was administered by the two sons 
of Aristode$'$mus, who reigned jointly; and this double monarchy was transmitted to the descendants of each for a period 
of eight hundred and eighty years. 

2. This radical principle of disunion, and consequently of 
anarchy, made the want of a regular system of laws severely 
felt. Lycur$'$gus, the brother of one of the kings of Sparta, 
a man distinguished alike for his great abilities and stern integrity, was invested, by the united voice of the sovereigns 
and the people, with the important duty of framing a new 
constitution for his country. The arduous task being at 
length completed, produced not only an entire change in the 
form of government, but also in the manners of the people. 
He instituted an elective senate, consisting of twenty-eight 
members, whose office was to preserve a just balance between 
the power of the kings and that of the people. Nothing 
could come before the assembly of the people which had not 
received the previous consent of the senate; and, on the other 
hand, no action of the senate was effectual without the sanction of the people. The kings were continued, but were 
nothing more than hereditary presidents of the senate and 
generals of the army. 

3. Lycurgus divided the territory of the republic into 
thirty-nine thousand equal portions among the free citizens. 
For the purpose of banishing luxury, commerce was abolished. Gold and silver coin was prohibited, and iron money 
was substituted as a medium of exchange. A uniformity of 
dress was established, and all the citizens, not excepting the 
kings, were required to take their principal meals at the public 
tables, from which all luxury and excess were excluded, and 
a kind of black broth was the chief article of food. Among 
some of the admirable ceremonies which prevailed at these 
public meals, the following is interesting and instructive. 
When the assembly was seated, the oldest man present, pointing to the door, said, ``No word spoken here goes out there.$'$$'$ 
This wise regulation produced mutual confidence, and rendered the people unrestrained in conversation. 

4. The institutions of Lycurgus, though in many respects 
admirable, had still a number of grave defects. Infants, shortly 
after birth, underwent an examination, and those that were 
well formed were delivered to public nurses; but all who 
were deformed or sickly were inhumanly exposed to perish. 
At the age of seven, children were sent to the public schools. 
The young were taught to pay the greatest respect to the 
aged and cherish an ardent love for their country, and the 
profession of arms was looked upon as the great business of 
life. Letters were only taught in so far as they were useful; 
hence the Spartans, while they were distinguished for many 
heroic virtues, were never eminent for learning. No production from the pen of a native of Sparta has come down to 
modern times. These hardy people were accustomed to express themselves in short, pithy sentences, so that even at 
the present time this style of speaking is called after them 
laconic---Laconia being one of the names of their country. 

5. The youth were early inured to hardship; and were 
accustomed to sleep on rushes, trained to the athletic exercises, and only supplied with plain and scanty food. They 
were even taught to steal whatever they could, provided 
they could accomplish the theft without being detected. 
Plutarch relates the fact of a boy who had stolen a fox and 
concealed it under his garments, and who actually suffered 
the animal to tear out his bowels, rather than discover the 
theft. The women of Lacedaemon were destitute of the 
milder virtues that most adorn the female character, and 
their manners were highly indelicate. Their education was 
intended to give them a masculine energy, and to fill them 
with admiration of military glory. Mothers rather rejoiced 
than wept when their sons fell nobly in battle. ``Return 
with your shield or on your shield,'' was the injunction of a 
Spartan mother to her son, when he was going to meet the 
enemy. She meant that he should conquer or die. 


6. For five hundred years the institutions of Lycurgus 
continued in force. During this period the influence of 
Sparta was felt throughout Greece; and her government 
acquired solidity, while the other states were torn by domestic dissensions. In the process of time, however, the 
severe manners and rigid virtues of her citizens began to 
relax; changes in her laws and institutions were finally introduced, particularly during the reign of Lysander, whose 
conquests filled the country with wealth. 

From this period luxury and avarice began to prevail, 
until Sparta, with the other states of Greece, sunk under the 
dominion of Philip, king of Macedon. 


\section{Review Questions}


1. What was Sparta? What is said of the government after the return of the Heraclidse?  

2. What is said of Lycurgus? 
With what was he invested? What did he institute? What is said of the 
kings?  

3, How did Lycurgus divide the territory? What is said of commerce? Of gold and silver? Of iron money? Of dress? Of public tables? 
What was said by the oldest man present?  

4, What is said of the 
institutions of Lycurgus? How were infants treated? What were the 
young taught to pay? What is said of letters? How were they accustomed to speak?

5. What is said of the youth? Of the manners of 
the women? What is said of mothers? 

6. How long did the institutions of Lycurgus continue? In the process of time, what took place? What is said of Sparta from this period? 



\chapter{The Republic of Athens.}

ATH$'$ENS, the capital of At$'$tica, was distinguished for 
its commerce, wealth, and magnificence, and as the seat 
of learning and the arts. The last king of Athens was Co$'$drus, who sacrificed himself, for the good of his country, in 
a war with the Heraclidae. After his death, no one being deemed worthy to succeed him, the regal government was 
abolished, and the state was governed by magistrates, styled 
archons. At first the office was for life, but it was afterwards reduced to a period of ten years; and finally the 
archons, nine in number, were annually elected, and were 
possessed of equal authority. 

2. As these changes produced convulsions in the state, 
and rendered the condition of the people miserable, the 
Athenians appointed Dra$'$co, a man of stern and rigid principles, to prepare a code of written laws. His laws were 
characterized by extreme severity. Every crime was punished with death. Draco being asked why he was so severe 
in his punishment, replied that the smallest offence deserved 
death, and that he had no higher penalty for the greatest 
crime. The severity of these laws prevented them from 
being fully executed, and at length caused them to be 
entirely abolished, after a period of one hundred and fifty 
years. 

3. So$'$lon, one of the seven wise men of Greece, being 
raised to the archonship, was intrusted with the care of 
framing a new system of laws for his country. His disposition was mild and condescending; and, without attempting to change the manners of his countrymen, he endeavored 
to accommodate his system to their prevailing customs, to 
moderate their dissensions, to restrain their passions, and to 
open a field for the growth of virtue. Of his laws he said, 
``If they are not the best possible, they are the best the 
Atheniaus are capable of receiving.''

4. Solon$'$s system divided the people into four classes, 
according to their wealth. To the first three, composed of 
the richest citizens, he intrusted all the offices of the commonwealth. The fourth class, which was more numerous 
than the other three, had an equal right of suffrage in the 
public assembly, where all laws were framed and measures 
of state decreed; and by this regulation the balance of power 
was thrown in favor of the people. He instituted a senate 
composed of four hundred, and afterwards increased it to 
five hundred persons. He restored the court of the Areop$'$agus, which had greatly fallen into disrepute, and committed to it the supreme administration of justice. Commerce and agriculture were encouraged. Industry and 
economy were enforced. And the father who had taught 
his son no trade could not claim a support from him in his 
old age. 

5. The manners of the Athe$'$nians formed a striking contrast with those of the Lacedaemonians. At Athens the 
arts were highly esteemed; at Sparta they were despised 
and neglected. At Athens peace was the natural state of 
the republic, and the refined enjoyments of life the aim of 
its citizens; Sparta was entirely a military establishment; 
her people made war the great business of life. Luxury 
characterized the Athenian, frugality the Spartan. They 
were both, however, equally jealous of their liberty and 
equally brave in war. 


6. Before the death of Solon, Pisistratus, a man of great 
wealth and eloquence, by courting the popular favor, raised 
himself to the sovereign power, which he and his sons retained for fifty years. 

He governed with great ability, encouraged the arts and 
sciences, and is said to have founded the first public library 
known in the world, and first collected the poems of Homer 
into one volume, which, before that time, were repeated in 
detached portions. 

Pisistratus transmitted his power to his sons, Hip$'$pias 
and Hippar$'$chus. They governed for some time with wisdom 
and moderation, but having, at length, abused their power, a 
conspiracy was formed against them, and their government 
was overthrown by Harmo$'$dius and Aristogit$'$on. Hipparchus was slain. Hippias fled to Darius, king of Persia, 
who was then meditating the invasion of Greece. He was 
subsequently killed in the battle of Marathon, fighting against 
his countrymen. 

\section{Review Questions}


1. For what was Athens distinguished? After the 
death of Codrus, how was the state governed? What is said of the 
office of archon? 

2. What is said of Draco? How were his laws distinguished? What reply did he make when asked why he was so severe? 

3. What is said of Solon? What did he endeavor to accomplish? 
Of his laws, what did he say? 

4. What is said of Solon$'$s system? Of 
the fourth class? What did he institute? What is said of commerce, 
etc.?

5. What was the striking contrast between the Athenians and the 
Lacedaemonians? 

6. What is said of Pisistratus? How did he govern? What is said 
of Hipparchus and Hippias? 



\chapter{From the Invasion of Greece by the Persians to the Peloponne$'$sian War.~B. C. 490 to 431.} 

THE period from the first invasion to the beginning of the 
Peloponnesian war is esteemed the most glorious in the 
history of Greece. The series of victories obtained by the 
inhabitants over the Persians are among the most splendid 
recorded in the annals of the world. The immediate cause 
which led to the invasion of Greece seems to have been to 
avenge the aid which the Athenians gave to the people of 
lo$'$nia, who attempted to throw off the yoke of Persia. 

2. Darius, King of Persia, having reduced the lonians, next 
turned his arms against the Greeks, their allies, with the design 
of making entire conquest of Greece. He despatched heralds 
to each of the Grecian states, demanding earth and water, 
which was an acknowledgment of his supremacy. Thebes 
and several of the other cities submitted to the demand; but 
Athens and Sparta indignantly refused, and, seizing the 
heralds, they cast one into a pit and another into a well, and 
told them to take there their earth and water. 

3. Darius now commenced his hostile attack both by sea 
and land. The first Persian fleet, under tljje command of 
Mardo$'$nius, was wrecked in doubling the promontory of 
Athos, with a loss of no less than three hundred vessels; a 
second, of six hundred sail, ravaged the Grecian islands; 
while an immense army, consisting of one hundred and ten 
thousand men, poured down impetuously on Attica. This 
formidable host was met by the Athenian army under the 
command of Milti$'$ades, on the plains of Mar$'$athon, where 
the Persians were signally defeated and fled with precipitation to their ships. The loss of the Persians amounted to 
six thousand three hundred; while the Athenian army, which 
did not exceed ten thousand men, lost only one hundred and 
ninety-two. A soldier covered with wounds ran to Athens 
with the news, and having only strength sufficient to say, 
``Rejoice! the victory is ours,'' fell down and expired. 

4. Miltiades, the illustrious general by whose valor this 
great victory was gained, received the most inhuman treatment from his ungrateful countrymen. Being accused of 
treason for an unsuccessful attack on the isle of Paros, he 
was condemned to death; this punishment, however, was commuted into a fine of fifty talents.\footnote{About \$50,000.} In consequence of his 
being unable to pay this amount he was cast into prison, 
where he died in a few days of the wounds he received in the 
defence of his country. 

5. The Athenians at this time were divided into two parties, under their respective leaders---Aristi$'$des, the advocate 
of aristocracy, and Themis$'$ tocles, of democracy. Aristides, 
who on account of his integrity was called the Just, through 
the intrigues of his great rival was banished for ten years 
by the Ostracism. It happened, while the people were giving 
their votes for his exile, that a certain citizen, who was unable to write, and who did not know him personally, brought 
him a shell and asked him to write the name of Aristides 
upon it. ``Why, what harm has Aristides ever done you?''
said he. ``No harm at all,'' replied the citizen, ``but I cannot bear to hear him continually called the just.'' Aristides 
smiled, and taking the shell wrote his own name upon it, 
and went into banishment. 

6. On the death of Darius, his son Xerx$'$es, who succeeded 
to the Persian throne, resolved to prosecute the war which 
his father had undertaken against Greece. Having spent four 
years in making the necessary preparations, he collected an 
army, according to Herodotus, numbering over two millions 
of fighting men; and including the women and retinue of attendants, the whole multitude is said to have exceeded five 
millions of persons. His fleet consisted of more than twelve 
hundred galleys of war, besides three thousand transports of 
various kinds. 

7. Having arrived at Mount Athos, he caused a canal, navigable for his largest vessels, to be cut through the isthmus 
which joins that mountain to the continent, and for the conveyance of his army he ordered two bridges of boats to be 
extended across the HeFlespont, at a point where it measures 
seven furlongs in breadth. The first of these bridges was 
destroyed by a tempest, on which account Xerxes, in transports of rage, ordered the sea to be scourged with three 
hundred stripes, and to be chained by casting into it a pair 
of fetters. The bridge being again repaired, the army commenced its march, and occupied seven days and seven nights 
in passing the straits, while those appointed to conduct the 
march lashed the soldiers with whips, in order to quicken 
their speed. 

8. Xerxes having taken a position on an eminence, from 
which he could view the vast assemblage he had collected, 
the plain covered with his troops, and the sea overspread 
with his vessels, at first called himself the most favored of 
mortals. But when he reflected that in the short space of a 
hundred years, not one of the many thousands then before 
him would be alive, he burst into tears, at the instability of 
all human things ! 

9. Most of the smaller cities of Greece submitted at the demand of the Persian monarch; of those which united to oppose him, Athens and Sparta took the lead. The Persian 
army advanced directly towards Athens, bearing down all 
before it until it came to the pass of Thermopylae, on the east 
of Thes$'$saly. On this spot Leon$'$idas, one of the kinrs of 
Sparta, with only six thousand men, had taken his position 
in order to oppose its pro.o-ress. Xerxes having arrived at 
this place, sent a herald to Leonidas, commanding him to deliver up his arms, to whom the noble Spartan replied with 
laconic brevity, ``Come and take them,'' For two days the 
Persians endeavored to force their passage through the defile, 
and were repulsed with great slaughter; but having at length 
discovered a secret path leading to an eminence which overlooked the Grecian camp, and having gained this advantageous post, under the cover of the night, the defence of 
the pass became impossible. 

10. Leonidas, foreseeing certain destruction, dismissed all 
his allies, retaining only three hundred of his countrymen, 
and, in obedience to a law of Sparta, which forbade her soldiers, 
under any circumstances, to flee from an enemy, resolved to 
devote his life for the good of his country. Animated by his 
example, the three hundred Spartans under his command 
determined to abide the issue of the conflict. Leonidas fell 
among the first, bravely contending against the thousands of 
his enemies; of the three hundred heroes, only one escaped to 
bear to Sparta the nev»s, that her patriotic sons had died in 
her defence; and this survivor, after his return, felt himself 
so disgraced at being alive, that he perished by his own hand. 
Aristode$'$mus, another of the band, being absent when the 
battle occurred, was considered so much disgraced by this 
accident that, when he afterwards distinguished himself at 
the battle of Platae$'$a, he was nevertheless deemed unworthy 
of any share of the spoils. A monument was afterwards 
erected on the spot, to commemorate this memorable battle, 
bearing this inscription, written by Simonides : 

``Go, stranger, and to listening Spartans tell, 
That here, obedient to their laws, we fell." 

11. Xerxes having forced the pass of Thermopylae, directed his march towards Athens, laying waste the country 
as he advanced with fire and sword. The Athenians, having 
conveyed their women and children, for safety, to the islands, 
retired to their fleet, leaving their city in the hands of the 
Persians, by whom it was pillaged and burnt. The only resource left to the Greeks was placed in their fleet; therefore 
they immediately commenced preparations for a naval engagement. Their fleet consisted of only three hundred and 
eighty sail, under the command of Themistocles and Aristides, 
while that of the Persians amounted to twelve hundred vessels. The engagement took place in the straits of SaVamis, 
which resulted in the total defeat of the Persian armament. 
Xerxes, who had seated himself upon an eminence, that he 
might behold the engagement, having seen the complete discomfiture of his squadron, fled with precipitation to the shores 
of the Hellespont. But, to his great mortification, he found 
that the bridge of boats which he left had been destroyed by 
a tempest; terrified, however, at the valor displayed by the 
Greeks, his impatience would admit of no delay; he therefore 
crossed the Hellespont in a fishing-boat to his own dominions. 

12. The Persian monarch left Mardonius, with three hundred thousand men, to complete the conquest of Greece. This 
army, early in the following season, was met at Platasa, by 
the combined forces oi Athens and Sparta, consisting of one 
hundred and ten thousand men, under the command of Aristides and Pausanias, and was defeated with tremendous 
slaughter, Mardonius himself being numbered among the 
slain. On the same day the Greeks engaged and destroyed 
the remains of the Persian fleet, at the promontory of 
Myc$'$ale, near Eph$'$esus. The Persian army was now completely destroyed, and Xerxes, having been frustrated in all 
his ambitious views, was soon afterwards assassinated, and 
was succeeded in the Persian throne by his son, Artaxerx$'$es 
Longim$'$anus, BC 464. 

13. At this period, the national character of the Greeks 
was at its highest elevation. The common danger had annihilated all petty jealousies between the states, and had given 
them union as a nation. Encouraged by their late victories, 
they resolved to bid defiance to the Persians; and undertook 
to aid the lonians, who had thrown off the yoke of Persia. 
The combined forces of Sparta and Athens, under the command of Pausan$'$ias and Ci$'$mon, expelled the Persians from 
Thrace, destroyed their fleet on the coast of Pamphylia, took 
the island qf Cyprus, and having reduced and plundered the 
city of Byzan$'$tium, they returned with immense booty. 

14. Pausanias, who had borne so distinguished a part in 
the late war, now became intoxicated with glory and power, 
and aspired to the sovereign dominion of Greece. For this 
purpose he wrote to Xerxes, offering to effect the subjugation of his country, and to hold it under the dominion of 
Persia, on the condition of receiving his daughter in marriage. The treachery was detected before it could be carried 
into execution, and Pausanias, being condemned by the 
Eph$'$ori, took refuge in the temple of Miner $'$va, where the 
sanctity of the place secured him from violence. Being unable to escape from this asylum, he soon perished by hunger. 
Themistocles, the great Athenian commander, being accused 
of participating in the treason of Pausanias, was banished 
from his country by the law of ostracism. The exiled general 
proceeded to Asia, wrote a letter to the Persian monarch, in 
which he said, " I, Themistocles, come to thee, who have 
done thy house most ill of all the Greeks, while I was of 
necessity repelling the invasion of thy father, but yet more 
good, when I was in safety, and his return was endangered." 
He was permitted to live in Persia in great splendor, but 
being- required by Artaxerxes to take up arms against the 
Greeks, rather than sully his former glory, by engaging in a 
war against his native country, although that country had been 
ungrateful towards him, he chose to suffer a voluntary death. 

15. Aristides, after the banishment of Themistocles, directed 
the affairs of Athens, and upon his death, which happened 
shortly afterwards, Gi$'$mon, the son of Miltiades, one of the 
most illustrious statesmen and warriors of Greece, became 
the most prominent man in the republic. He gained two 
important victories over the Persians on the same day, the 
one by sea and the other by land, near the river Eury$'$medon, 
in Asia Minor. But it was the characteristic of the Athenians to treat their most distinguished citizens with ingratitude. Cimon, through the influence of faction, was banished 
by the ostracism, while Per$'$icles, a young man of exalted 
talents and extraordinary eloquence, succeeded in gaining the 
ascendancy at Athens. 

16. Cimon, however, after a banishment of five years, was 
recalled, and, being restored to the command of the army, 
gained several other important victories over the Persians, 
and finally died of a wound he received at the siege of Cic$'$tium, in Cyprus. Shortly after this event the Persian war, 
which had lasted, with some slight intermissions, for about 
fifty years, was brought to a termination. Artaxerxes, weary 
of a war that only brought disgrace upon his arms and 
weakened his resources, sued for peace, which was granted 
on condition that he should give freedom to all the Grecian 
cities in Asia, and that no Persian ship of war should enter 
the Grecian seas. 

17. After the death of Cimon, Pericles rose to the summit 
of power. He governed Athens with almost arbitrary sway 
for nearly forty years. He adorned the city with master-pieces 
of architecture, sculpture, and painting, patronized the arts 
and sciences, celebrated splendid games and festivals, and 
his administration forms an era of splendor and magnificence 
in the history of Greece. In all his public acts he displayed 
the greatest moderation and prudence, and Hie end of all his 
projects seems to have been the glory of his country and the 
happiness of his fellow-citizens. He died of a plague which 
raged at Athens. A little before his death, hearing some of 
his friends speaking of his achievements, he said, ``You have 
forgotten the most glorious action of my life, which is, that 
I never caused a single citizen to put on mourning.''


\section{Review Questions}


1. What is said of this period? What was the immediate cause which led to the invasion of Greece? 

2. What is said of 
Darius? 
How did Athens and Sparta treat the heralds? 

3. What is said of 
the first Persian fleet? What was the number of the second fleet? By 
whom was this host met? What was the loss of the Persians? Of the 
Athenian army? What is said of an Athenian soldier? 

4. What is 
related of Miltiades, the illustrious general? 

6. What is said of Xerxes? What was the number of his array? Of 
his fleet?

7. Having arrived at Mount Athos, what did he cause?
What did he order? How long was the army in passing the straits? 

8. What is now related of Xerxes? 

9. What is said of the Persian 
On this spot who opposed its progress? What reply did he make? 
How long were the Persians stopped?

10. What did Leonidas now 
do? Of the three hundred how many escaped? What inscription was 
afterwards placed upon the monument? 

11. Where did Xerxes now 
march? What is said of the Athenians? 
Who commanded their fleet? What engagement took place? What is 
said of Xerxes? How did he cross the Hellespont? 

12. What did the 
Persian monarch leave? By whom was this army met? and what was 
the issue of the battle? On the same day what took place? What was 
the end of Xerxes? 

13. At this period what is said of the Greeks? 
What did they undertake? What did they effect? 

14. What is said of Pausanias? Where did he take refuge? 
What is related of Themistocles? What was his end? 

15. What is 
said of Aristides and Cimon? After the banishment of Cimon who 
gained the ascendancy at Athens? 

16. Was Cimon again recalled? 
What is said of the Persian war? What were the conditions of peace? 

17, What is said of Pericles? In all his public acts what did he 
display? How did he die? What did he say before his death? 



\chapter{From the Beginning of the Peloponnesian War to 
the Reign of Philip of Macedon--- B. C. 431 to 360.} 

A FEW years before the death of Pericles, the Peloponnesian war began. This long and desperate struggle 
grew out of the ceaseless rivalship between Athens and 
Sparta; and for twenty-seven years, with little intermission, 
it inflicted the deepest calamities upon the Grecian States. 
The orioin of this war seems to have been as follows : The 
inhabitants of Corcy$'$ra, while engaged in a contest with the 
Corin$'$thians, applied for aid to the Athenians, who readily 
granted them assistance; this conduct on the part of the 
latter was deemed a violation of the treaty of the confederate 
states of Peloponnesus, and war was immediately declared 
against Athens. 

2. Sparta, joined by all the Peloponne$'$sian states, except 
Ar$'$gos, which remained neutral, took the lead against the 
Athenians, who had but few allies. The Peloponnesian forces, 
under the command of Archida$'$mus, the king of Sparta, 
amounted to sixty thousand, while the Athenian army did not 
exceed thirty-two thousand, but the fleet of the latter was 
much the superior. During the first year of the war the confederate forces entered Attica, laid waste the country, and 
besieged Athens; in the second year, the city was visited by 
a dreadful plague, which carried off several thousands, and 
among its victims was the renowned Pericles. The pestilence, however, did not arrest the progress of the war, which 
continued to rage with unabated fury. 

3. After the death of Pericles, Cle$'$on grew into power, 
and for a short time directed the Athenian counsels; but he 
was slain at Amphep$'$otis, in a battle with Bras$'$idas, the 
Spartan general, who was also mortally wounded in the same 
engagement. After this event, a treaty of peace was concluded between Athens and Sparta, through the influence of 
Nic$'$ias, who now became the popular leader at Athens. 
Peace, however, was of short duration, war being again declared, through the influence of Alcibi$'$ades, one of the 
greatest of the Athenian generals, and the most accomplished 
orator of his time. 

4. An expedition was next sent against the island of Sicily, 
under the command of Alcibiades and Nicias, but the former, 
being accused of misconduct, was recalled, and the latter was 
defeated and slain. Alcibiades, after some time, was again 
placed at the head of the Athenian army, and gained several 
important victories, but falling a second time into disrepute, 
he was banished from his country, and took refuge in Asia, 
where he died. 

5. Lysan$'$der, the Lacedaemonian general, having defeated 
the Athenian fleet, at jAEgos-Potamos, on the Hellespont, reduced Athens to the last extremity, by blockading the city 
by sea and land. The wretched Athenians were at length 
compelled to accept the most humiliating terms of peace; 
they agreed to demolish their port, to limit their fleet to 
twelve ships, and to undertake for the future no military 
enterprise, but under the command of the Lacedsemonians. 
Thus ended the Peloponnesian war, by the submission of 
Athens and the triumph of Sparta, which now became the 
leading power in Greece, BC 403. 

6. Lysander, after the reduction of Athens, abolished the 
popular government of that state, and established in its place 
an oligarchy, consisting of thirty magistrates, with absolute 
power, who, from their atrocious acts of cruelty, were called 
the Thirty Tyrants. In the space of eight months we are 
told that fifteen hundred citizens fell victims to their avarice 
and vengeance, while many others fled from their country. 
At length ThrasybuIus, aided by a band of patriots, expelled 
the tyrants from the seat of their power, and restored the 
democratic form of government. 

7. An event, which happened about this time, reflected indelible disgrace upon the fickle-minded Athenians, which was 
the persecution and death. of the illustrious philosopher, 
Soc$'$rates, a name at once the glory and the reproach of his 
country. The sophists, whose futile logic he derided and 
exposed, represented him as an enemy to the religion of his 
country, because he attempted to introduce the knowledge 
of a Supreme Being, the Creator and Ruler of the universe, 
and to inculcate the belief in a future state of retribution; 
and being accused, moreover, of corrupting the youth, he was 
condemned by the assembly of Athens to die by poison. 

8. He made his defence in person, with all the manly fortitude of conscious innocence; but the majority of his judges, 
being his personal enemies, determined on his ruin. During 
the forty days of his imprisonment, he conducted himself with 
the greatest dignity; refused to escape when an opportunity 
offered; conversed with his friends on subjects of moral 
philosophy, particularly the immortality of the soul; and 
when the appointed time arrived drank the fatal cup of hem
lock, and died with the utmost composure. After the fatal 
deed was accomplished, the Athenians began to see the sad 
error into which they had fallen. The judges and accusers 
of Socrates were either put to death or banished from the 
city; a brazen statue was erected to his memory, the workmanship of the celebrated Lysip$'$pus. Thus these fickle ancients endeavored to repair, in some degree, the injustice they 
had permitted against the most virtuous of their citizens. 

9. On the death of Darius, the Persian throne was left to 
his son, Artaxerxes II., but his younger brother, Cyrus, attempted to dethrone him, and for that purpose he employed 
about thirteen thousand Grecian troops; but both Cyrus and 
the Grecian commander were slain in a battle, which was 
fought at Gunax$'$a, near Babylon. The remainder of the 
Grecian army, which numbered about ten thousand, under 
the command of Xen$'$ophon, effected a most extraordinary 
retreat, traversing a hostile country of sixteen hundred miles 
in extent, from Babylon to the shores of the Euxine. This 
celebrated return, usually called the retreat of Ten Thousand, 
is beautifully described by Xenophon himself, and is regarded 
as one of the most extraordinary exploits in military history. 

10. The Grecian colonies in Asia having taken part with 
Cyrus, were assisted by the Spartans, under their king 
Agesila$'$us. The Persian monarch, however, by means of 
bribes, induced Athens and other of the Grecian states, jealous of the power of the Lacedoemonians, to enter into a 
league against them. Agesilaus was obliged to return in 
order to protect his own dominions. He defeated the confederate forces in the battle of Corone$'$a, but the Spartan 
fleet was defeated by the Athenians under Conon near 
Cni$'$dus. A treaty of peace was finally concluded, by which 
it was agreed that all the Grecian cities of Asia should belong to Persia, and all others should be independent, with 
the exception of the islands of Lemnos, Scy$'$ros, and Imbros, 
which should remain under the dominion of Athens. 

11. While Athens and Sparta were visibly tending to decline, Thebes emerged from obscurity, and rose for a time to 
a degree of splendor eclipsing all the other states of Greece. 
The Spartans, jealous of its growing prosperity, took advan
tage of some internal dissension and seized upon the citadel. 
Pelop$'$idas, with a number of Thebans, fled for protection to 
Athens, where he planned the deliverance of his country. 
Disguising himself and twelve of his friends as peasants, he 
entered Thebes in the evening, and, joining a patriotic party 
of citizens, they surprised the leaders of the usurpation 
amidst the tumult of a feast and put them all to death; and 
pursuing his success, in conjunction with his friend Epaminon$'$das, who shared with him the glory of the enterprise, 
he finally succeeded in expelling the Lacedagmonian garrison 
from the Theban territory. 

12. A war necessarily ensued between Thebes and Sparta. 
The Theban army, under the command of Pelopidas and 
Epaminondas, gained the memorable battle of Leuctra, in 
which they lost only three hundred men, while the Spartan 
loss amounted to four thousand, together with their king, 
Cleom$'$brotns, who was numbered among the slain. The victorious Thebans, under Epaminondas, joined by many of the 
other Grecian states, entered the territories of Lacedasmon, 
and overran the country with fire and sword. The Spartans, 
who had long boasted that their women had never beheld 
the smoke of an enemy$'$s camp, were mortified to see the 
invaders now encamped within the very sight of their capital. 

13. Having humbled the power of Sparta, the Theban 
commander returned with his victorious army to his native 
city; but the war being again renewed, he gained another 
great victory over the Lacedaemonians and Athenians at the 
battle of Mantine$'$a. In the moment of victory he fell mortally wounded; and with the fall of Epaminondas, who was 
equally eminent as a philosopher, statesman, and general, fell 
the glory of his country. 

The battle of Mantinea was followed by a peace between 
all the Grecian states, by which each city established its independence. 

\section{Review Questions}


1. What was commenced previous to the death of 
Pericles? What was the origin of this war? 

2. What state took the lead against Athens? During the first year 
of the war what took place? During the second?

3. After the death 
of Pericles, who grew into power? What was his end? After this 
event what took place?

4. What expedition was next undertaken? 
What is said of Alcibiades?

5. What is said of Lysander? 
Of the Athenians? What were the terms of peace? How did the 
war end?

6. What did Lysander do? In eight months, how many 
citizens perished? What did Thrasybillus do?

7. What events took 
place at this time? How did the Sophists represent him? Why?

8. How did he make his defence? What is said of him during his imprisonment? 
How did he die? What is said of the Athenians?

9. What did Cyrus attempt? What did the remainder of the army effect after this 
event? 

10. By whom were the Grecian colonies assisted? What did 
the Persians effect by bribes? What is said of Agesilaus? What battles 
were fought? What was agreed by the treaty of peace? 

11. What 
state emerged from obscurity? What did the Spartans do? 
What is said of Pelopidas? 

12. What ensued? What battle did the 
Theban army gain? What was the loss on both sides? What is said 
of the Spartans? 

13. What is said of the Theban commander? What 
followed the battle of Mantinea? 


\chapter{Philip of Macedon. The Exploits and Death of 
Alexander.--- B. C. 360 to 324.}

GREECE was now in the most abject situation. The 
spirit of patriotism seemed utterly lost and military 
glory at an end. Athens, at this time the most prominent 
state, was sunk in luxury and pleasure; yet she was distinguished for her cultivation of literature and the arts. 
Sparta, no less changed from the simplicity of her ancient 
manners, and her power weakened by the new independence 
of the state of Peloponnesus, was in no capacity to attempt 
a recovery of her former greatness. Such was the situation 
of Greece when Philip of Macedon formed the ambitious 
design of bringing the whole country under his dominion. 

2. The kingdom of Macedon had existed upwards of four 
hundred years, but it had not risen to any considerable eminence. It formed no part of the Greek confederacy, and had no 
voice in the Amphictyonic council. The inhabitants boasted 
of the same origin of the Greeks, but were considered by the 
latter as barbarians. Philip, who laid the foundation of the 
Macedo$'$nian Empire, or, as it is sometimes called, the Grecian 
Empire, because Greece in its most extensive sense included 
Macedonia, was sent as a hostage to Thebes, at the age of$'$ 
ten years, where he enjoyed the advantage of an excellent education under Epaminondas. At the age of twenty-four years 
he ascended the throne of Macedon, by the popular voice, in 
violation of the natural right of the nearer heirs to the crown. 

3. Philip was possessed of great military and political 
talents, and was equally distinguished for his consummate 
artifice and address. In order to accomplish the subjugation 
of the Grecian states, he cherished dissensions among them, 
and employed agents in each with a view of having every 
public measure directed to his advantage. The attempt of 
the Pho$'$cians to occupy and cultivate a tract of land consecrated to the Delphian Apollo, gave rise to a contest called 
the Sacred War, in which most of the states of Greece were 
involved. The The$'$bans, Thessa$'$lians, and other states undertook to punish the Phocians, who were supported chiefly 
by Athens and Sparta. 

4. Philip proposed to act as arbitrator of the matter in 
dispute, and procured himself to be elected a member of the 
Amphictyonic Council. Shortly after this event, the Loc$'$rians 
having encroached upon the consecrated ground of Delphi, 
and having refused to obey the order of the Amphictyonic 
Council, Philip was invited to vindicate their authority by 
force of arms. Philip began his hostilities by invading 
Phocis, the key to the territory of Attica. AEs$'$chines, the 
orator, moved by a bribe, endeavored to quiet the alarms of 
the Athenians, by ascribing to him a design only of punishing the sacrilege and vindicating the cause of Apollo. Demos$'$thenes, with the true spirit of a patriot, exposed the 
artful designs of the invader, and, with most animated eloquence, roused his countrymen to a vigorous effort for the 
preservation of their liberties. The attempt, however, was 
unsuccessful; the battle of Cheronae$'$a decided the fate of 
Greece, and subjected all the states to the dominion of the 
king of Macedon, BC Sol. 

5. It was not the policy of the conqueror to treat the 
several states as a vanquished people; they were allowed to 
retain their separate independent governments, while he reserved for himself the direction and control of all national 
measures. Convoking a general council of the states, he 
laid before them his project for the invasion of Persia, and 
was appointed commander-in-chief of the forces of all the 
Grecian states. On the eve of this great enterprise, Philip 
was assassinated by Pausa$'$nias, the captain of his guards, 
while solemnizing the nuptials of his daughter, in the fortyseventh year of his age. The news of the event caused the 
most tumultuous joy among the Athenians, who indulged the 
vain hope of again recovering their liberty. But the visionary prospect was never realized. The spirit of the nation 
was gone, and in all their subsequent revolutions they only 
changed their masters. 

6. On the death of Philip, his son Alexander, surnamed 
the Great, succeeded to the throne of Macedon, at the age of 
twenty years. The young king, having reduced to subjection 
some of the states to the north of Macedon, turned the whole 
power of his arms against the revolted states of Greece. He 
defeated the Thebans with immense slaughter, caused their 
city to be razed to the ground, and thirty thousand of its inhabitants to be sold as slaves. These acts of severity so intimidated the other states of Greece that they immediately 
submitted to his dominion. Alexander then assembled the 
deputies of the Grecian states at Cor$'$inth, and renewed the 
proposal of invading Persia, and was appointed, as his father 
had been, the commander-in-chief of their united forces. 

7. With an army of thirty thousand foot and five thousand 
horse, with the sum of only seventy talents and provisions 
for a single month, he crossed the Hel$'$lespont, and traversing 
Phry$'$gia, proceeded to the site of Troy and visited the tomb 
of Achilles, whom he pronounced the most fortunate of men 
in having Pat$'$rocles for his friend and Ho$'$mer for his panegyrist. Darius Godoma$'$nus, resolving at once to crush the 
youthful hero, met him on the banks of the Grani$'$cus, with 
an army of one hundred thousand foot and twenty thousand 
horse. Here an obstinate battle was fought, in which the 
Persian monarch was defeated with a loss, according to Plu$'$tarch, of twenty-two thousand men, while the Macedonian 
loss was only thirty-four. In this battle, Alexander escaped 
narrowly with his life---being attacked by an officer, who was 
about to cleave his head with a battle-axe, when the blow 
was prevented by Cly$'$tus, who cut off the hand of the officer 
with his cimiter, and thus saved the life of his sovereign. 

8. The success of this battle was important to Alexander, 
as it put him in possession of Sar$'$dis with all its riches. He 
generously gave the citizens their liberty, and permitted them 
to live under their own laws. He soon after took Mile$'$tus, 
Halicarnas$'$sus, and other important places. The next important victory was obtained in the great battle of Issus. 
The Persian army, consisting of six hundred thousand men, 
was defeated with prodigious slaughter, no less than one 
hundred and ten thousand being killed, while the Macedonians numbered only four hundred and fifty among the slain. 

The mother, wife, and two daughters of Darius fell into the 
hands of the conqueror, who treated them with the greatest 
delicacy and respect. Darius, on hearing of the kindness of 
Alexander towards his family, offered for their ransom the 
sum of ten thousand talents---about \$10,000,000---and proposed a treaty of peace and alliance, with the further offer of 
his daughter in marriage, and all the country between tlie 
Euphra$'$tes and the AE$'$gean sea. 

9. When the offer was laid before Alexander's council, Parme$'$nio is reported to have said, ``If I were Alexander, I 
would accept the terms.'' ``And so would I,'' replied Alexander, ``were I Parmenio.'' After this he overran Syria, 
took Damascus, and laid siege to Tyre, which surrendered 
after a noble defence of seven months. On this occasion, the 
conqueror exercised an act of barbarous cruelty by causing 
two thousand citizens of Tyre to be crucified, besides all those 
who were put to the sword or sold into slavery. He then 
directed his march towards Jerusalem, which he entered without opposition. Having taken the city of Gaza, he inhumanly 
sold ten thousand of its inhabitants into slavery, and dragged 
Betis, its illustrious defender, at the wheel of his chariot, in 
imitation of Achilles after the taking of Troy. 

10. Alexander next proceeded to Egypt, which readily submitted to his arms; and, with incredible fatigues, he led his 
army through the deserts of Libya to visit the temple of 
Jupiter- Ammon, and caused himself to be proclaimed the son 
of that deity. On his return he commenced the building 
of the city of Alexandria, afterwards the capital of Lower 
Egypt, and, for a time, one of the greatest commercial cities 
in the world. He is said to have founded twenty other cities 
during the course of his conquests. Returning from Egypt, 
he again received proposals from Darius, who offered to surrender to him the whole of his dominions to the west of the 
Euphrates; but he haughtily rejected the offer, saying, that 
'' the world could no more admit of two masters than of two 
suns." 

11. Having crossed the Euphrates, he was met at the vil
lage of Arbe$'$la by Darius, at the head of seven hundred 
thousand men. A dreadful battle was fought, in which the 
Persians were defeated, with a loss of three hundred thousand 
men, while that of Alexander was only about five hundred. 
This great conflict decided the fate of Persia. Darius first 
escaped to Media and afterwards into Bac$'$tria, where he was 
betrayed by Bessus, the satrap of that province, and murdered; 
and shortly after this event the whole Persian empire submitted to the conqueror. 

12. Alexander now projected the conquest of India, and 
having penetrated beyond the Hydas$'$pes, defeated Po$'$rns, 
the illustrious king of that country. He still continued his 
march to the East; but when he arrived at the banks of the 
Gan$'$ges, his soldiers seeing no end to their toils, refused to 
proceed any further and demanded that they might be permitted to return to their country. Finding it impossible to 
overcome their reluctance, he returned to the In$'$dus, and 
pursuing his course southward by that river, he arrived at 
the ocean, and, sending his fleet to the Persian Gulf, he led 
his army across the desert to Persep$'$olis, which, in a fit of 
frenzy, he ordered to be set on fire. From Persepolis he returned to Babylon, which he chose as the seat of his Asiatic 
empire. Here, giving himself up to every excess, he was 
seized with a violent fever, brought on by extreme intemperance, and thus died Alexander the Great, in the thirtythird year of his age, and thirteenth of his reign, BC 324. 

13. Perceiving that his end was approaching, he raised 
himself upon his elbow and presented his dying hand to his 
soldiers to kiss. Being asked to whom he left his empire, he 
answered, "To the most worthy." Alexander was the most 
renowned hero of antiquity. He possessed talents which 
might have rendered him distinguished as a statesman and a 
benefactor of mankind, but it was to his military exploits 
alone that he is entitled to the surname of Great. In the early 
part of his career he was distinguished for self-government, 
and exhibited many noble and generous traits of character; 
but when intoxicated with his extraordinary success, he gave 
himself up to unbounded indulgence and to deeds of cruelty 
and ingratitude. He caused Parnenio, his most distinguished 
general, who had assisted him in gaining all his victories, to 
be assassinated on mere suspicion. His friend Cly$'$tus, who 
had saved his life in the battle of the Granicus, he struck 
dead upon the spot, because he contradicted him when heated 
with wine. He caused the philosopher Callis$'$tlienes to be 
put to death for refusing to pay him divine honors.* 


\section{Review Questions}

1. What is said now of Greece? Of Athens? Of 
Sparta?

2. How long had the kingdom of Macedon existed? What 
is said of the inhabitants? Of Philip? At what age did he ascend the 
throne of Macedon?

3. What did he possess? What did he cherish? 
What gave rise to the Sacred War? 

4. What did Philip propose? After this event what took place? 
How did he commence hostilities? What is said of AEschines and Demosthenes? What is said of the battle of Cheronaea? 

5. What was 
the policy of the conqueror? Having convoked a council of the states, 
what did he lay before them? On the eve of this enterprise what happened to Philip? What did the news of this event cause among the 
Athenians? 

6. Who succeeded Philip? 
How did he treat the Thebans? Having assembled the deputies of 
the Grecian states, what proposals did he renew? 

7. What was the 
number of his army? Where did he proceed? By whom and where 
was he met? What wa3 the issue of the battle and the loss on both 
sides? In this battle what is said of Alexander? 

8 What places did 
he next take? Where was the next victory obtained? What was the 
number of the Persian army? The number of the slain on both sides? 
Who fell into the hands of the conqueror? How were they treated? 
What did Darius offer for their ransom?

9. When the offer was laid 
before the council, what was said by Parraenio, and what was Alexander's reply? After the siege of Tyre what act of cruelty did he exercise? Having taken the city of Gaza what did he do? 

10. Where did 
he next proceed? On his return what city did he commence? What 
reply did he make to the proposals of Darius?

11. Where was he met 
by Darius? 
What ensued? What was the loss on both sides? What was the fate 
of Darius? 

12. What did Alexander next project? When he arrived 
on the banks of the Ganges what happened? Where did he die? 
What was his age and the length of his reign?

13. Perceiving that 
his end was approaching what did he do? What is said of Alexander? In the early part of his career? When intoxicated with success? 
Whom did he cause to be assassinated? Whom did he strike dead? 


\chapter{From the Death of Alexander to the Subjugation 
of Greece by the Romans. B. C. 324 to 146.} 

ALEXANDER having named no successor, his vast empire was divided into thirty-three governments, and distributed among as many of the principal officers. Hence arose 
a series of intrigues, fierce and sanguinary wars, which resulted in the total extinction of every member of Alexander's 
family, and finally terminated in a new division of the empire 
into four kingdoms, namely, that of Egypt under Ptolemy; 
Macedo$'$nia, including Greece, under Cassan$'$der; Thrace, 
together with Bithyn$'$ia, under Lysima$'$chus; and Syria, 
under Seleu$'$cus. 

2, From the period of Alexander's death, the history of 
the Grecian states, to the time of their subjugation by the 
Romans, presents only a series of uninteresting revolutions. 
When the news of this event reached Athens, Demosthenes 
once more made a noble effort to vindicate the national freedom, and to arouse his countrymen to shake off the yoke of 
Macedon. His counsels so far prevailed that the Greeks 
formed a confederacy for the purpose of recovering their 
liberty. But they were finally defeated by Antip$'$ater, and 
Athens was obliged to purchase a peace by the sacrifice of 
ten of her public speakers, among whom the renowned orator 
Demosthenes was included. But to avoid falling into the 
hands of his enemies, he put an end to his own life by taking 
poison. 

* See Biography of Eminent Personages. 


3. Under the administration of Polysper$'$chon, who succeeded Antipater in the government of Macedon, independence for a short time was restored to the Grecian states. 
Scenes of turbulence were soon renewed among the Athenians; they put to death many of the friends of Antipater, 
and among the rest was the venerable Pho$'$cion, now upwards 
of eighty years of age. He was eminent in his public character and private virtues, and had been forty-five times governor of Athens. To a friend who lamented his fate, he said, 
" This is only what I long expected. It is thus that Athens 
has rewarded her most illustrious citizens." 

Cassander, who succeeded Polysperchon, appointed Demet$'$rius Phale$'$reus governor of Athens. Under his wise 
administration, which continued twelve years, the city enjoyed a considerable degree of prosperity, and the Athenians, 
to testify their gratitude, erected no less than three hundred 
and sixty statues to his memory. 

4. The last effort made to revive the expiring liberty of 
Greece, was the formation of the Achae$'$an League, which 
was a union of twelve of the smaller states for that object. The 
government of this confederacy was committed to Ara$'$tus, a 
young man of eminent abilities, who took the title of praetor. 
He formed the noble design of liberating his country from 
the dominion of Macedon, and establishing the independence 
of all Greece; but the jealousy of some of the principal 
states, particularly of Sparta, rendered the plan abortive. 

Aratus was succeeded by Philopoe$'$men, who triumphed over 
the Spartans and AEto$'$lians, but in an expedition against the 
Messe$'$nians, who had revolted, he was defeated and slain. 
Philopoemen was styled the "last of the Greeks," because 
after him Greece produced no leader worthy of her former 
glory. 

5. The Macedonians having declared war against the AEtolians, the latter applied for aid to the Romans, who had now 
become the most powerful nation in the world. The offer 
was joyfully accepted by the Romans, who had long wished 
for an opportunity of adding this devoted country to their 
dominion. Their army, under the command of Quin$'$tus 
Flami$'$nius, defeated Philip, king of Macedon, and proclaimed 
liberty to all the Grecian states. About thirty years after this 
event, the Romans, under the command of Paulus AEmilius, 
again invaded Greece, in a war with Perse$'$us, the son of 
Philip, who was entirely defeated in the battle of Pyd$'$na, 
and failing into the hands of the conqueror, with all his family, 
he was led captive to Rome, to grace the triumph of the 
general. Macedonia was thus reduced to a Roman province, 
BC 167. 

6. The Romans, already jealous of the power of the Achsaan 
League, endeavored to weaken it by cherishing divisions 
among the states, and sought the earliest opportunity of 
again unsheathing the sword against Greece. At length the 
Spartans, in a contest with the Achaean states, applied for 
assistance to Rome. The Romans, under the command of 
3feteVlus, marched into Greece and gained a complete victory 
over the Achaean army. The consul Mummius completed 
the conquest by taking and destroying the city of Corinth, 
in which the remainder of the Achaean forces had taken refuge. The Achaean constitution was dissolved, and all Greece 
was reduced to a Roman province, under the name of Acha$'$ia, 
BC 146. 

1. In reviewing the history of this extraordinary people, 
we find much to admire and much to condemn. In point of 
genius, taste, learning, patriotism, and valor, the Greeks surpassed all the other nations of antiquity. With regard to 
their forms of government, they were far from corresponding 
in practice with what they expressed in theory. Even in 
the palmiest days of Greece, we look in vain for that beautiful idea presented by a well-regulated commonwealth. The 
condition of the people frequently partook more of servitude 
than of liberty. Slaves formed the great majority of the 
inhabitants of the Grecian states; and bondage being a consequence of the contraction of debt, even by free men, a 
great proportion of these were subject to the tyrannical control of their fellow-citizens. They were perpetually divided 
into factions, and torn by internal dissensions, which finally 
led to the downfall of their liberties. 

8. In pursuing the history of Athens, the mind is forcibly 
struck with the injustice and ingratitude frequently manifested towards the most illustrious of her citizens. Miltiades, Aristides, Themistocles, Phocion, Cimon, and Socrates, 
were all sentenced to death or banishment, yet the Athenians, 
with their characteristic fickleness and inconstancy, did ample justice to their merits, and sought to punish those by 
whom they were accused. The most remarkable circumstance 
which strikes us, in comparing the later with the more early 
period of Grecian history, is the total change in the genius 
and spirit of the people. The ardor of patriotism, the thirst 
for military glory and love of liberty, decline with the rising 
grandeur of the nation; while a taste for the fine arts, a love 
of science and the refinements of luxury are introduced. 


\section{Review Questions}

1. How was the empire divided? What arose? Name 
the four chief empires.

2. From Alexander's death what is said of the 
history of the Grecian states. When the news reached Athens what 
did Demosthenes do? What was his end? 

3. Under the administration of Polysperchon what was said? What 
is said of Phocion? What reply did he make to a friend? Who was 
appointed governor of Athens? What is said of his administration?

4. What was the last effort to revive the hberty of Greece? To whom 
was the government committed? What did he form? Who succeeded 
Aratus? What was he styled?

5. What is said of the Macedonians? 
What was done by their army? 
When did the Romans invade Greece? What is said of Philip? 
What was his fate? 

6. What is said of the Romans? Who completed 
the conquest of Greece? To what was it reduced? 

7. In reviewing 
the history what do we find? What is said of the forms of government? 
Of the people? Of slaves?

8. In pursuing the history of Athens how 
is the mind struck? 
Who were sentenced to death or banishment? What remarkable circumstance strikes us? What declined? 


\chapter{Grecian Antiquities.}


\section{Philosophy.}

Philosophy among the pagan Greeks 
was divided into various sects or schools. Of these, the 
lon$'$ic sect was the most ancient, founded by Tha$'$les, BC 
640. He was eminently distinguished for his knowledge of 
geometry and astronomy, and taught the belief of a first 
cause and overruling Providence, but erroneously supposed 
the Deity to animate the universe, as the soul does the body. 

The Italian, or Pythago$'$rean, sect was founded by Fythag$'$oras, who taught the absurd doctrine of the transmigration 
of souls through diff*erent bodies. He believed the earth to be 
a sphere, the planets to be inhabited, and fixed stars to be 
the suns and centres of other systems. 

The Socrat$'$ic school was founded by Soc$'$rates, who was 
esteemed the wisest and most virtuous of the Greeks, and 
the father of moral philosophy. He taught the belief of a 
First Cause, whose beneficence is equal to his power, the Creator and Ruler of the universe. He inculcated the immor
tality of the soul, and a future state of rewards and punishments. 

The Cynics, a ridiculous sect founded by Antis$'$thenes, and 
supported by Diogenes, condemned knowledge as useless, renounced social enjoyments and conveniences of life, and indulged themselves in scurrility and invective. 

The Academic sect was founded by Pla$'$to, a philosopher 
whose doctrines have had, perhaps, a more extensive influence 
over the minds of mankind than those of any other of the 
ancients. Plato had the most sublime ideas of the Deity and 
his attributes. He incorrectly taught, however, that the 
human soul was a portion of the Divinity, and that this alliance with the Eternal Mind might be improved into actual 
intercourse with the Supreme Being, by abstracting the soul 
from all the corruptions it derives from the body. He gave 
his lectures in the grove of Academus, near Athens. 

The Peripatetic sect was founded by Aristo$'$tle, who established his school in the Lyce$'$um, at Athens. His philosophy was taught in the schools for sixteen hundred years. 

The Skeptical sect was founded by Pijr$'$rho, who stupidly 
inculcated universal doubt as the only true wisdom. There 
was, in his opinion, no essential difference between vice and 
virtue, further than as human compact had discriminated 
them. Tranquillity of mind he considered to be the greatest 
happiness, and this was to be obtained by absolute indifference to all dogmas or opinions. 

The Sto$'$ic sect was founded by Ze$'$no. The Stoics inculcated fortitude of mind, denied that pain is an evil, and endeavored to raise themselves above all the passions and 
feelings of humanity. They taught that virtue consists in 
accommodating the dispositions of the mind to the immutable laws of nature, and vice in opposing these laws. Vice, 
therefore, they regarded as folly, and virtue the only true 
wisdom. 

The Epicu$'$reans, named from Epicu$'$rus, the founder of 
the sect, maintained that the supreme happiness of man consisted in pleasure. 

The principle of all things was a subject of special re
search by the philosophers of Greece. Tha$'$les taught that 
this principle consisted of water; Anaxag$'$oras, of infinite 
air; Herac$'$litus, of fire; Democ$'$ritus, of atoms; Pythag$'$oras, of unity; Pla$'$to, of God, idea, and matter; Aristot$'$le, of 
matter, form, and privation; Ze$'$no, of God and matter; Epicurus, of matter and empty space. 

\section{The Seven Wise Men.}

The seven wise men of Greece 
were Tha$'$les, of Miletus; So$'$lon, of Athens; Bias, of Priene; Chi$'$lo, of Lacedaemon; Pit$'$tacus, of Mitylene; Cleohii$'$lus, of Lindos, and Perian$'$der, of Corinth. Instead of 
Periander, some enumerate My$'$ son, and others Anachar$'$ sis. 

\section{The Council of the Amphio$'$tyons.}

This Council is supposed to have been instituted by Am.phictyon, the son of 
Deucalion, king of Thessaly, at an early period of the history of Greece. It consisted, at first, of twelve deputies, 
from the twelve difl$'$erent cities or states; but the number 
was afterwards increased to thirty. They met twice a year---
in the spring at Delphi, and in the autumn at Therm opylse. 
The objects of this assembly were to unite in strict unity the 
states which were represented; to consult for their mutual 
welfare and defence; to decide all differences between cities; 
and to try offences against the laws of nations. 

\section{Public Games.}

There were four public and solemn games 
in Greece, namely, the Olym$'$pic, Pytli$'$ian, Ne$'$mean, and 
Isth$'$mian. The exercises practised at these games were 
leaping, running, throwing, boxing, and wrestling; also the 
horse and chariot races, and contests between the poets, orators, musicians, philosophers, and artists. 

The Olympic games were instituted by Her$'$cules, in honor 
of Jupiter Olympus, BC 1222 years; they were celebrated 
in the town of Olympia, in the first month of every fifth year, 
and lasted five days. The space between one celebration to 
another was called an Olym$'$piad, by which the Greeks computed their time. The prize bestowed on the victor was a 
crown of olive; yet trifling as was this reward, it was considered as the highest honor, and was sought for with the 
utmost eagerness. The victor was greeted with loud acclamations, and his return home was in the style of a warlike 
conqueror. 

The Pythian games were celebrated every fifth year, in the 
second of every Olympiad, near Delphi, in honor of Apollo. 
The reward of the victors was a crown of laurel. 

The Nemean games were celebrated in the town of Nemea 
every third year. The victors were crowned with parsley. 

The Isthmian games, so called from b$'$eing celebrated on 
the isthmus of Corinth, were instituted in honor of Neptune, 
and observed every third or fifth year. They were held so 
sacred that even a public calamity could not prevent their 
celebration. The victors were rewarded with a garland of 
pine leaves. 

\section{Literature.}

No nation of ancient or modern times surpassed the Greeks in literary taste and genius. In subsequent ages, great advances have been made in science, and in 
some of the branches of polite learning, yet in chaste and 
beautiful composition, in brilliancy of fancy, in harmony of 
periods, in various forms of intellectual efforts, under the 
name of poetry, oratory, and history, they are still unrivalled. 

Poetry in Greece was extremely ancient; it was even cultivated before the introduction of letters. In epic poetry, 
Homer stands unrivalled. In lyric poetry, the names of 
Ana$'$creon, Sap$'$pho, and Fin$'$dar, have attained imperishable fame. 

History did not engage the attention of the Greeks till a 
comparatively late period; but Herod$'$otus, Thticyd$'$ides, and 
Xen$'$ophon will ever be numbered among the greatest masters of narration. 

Oratory was cultivated among the Greeks, particularly the 
Athenians, with the utmost care. The study of eloquence 
formed the principal employment of the young citizens at 
Athens. It was that which opened the way to the highest 
offices, reigned absolute in the assemblies, decided the most 
important affairs of the state, and was an almost unlimited 
power to those who had the talent of oratory in an eminent 
degree. Music was cultivated with great success, and was 
considered an essential part in the education of the youth. 
The ancients ascribed to it wonderful effects; they believed 
it well calculated to calm the passions, soften the manners, 
and even to harmonize nations naturally rude and barbarous. 
Dancing was also cultivated with considerable care and attention. 

\section{Arts.}

In the more useful and necessary arts of life, the 
Greeks were never much distinguished. But in those which 
are termed the fine arts, Greece far surpassed all other nations 
of antiquity; and those specimens which have survived the 
wreck of time are regarded as models for imitation, and are 
acknowledged as standards of excellence, in the judgment of 
the most polished nations of modern times. During the administration of Pericles, which is called the golden age of 
the Grecian arts, architecture, sculpture, and painting were 
carried to the summit of perfection. The architecture consisted of three distinct orders, the Dor$'$ic, the lon$'$ic, and 
the Corin$'$thian. The Doric has a masculine grandeur, and 
an air of strength superior to both the other orders. It is, 
therefore, well adapted to works of great magnitude. Of 
this order is the temple of Theseus, at Athens, built ten 
years after the battle of Marathon. It is almost entire at 
the present day. 

The Ionic is distinguished for its elegance and simplicity, 
the latter quality being essentially requisite in true beauty. 
Of this order were the temple of Apollo, at Miletus, the 
temple of the Delphic oracle, and the temple of Dia$'$na, at 
Eph$'$esus. The Corinthian assumed the highest magnificence by uniting the characters of all the orders. 

In sculpture the Greeks excelled no less than in architecture. Specimens of their skill in this respect are perfect models. The Dying Gladiator, the Venus, and the 
Laoc$'$oon of the Grecian sculptors have an imperishable 
fame. 

In painting, though very few specimens have descended 
to us, they are supposed to have excelled. The productions 
of Zeux$'$is, Apelle$'$us, Timanthes, and others which perished, were highly extolled by the writers of antiquity. 

\section{Private and Domestic Life.}

The dress of the Greeks differed much from that of most of the modern nations. The 
men wore an inner garment called a tunic, over which they 
threw a mantle; their shoes or sandals were fastened under 
the soles of their feet with thongs. The women, particularly at Athens, wore a white tunic, which was closely bound 
with a broad sash, and descended in graceful folds to the 
ground; also a shorter robe, confined round the waist with 
a ribbon, and bordered at the bottom with stripes of various 
colors. Over this they sometimes put on a robe which was 
worn much like the present scarf. In the earlier ages of 
Greece the inhabitants usually wore no covering on their 
heads, but in aftertimes they made use of a kind of hat, 
tied under the chin. The women, however, always had 
their heads covered. The Athenians wore in their hair a 
golden grasshopper, as an emblem of the antiquity of their 
nation, intimating that they sprung from the earth. In 
Sparta the kings, magistrates, and citizens were but little 
distinguished by dress. The military costume was of a red 
color. 

The meals of the Greeks were usually four in number. 
The breakfast was taken about the rising of the sun; the 
next meal at midday; then came the afternoon repast; and, 
lastly, the supper, which was the principal meal. Everything capable of sustaining life was used as food, though 
they were generally fond of jQsh. Water and wine were the 
usual drink. At first they sat upright at their meals; but, 
as luxury prevailed, couches were introduced, on which the 
guests reclined while at table. Marriage among the Greeks 
was only lawful when the consent of the parents or other 
relatives could be obtained. Polygamy was allowed only 
after great calamities, such as war or pestilence. 

The Grecian women seldom appeared in strange company, 
but were confined to the remote parts of the house, into 
which no male visitor was admitted. When they went 
abroad, they wore veils to conceal their faces. It was disreputable, however, for them to appear much abroad. Children were required to maintain their parents in old age; but, 
according to the laws of Solon, parents vfho did not bring 
up their children to some useful employment could not exact a support from them. 

The funerals of the Greeks were attended with many ceremonies, showing that they considered the duties belonging 
to the dead to be of the highest importance. In their view, 
it was the most awful of all imprecations to wish that a 
person might be deprived of funeral honors.\footnote{For the Oracles and Religion of the Greeks, see the chapter on 
Mythology.} 

Of some of the peculiar institutions of Greece, the court 
of the Areop$'$agus and Ostracism were most remarkable. 
The Areopagus, which signifies the Hill of Mars, from the 
place where it was held, was the most distinguished and 
venerable court of justice in ancient times, and took cognizance of crimes, abuses, and innovations, either in religion 
or government. The Areopagites were the guardians of 
education and manners, and inspected the laws. To laugh 
in this assembly was an unpardonable act of levity. 

One of the absurd peculiarities in - the government of 
Athens was the practice of Ostracism. This was a ballot 
of all the citizens, in which each wrote down the name of 
the individual most offensive to him; and he who was 
marked out by the greatest number of votes was banished 
from his country for a specified time, often for a number of 
years. It was not necessary that any crime should be alleged. Neither the property nor the honor of the exile sustained the least injury. By this institution the most flagrant 
injustice was often committed against the most virtuous 
citizens. 

\section{Origin of Tragedy.}

Tragedy owes its origin to the feasts 
of Bac$'$chiis, usually celebrated at the time of the vintage, 
and at first consisted of a few rude comic scenes, intermixed 
with songs in praise of that god. Thes$'$pis, owing to several improvements which he made in tragedy, is generally 
esteemed its inventor, although there were several tragic 


What were children required to do? What is said of funerals? 
What was thought the most awful imprecation? What were some of 
the peculiar institutions? What is said of the Areopagus? Of what 
were thev the guardians? What was deemed an unpardonable act of 
levity? $'$What was the Ostracism? By this institution, what was often 
committed? 
and comic poets before his time. He carried the actors 
about in carts, whereas before they were accustomed to sing 
or recite in the streets, wherever chance led them; he also 
caused their faces to be smeared over with lees of wine, instead of acting- without disguise, as at first; and he introduced a character among the chorus, who, to give the actors 
time to rest, repeated the adventures of some illustrious 
person. The alterations which Thespis made in tragedy 
gave room for AEs$'$chylus to make still further improvements. He was a man of superior genius, and took upon 
himself to reform rather than to create tragedy anew. He 
gave masks to his actors, adorned them with robes and 
trains, and made them wear buskins. Instead of a cart, he 
erected a stage of a moderate elevation, and entirely changed 
their style, which, from being merely burlesque, became 
serious and majestic. But ihe most important and essential 
addition of AEschijlus consisted in the vivacity and spirit of 
the action, sustained by the dialogue of the persons of the 
drama, introduced by him---in the artful working up the 
stronger passions, especially of terror and pity, which, by 
alternately afflicting and agitating the soul with mournful 
and terrible objects, produces a grateful pleasure and delight 
from that very trouble and emotion; and, lastly, in the 
choice of his subjects, which were always great, noble, interesting, and contained within due bounds by the unity of 
time, place, and action. Of the ninety tragedies composed 
by AEschylus, about seven are now in existence. 

AEschylus was in sole possession of the glory of the stage, 
when a young rival made his appearance in the person of 
Soph$'$ocles, to dispute with him the palm. Twenty times he 
obtained the prize of poetry over his competitors. Of one 
hundred and twenty tragedies which he composed, only seven 
are now extant, but these prove him to have carried the drama 
almost to perfection. 

Eurip$'$ides was the contemporary and the great rival of 
Sophocles. But nineteen of his seventy-five tragedies remain.\footnote{For a fuller account of ancient Greece, see Dr. William Smith's 
History of Greece.}


\section{Review Questions}

Philosophy.---What is said of philosophy among the 
Greeks? Who was the founder of the Ionic sect? What is said of him? 
Who was the founder of the Pythagorean sect? What did lie teach? 
Who was the founder of the Socratic sect? What did he teach and inculcate? 


Who founded the Cynic sect? What did he condemn? Who 
founded the Academic sect? What did he teach? Where did he 
give his lectures? Who founded the Peripatetic sect? Who founded 
the Skeptic sect? What did he inculcate? Who was the founder of 
the Stoic sect? What did they inculcate? What did they teach? 
What is said of the Epicureans? What was a subject of special research? 

What were the various opinions of the philosophers on this subject? 

The Seven Wise Men.---Who were the seven wise men of Greece? 

The Council of the Amphictyons.---By whom was it instituted? 
Of what did it consist? Where did they meet? What was the object 
of this assembly? 

Public Games.---What were the four public games? What were 
the exercises? By whom were the Olympic games instituted? How 
often were they celebrated? What was the prize of the victor? What 
is said of it? 

How often were the Pythian games celebrated? In honor of whom? 
What was the reward of the victors? At what place were the Nemean 
celebrated? With what were the victors crowned? Why were the 
Isthmian games so called? What is said of them? What was the 
reward of the victors? 

Literature.---What is said of the Greeks in literary taste? In 
what are they still unrivalled? What is said of poetry? Of Homer? 
Of Anacreon, etc.? What is said of History? What is said of Oratory? The study of Elwjuence? 

Private and Domestic Life.---What was the dress of the men? 
Of the women? What did tlie Athenians wear? What was the 
number of their meals, and when were they taken? What was used? 
How did they sit at their meals? What is said of marriage? What 
was allowed? What is said of the Grecian women? What was disreputable? 

What is said of Music? What did they ascribe to it? What is said 
of Dancing? 

Arts.---in what were the Greeks never greatly distinguished? In 
wliat did they surpass all others? What were the three orders of architecture? What is said of the Doric? How was the Ionic distinxuished? 
What did the Corinthian effect? What is said of sculpture? What have 
an imperishable fame? What is said of painting? 

Origin of Tragedy.---To what does tragedy owe its origin? What 
is said of Thespis? 

How did Thespis carry his actors? What improvement did AEschylus 
make? Of his tragedies, how many remain? Who disputed the palm 
with him? What is said of Euripides? 

\end{document}


